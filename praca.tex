\documentclass[11pt,oneside]{article}
\usepackage[utf8]{inputenc}
\usepackage[polish]{babel}
\usepackage[T1]{fontenc}
\usepackage[paper=A4,pagesize]{typearea}
\usepackage{graphicx}
\usepackage{comment}
\usepackage{multirow}
\usepackage{courier}
\usepackage{float}
\usepackage{array}
\usepackage{amsmath}
\usepackage{url}
\usepackage{rotating}
\graphicspath{ {./images/} }

\begin{document}

\begin{figure}[H]

\centering
\includegraphics[width=1.0\textwidth]{logopjwstk}

\begin{Huge}
\begin{center}
	\texttt{\textbf {KARTA PROJEKTU}}
\end{center}
\end{Huge}

\includegraphics[width=1.0\textwidth]{cheat1}
\includegraphics[width=1.0\textwidth]{cheat2}
\end{figure}

\begin{figure}[H]
		\centering
		\includegraphics[angle=90,origin=c,width=.4\textwidth]{cheat3}
\end{figure}

\newpage
\begin{center}
	\textbf{Oświadczenie autora pracy dyplomowej}
\end{center}

Świadom/a odpowiedzialności prawnej oświadczam, że niniejszą pracę dyplomową w zakresie przeze mnie przedstawionym wykonałem/am samodzielnie i nie zawiera ona treści uzyskanych w sposób niezgodny z obowiązującymi przepisami.

Oświadczam również, że praca w przedstawionym przeze mnie zakresie nie była wcześniej przedmiotem procedur związanych z uzyskaniem tytułu ukończenia studiów wyższych.

Oświadczam ponadto, że niniejsza wersja pracy dyplomowej jest identyczna z załączoną wersją elektroniczną.
\\
\vspace*{\fill}
\begin{flushright}
	{\tiny{\textit{podpis}}\hspace{1.5cm} }\linebreak 
	\tiny{..................................................}
\end{flushright}

\newpage
\vspace*{\fill}
\begin{flushright}
{\Large \bfseries  Panu dr.\ Tadeuszowi Puźniakowskiemu}
\linebreak
{\large za pomoc i cenne wskazówki przy wykonaniu} 
\linebreak
{\large  niniejszej pracy oraz poświęcony czas}
\linebreak
{\large  serdecznie dziękuję}
\end{flushright}


\newpage
\vspace*{\fill}
\begin{flushright}
{\Large \bfseries  Panu dr.\ inż.\ Stanisławowi Szejko}
\linebreak
{\large za poświęcony czas podczas konsultacji}
\linebreak
{\large serdecznie dziękuję}
\end{flushright}

\newpage

\setcounter{tocdepth}{2}
\tableofcontents

\newpage

\section{Wstęp}

Na fali pojawiających się nowych zdecentralizowanych aplikacji,
protokołów komunikacj p2p i algorytmów rozproszonego konsensusu, narodziła się w wśród nas koncepcja systemu wspierającego i nadzorującego szeroko rozumianą wymianę obiektów,
w szczególności dóbr fizycznych, który wpasowywałby się do tego nowego, wspaniałego świata.
Porwani przez tę wizję, poświęcamy tę pracą pierwszemu podstawowemu wydaniu systemu ECMarket,
które będzie wstępem do stworzenia analogu do serwisów aukcyjnych typu Allegro czy Ebay w uniwersum
aplikacji zdecentralizowanych.

\newpage
\section{Opis problemu}

\subsection{Prezentacja problemu}

W związku z rosnącą centralizacją Internetu, problemem cenzury oraz spadkiem zaufania do
instytucji państwa\cite{work:trustPOL,website:trustUSA} jak i korporacji, powstało wiele projektów mających
na celu rozproszenie tej koncentracji władzy, jednak przez długi czas rozwiązania te nie zyskiwały
dużej popularności poza specjalistycznymi dziedzinami.\\

Odwilż w tej kwestii przyniósł protokół Kademila\cite{work:Kademila}, na której oparte zostały sieci wymiany plików takie
jak BitTorrent, czy Kad Network. Ich funkcjonalność ograniczała się do wymiany plików, jednak nie była
pozbawiona wad, takich jak niska żywotność niepopularnych plików, egoistyczne zachowania klientów sieci
(leeching), brak odporności na ataki DDoS (Distributed Denial-of-Service).\\


Przełom nastąpił w 2009 wraz z prezentacją technologi Bitcoin\cite{work:BitcoinPaper}, która oferowała rozwiązanie wymienionych problemów, z którymi borykały się poprzednie rozwiązania. Zastosowanie mechanizmu PoW (Proof-of-Work), jako zabezpieczenia przed atakami DDos i Sybili, użycie nowej struktury -- blockchainu, jako rozproszonej bazy danych, a przede wszystkim zastosowanie zachęty finansowej dla użytkowników przetwarzająchych dane, stworzyło pierwszą tego typu wirtualną walutę odnoszącą sukcesy. Dowodem tego jest osiągnięcie przez Bitcoina pod koniec listopada 2013 wartości ponad 1200 USD/BTC\cite{website:BitcoinPrice}.\\

Te wydarzenia dały początek nowej dziedzinie wiedzy nazwanej krypto-ekonomią, rozpoczynając eksplozję nowych kryptowalut i rozwiązań pochodnych. Jednym z nowych rozwiązań, które pojawiło się na horyzoncie na przełomie 2013 i 2014 jest Ethereum\cite{website:ethereum2018whitepaper}. Rozwiązanie zaproponowane przez Vitalika Buterina, Gavina Wooda oraz Josepha Lubina oferuje rozproszoną platformę obliczeniową z systemem operacyjnym obsługującym inteligentne kontrakty (smart contracts). Podobnie jak Bitcoin oferuje możliwość przesyłania wewnętrznych tokenów (w przypadku Ethereum nazywanych Etherem) między użytkownikami, a dzięki zmodyfikowanemu konsensusowi Nakamoto (Ethereum 2.0 będzie używać innych mechanizmów), daje podobne gwarancje bezpieczeństawa. Najbardziej istotna różnica występuje jednak w systemie skryptów (inteligentnych kontraktów), te oferowane przez Ethereum, w przeciwieństwie do Bitcoina, są kompletne w sensie Turinga i mogą posiadać bardziej skomplikowany stan.

\subsection{Proponowane rozwiązanie}

Dzięki takiemu zestawowi cech, Ethereum stanowi dobrą platformę, do budowania własnych, wyspecjalizowanych systemów, które są bezpieczne i niezależne od potencjalnie nieuczciwych stron trzecich. Szczególnym przypadkiem się tutaj platformy handlowe, które agregują oferty sprzedających, pośredniczą w wykonaniu opłat i dostarczają narzędzia od oceny kupujących i sprzedających. W przypadku scentralizowanych wersji takich platform, bardzo częstym problemem jest faworyzowanie pewnych grup sprzedawców, różne stopy opłat dla różnych uczestników i stronniczy moderatorzy.\\

Mając na uwadze wymienione wcześniej fakty, proponowanym przez nas rozwiązaniem problemów z centralizacją i nieuczciwymi pośrednikami, przynajmniej w sferze platform handlowych, jest stworzenie takiej aplikacji w ekosystemie Ethereum i zapewnienie, że w późniejszych etapach rozwoju, będzie ona rozwijana w ścisłej współpracy ze społecznością sieci. Z tego powodu chcielibyśmy zaprezentować nasze autorskie podejście w postaci projektu ECMarket.


\subsection{Rich picture}

\begin{figure}[H]
	\includegraphics[width=1.0\textwidth]{richpicture}
\end{figure}


\newpage
\subsection{Konkurencja}

ECMarket ma na chwilę obecną kilka wartych uwagi konkurentów, którzy operują w nieco innych, ale pokrywających się z naszym obszarach.
\subsubsection{Konkurencja zdecentralizowana}
\begin{itemize}
	\item \textit{uPort} -- Pozwala użytkownikom tylko zarejestrować własną tożsamość w Ethereum i zarządzać własnymi kluczami i danymi w bezpiecznym portfelu \cite{website:uPort}.
	\item \textit{Ujo Music} -- Handluje tylko utworami muzycznymi, nie posiada również rozbudowanego systemu konktraktów \cite{website:UjoMusic}.
	\item \textit{IDEX} -- Oferuje tylko usługi wymiany tokenów ERC20 \cite{website:IDEX}.
	\item \textit{OpenBazaar} -- Oferuje tylko jeden rozdaj umowy (umowa między dwiema stronami z depozytem)\cite{website:openbazaar}. 
\end{itemize}
\subsubsection{Konkurencja scentralizowana}
\begin{itemize}
    \item \textit{eBay} -- Największy serwis aukcyjny na świecie. Każdy zarejestrowany użytkownik może stworzyć aukcję lub wziąć udział w niej udział. Na eBayu możemy płacić tylko w standardowy sposób, czyli kartą kredytową, debetową lub za pośrednictwem PayPala. Strona nie oferuje żadnych płatności za pomocą kryptowalut. eBay posiada regulamin, w którym jest jasno określone jakie przedmioty mogą być obiektem sprzedaży. Definiuje on również sytuacje, w ktorych administracja podejmie kroki, by zamknąć aukcję. Dodatkowo eBay oferuje biuro obsługi klienta dla każdego, kto potrzebuje pomocy w sprawie oszustw, czy też problemu z płatnościami.
    \item \textit{Allegro} -- Polski odpowiednik amerykańskiej platfromy eBay. Jest to portal, który oprócz standardowych transakcji pomiedzy użytkowikami, pozwala na umieszczanie ofert przez profesjonalnych handlarzy. Allegro posiada system ocen który ma za zadanie weryfikować każdego użytkownika, jednak w trudnych przypadkach również oferuje pomoc w ich rozwiązaniu. Tak samo jak amerykański odpowiednik, nie oferuje płatności za pomocą kryptowalut. 
\end{itemize}
    Nasza oferta stanowi swoiste połączenie i wariację niektórych własności oferowanych przez wyżej wymienionych konkurentów, dlatego sądzimy, że mamy szansę wypromować się na tym rynku.


\newpage
\section{Planowanie}

\subsection{Cele i zakres projektu}

Celem projektu jest dostarczenie odpornej na cenzurę, ogólnodostępnej, otwartej platformy, w ramach której możliwe jest przeprowadzanie kupna oraz sprzedaży dóbr w sposób niezależny od osób trzecich lub w ramach grup handlowych, tworzonych przez użytkowników systemu z określanymi przez nich zasadami.

\subsection{Kontekst projektu}

System będzie docelowo uruchomiony na Testnecie Ethereum i będzie dostarczał rozproszoną, odporną na cenzurę, na ile pozwala na to protokół Ethereum \cite{website:TheDAOHack, website:ethereum2018whitepaper} platformę do handlu. Użytkownicy w pierwszym stadium rozwoju platformy będą się dzielić na deweloperów i klientów. Zadaniem deweloperów będzie dostarczanie kodu i jego deployment oraz ponoszenie kosztów tych operacji. Developerzy będą również beneficjentami oprocentowania transakcji. Klientami nazywamy wszystkich użytkowników, którzy dokonują transakcji, mogą oni zarządzać swoimi portfelami, oraz wchodzić w interakcje z dostępnymi dla nich umowami. Liczba klientów nie jest w żaden sposób ograniczona.

\subsection{Komercjalizacja platformy}
ECMarket stanowi demo technologiczne, które w przyszłych wydaniach miałoby przyjąć formę pełnoprawnego produktu opartego o Open Source. Wymagałoby to uzyskania praw autoriskich od uczelnii, co mamy zamiar zrobić. Naszym celem jest uczłowieczenie technologii Ethereum tak by stała sie ona zrozumiała dla przeciętnego użytkownika.  Kluczową rolę pełni tutaj frontend znacznie zwiększając przystępność dla przeciętnego użytkownika, który niekoniecznie musi być ekspertem z dziedziny IT.  Początkowo mamy zamiar działać na portalu reddit uderzając do osób zainteresowanych technologią Ethereum jak i nieco szerszą rodziną kryptowalut opartych o blockchain w ten sposób pozyskując pierwszych użytkowników. Następnym celem byłyby portale społecznościowe na których staralibyśmy się promować nasze rozwiązanie jak i zalety technologii Ethereum. 

\subsection{Analiza biznesowa}

System dostarczy klientom zestaw gotowych do użycia umów i system ocen (w drugim wydaniu), który będzie dawać nam przewagę nad konkurencją w takich kwestiach jak:
\begin{itemize}
\item Niezależność od walut FIAT -- na platformie płatności odbywać się będą przy pomocy Ether'u (ETH), kryptowaluty której wartość jest niezależna od jakiegokolwiek centralnego organu finansowego. Naszych środków nie da się zamrozić (z pewnymi szczególnymi wyjątkami opisanymi w \ref{sec:ZnanePodatnościKontraktów}) tak jak np.\ w przypadku Peso Argentyńskiego (ARS) w 2001 roku podczas kryzysu finansowego. Nie grozi nam także hiperinflacja, pieniędzy nie da się po prostu dodrukować przez specyfikę pozyskiwania Ether'u (ETH).
\item Niezależność od stron trzecich -- transakcje odbywają się w ramach zdecentralizowanej (kwestia ta jest poruszana w \ref{sec:ProblemCentralizacji}) sieci Ethereum, gdzie nie występuje żaden administrator czy istytucja będąca w stanie zablokować bądź cofnąć naszą transakcję. Daje to poczucie bezpieczeństwa i niezależności wyróżniające nas na tle konkurencyjnych serwisów nieopartych na technologii blockchain, gdzie strona trzecia już występuje. Zwiększa to także bezpieczeństwo naszych danych osobowych zachowując większą anonimowość.
\item Różnorodność umów -- wiele sposobów przeprowadzenia transakcji - Przygotowaliśmy 11 przykładowych modeli umów z czego 2 znajdą się w pierwszym wydaniu systemu. Gwarantuje to dostosowanie systemu do potrzeb użytkowników. Dodatkowo ilość ta mogłaby się powiększyć wraz z powiększeniem bazy użytkowników. 
\item Globalny zasięg ofert -- sieć Ethereum wiąże się z brakiem ograniczenia co do lokalizacji na kuli ziemskiej. Korzystać z niej można z dowolnego zakątku, wystarczy jedynie połączenie z internetem. Ether może nabyć każdy co stanowi jedyny wymóg by korzystać z naszej platformy. Daje to o wiele większą wszechstronnoć w porównadniu do innych platform handlowych ograniczonych jedynie do rynku lokalnego. Konkurować z nami pod tym względem mogliby jedynie giganci tacy jak np.\ Amazon czy eBay.
\item Wiarygodny system ocen -- w naszym systemie nie występuje administrator, to też niezwykle istotne jest stworzenie środowiska promującego bycie uczciwym. Podobne rozwiązania możemy zaobserwować w przeróżnych serwisach aukcyjnych, gdzie stanowią pewnego rodzaju informację dla kupującego o tym czy może ufać sprzedającemu. Nasze rozwiązanie pozwalać będzie na ocenienie drugiej strony umowy -- takowej oceny dokonuje także sprzedający, a nie tylko kupujący. Dodatkowo ocenom tym zostaną nadane odpowiednie wagi tak by zapobiec próbom wpłynięcia na system ocen np.\ poprzez wchodzenie w małe transakcje z dużą ilością kont powiązanych ze sprzedającym. Wagi przyznawane byłyby na podstawie ocen posiadanych przez stronę wystawiającą takową ocenę, a także jej wiarygodności dla społeczności platformy wyrażoną poprzez oceny pozytywne oraz ogólną ilość ocen. Jednakże dodanie funkcjonalności związanych z tą cechą, jest zaplanowane na przyszłe wydania
\item Budowanie nowych rozwiązań -- unikalna cecha wyróżniająca nas na rynku. Jestemy otwarci na społeczność tworząca się wokół naszej platformy w znacznie szerszy sposób niż nieblockchainowa konkurencja. Po wypuszczeniu produktu żyłby on własnym życiem kontrolowany w całości przez społeczność posiadającą możliwość wprowadzania modyfikacji, które mogłyby zostać przez większość zaakceptowane bądź odrzucone co stanowi gwarancję dostosowania ECMarket'u do potrzeb potencjalnego użytkownika. Jednakże dodanie funkcjonalności związanych z tą cechą, jest zaplanowane na przyszłe wydania.
\end{itemize}

\nocite{website:CyfrowaEkonomia}

\subsection{Aspekty społeczne}
Charakterystyczną cechą naszej platformy jest niezmienność początkowych warunków użytkowania bez zgody społeczności. Kontrastuje to z coraz częściej spotykaną polityką dużych firm nieustannie zmieniających regulamin, tak by móc np. wykorzystywać nasze dane w większym zakresie niż często byśmy tego chcieli. Z powodu nieświadomości, lenistwa czy też potrzeby korzystania z danego rozwiązania na takowe zmiany się godzimy. Odpowiedź na tego typu politykę stanowi charakterytyka działania Ethereum na której opiera się nasza platfmorma -- przy jakiejkolwiek zmianie w kodzie, który stanowi w pewnym rozumieniu odpowiednik regulaminu np. przetwarzania naszych danych, musimy wyrazić zgodę na to by korzystać z nowszej wersji. W przeciwnym wypadku posiadamy możliwość zrobienia fork'a i korzystania z wersji, która nam odpowiadała. Sprawia to że przy odpowiedniej świadomości społeczności nieosiągalne są praktyki takie jak np. sprzedaż danych osobowych do firm zewnętrznych w celach marketingowych. Społeczność skupiona wokół platformy nie wyrazi na to zgody i korzystać będzie ze starszej wersji przed zmianami w kodzie. 

Zdecentralizowana specyfika platformy utrudnia prowadzenie nieuczciwych praktyk rynkowych przez twórców serwisu. Dobrze pokazuje to chociażby polityka firmy Google, która to pozycjonowała swoją porównywarkę cen przed ofertą konkurencji w przypadku korzystania z ich wyszukiwarki. Firma została ukarana\cite{website:EuropeanCommission} za tego typu praktykę przez komisję europejską w 2017 roku. Nieco bliższy specyfice naszej platfmormy Allegro -- serwis aukcyjny z rodzimego rynku jest obiektem postępowania Urzędu Ochrony Konkurencji i Konsumentów\cite{website:UOKiK} w sprawie faworyzowania własnego sklepu internetowego kosztem innych sprzedawców poprzez lepszą prezentację swoich ofert. W przypadku naszej platfmormy tego typu praktyki są niemożliwe bez głęboko idących modyfikacji kodu na wejście których wyrazić zgodę musiałaby społeczność korzystająca z platfomy -- w przeciwnym wypadku stracilibyśmy użytkowników. Zależność ta wnosi także gwarancję co do niezmienności regulaminu serwisu, którego odpowiednik w naszym przypadku stanowi kod -- jeśli jakaś zmiana nie przypadnie do gustu użytkownikowi może on korzystać ze starszej wersji robiąc forka. Sprawia to, że wykształca się unikalna relacja na linii użytkownik-twórca, gdzie to ten pierwszy posiada decydujący głos. Użytkownik posiada także możliwość zostania twórcą dzięki docelową naturę platformy -- OpenSource. Ma to zwiększać integrację społeczności z serwisem jak i umożliwić jego szybkie dostosowanie do nowych potrzeb użytkowników. Doskonały, analogiczny przykład stanowi zupełnie inna branża -- gry. Odpowiednik aktywnie rozwijającej platformę społeczności stanowią moderzy przedłużający żywotność gry poprzez dostarczanie za darmo nowej zawartości. Zdarzały się już przypadki, że twórcy zatrudniali taką osobę do pracy\cite{website:TheWashingtonPost} przy jakim projekcie bądź też wydawali rozwiniętego i dopracowanego moda jako pełnoprawny dodatek, czy też modyfikacja przekształcała się w zupełnie nowy produkt co stanowi dowód na skuteczność takiego podejścia.

\newpage
\subsection{Rozważane strategie}

Przy określaniu paradygmatu, metodologii i procesu wytwarzania oprogramowania, które zastosujemy w naszym projekcie rozważaliśmy następujące strategie:

\begin{itemize}
	\item Ewolucyjno-przyrostowa \\
	Cechy:
	\begin{itemize}
		\item nie wymaga pełnej specyfikacji problemu
		\item można regulować intensywność i czas przyrostów
		\item pozwala na zmianę wymagań i dostosowywanie poprzednich przyrostów
		\item funkcjonalności systemu muszą być możliwe do rozbicia na podfunkcjonalności
		\item proces zajmuje dużo czasu
	\end{itemize}
	\item Scrum \\
	Cechy:
	\begin{itemize}
		\item szybka adaptacja do zmian wymagań
		\item wymagana stała komunikacja w zespole
		\item nieduże przyrosty w małych odstępach
		\item wymaga podobnych umiejętności wszystkich członków zespołu
		\item samozarządzanie się zespołu
		\item szybko tworzone oprogramowanie
	\end{itemize}
	\item Prototypowanie \\
	Cechy:
	\begin{itemize}
		\item wspomaga identyfikację wymagań
		\item może przeszkodzić w kompleksowej analizie problemu
		\item duże nakłady czasu na przygotowanie prototypu
		\item logiczna walidacja problemu
		\item najlepiej się sprawdza przy budowaniu UI
	\end{itemize}
\end{itemize}

\nocite{academy:BYT, website:Scrum}

\subsection{Wybór strategii i argumentacja}

Metodologią, którą rozważaliśmy na początku był Scrum, jednak wymóg stałej komunikacji i intensywności pracy, zdyskwalifikował ją dość szybko. Problemem również była bardzo mocno zróżnicowana wśród zespołu znajomość technologii (Ethereum). Wahaliśmy się, czy nie użyć protypowania, jednakże mieliśmy już na samym początku, w pewnym stopniu określone wymagania, więc zdecydowaliśmy się na bardziej znaną nam metodę ewolucyjno-przyrostową.

\subsection{Plan zapewnienia jakości}

Jakość w naszym produkcie jest zapewniona dzięki wykorzystaniu metodyki zwinnej TDD (Test Driven Development). Zdecydowaliśmy się na nią, ponieważ nasz system operuje na kryptowalutach oraz na dobrach każdego użytkownika. Sama idea zdecentralizowanej platformy handlowej jest mało popularnym zagadnieniem, więc aby skłonić użytkowników do zaufania nowej technologii i platformy, musimy dostarczyć wysoką jakość naszego produktu.

\subsection{Udziałowcy}


\begin{table}[H]
\begin{tabular}{|l|m{11cm}|}
\hline
\multicolumn{2}{|l|}{\textit{\textbf{Karta udziałowca}}}                               \\ \hline
\textit{Identyfikator:}  & \textit{UNP 01}                                             \\ \hline
\textit{Nazwa:}          & \textit{PJATK}                                              \\ \hline
\textit{Opis:}           & \textit{PJATK umożliwiło nam wykonanie pracy inżynierskiej} \\ \hline
\textit{Typ udziałowca:} & \textit{Nieożywiony pośredni}                               \\ \hline
\textit{Punkt widzenia:} & \textit{Ekonomiczny}                                       \\ \hline
\textit{Ograniczenia:}   & \textit{PJATK nie może narzucać rozwiązań systemowych}                \\ \hline
\end{tabular}
\end{table}

\begin{table}[H]
\begin{tabular}{|l|m{11cm}|}
\hline
\multicolumn{2}{|l|}{\textit{\textbf{Karta udziałowca}}}                               \\ \hline
\textit{Identyfikator:}  & \textit{UNB 02}                                             \\ \hline
\textit{Nazwa:}          & \textit{Sieć Ethereum}                                              \\ \hline
\textit{Opis:}           & \textit{Zdecentralizowana sieć która pozwala nam wykonywać smart contracty.} \\ \hline
\textit{Typ udziałowca:} & \textit{Nieożywiony bezpośredni}                               \\ \hline
\textit{Punkt widzenia:} & \textit{Ekonomiczny, techniczny}                                       \\ \hline
\textit{Ograniczenia:}   & \textit{Brak}                                          \\ \hline
\end{tabular}
\end{table}

\begin{table}[H]
\begin{tabular}{|l|m{11cm}|}
\hline
\multicolumn{2}{|l|}{\textit{\textbf{Karta udziałowca}}}                               \\ \hline
\textit{Identyfikator:}  & \textit{UOB 01}                                             \\ \hline
\textit{Nazwa:}          & \textit{Developerzy}                                              \\ \hline
\textit{Opis:}           & \textit{Członkowie zespołu wykonującego system ECMarket.} \\ \hline
\textit{Typ udziałowca:} & \textit{Ożywiony bezpośredni}                               \\ \hline
\textit{Punkt widzenia:} & \textit{techniczny, wykonawczy}                                       \\ \hline
\textit{Ograniczenia:}   & \textit{Brak}                                             \\ \hline
\end{tabular}
\end{table}

\begin{table}[H]
\begin{tabular}{|l|m{11cm}|}
\hline
\multicolumn{2}{|l|}{\textit{\textbf{Karta udziałowca}}}                               \\ \hline
\textit{Identyfikator:}  & \textit{UOB 02}                                             \\ \hline
\textit{Nazwa:}          & \textit{Klienci}                                              \\ \hline
\textit{Opis:}           & \textit{Grupa która dzięki systemowi ECMarket kupuje i sprzedaje dobra.} \\ \hline
\textit{Typ udziałowca:} & \textit{Ożywiony bezpośredni}                               \\ \hline
\textit{Punkt widzenia:} & \textit{Operator systemu}                                       \\ \hline
\textit{Ograniczenia:}   & \textit{Klienci nie mogą narzucać wymagań systemowych.}          \\ \hline
\end{tabular}
\end{table}

\begin{table}[H]
\begin{tabular}{|l|m{11cm}|}
\hline
\multicolumn{2}{|l|}{\textit{\textbf{Karta udziałowca}}}                               \\ \hline
\textit{Identyfikator:}  & \textit{UNP 02}                                             \\ \hline
\textit{Nazwa:}          & \textit{Regulacje prawne}                                              \\ \hline
\textit{Opis:}           & \textit{Prawo ma duży wpływ na możliwości naszego systemu.} \\ \hline
\textit{Typ udziałowca:} & \textit{Nieożywiony pośredni}                               \\ \hline
\textit{Punkt widzenia:} & \textit{Prawny}                                       \\ \hline
\textit{Ograniczenia:}   & \textit{Brak}                                               \\ \hline
\end{tabular}
\end{table}


\subsection{Ograniczenia}

Głównym ograniczeniem jest czas na wykonanie projektu, który wynosi niepełny rok.

Każda instrukcja wykonana na Wirtualnej Maszynie Ethereum (EVM), kosztuje pewną ilość etheru, co stanowi ograniczenie jak bardzo skomplikowany kod może być na niej wykonywany\cite{work:ethereum2018yellowpaper} . Ten problem można częściowo rozwiązać delegując wykonanie do klientów niektórych zadań, które nie wymagają wiarygodności lub dla których można skonstruować dowód poprawności obliczeń, którego weryfikacja jest tańsza niż sam kod.

Nasz projekt, by mógł spełnić pokładane w nim nadzieje, będzie potrzebować licencji z rodziny GPL lub pochodnych. Jednak kwestia wydania go pod taką licencją, jest zależna od zgody uczelni.

\subsection{Zagrożenia}

Pełna analiza zagrożeń płynących z użycia technologii Ethereum została przeprowadzona w rozdziale \ref{sec:EkosystemEthereum}. Oprócz nich rozważamy problemy płynące ze specyfiki kontraktów i języka Solidity w rozdziale \ref{sec:ZidentyfikowaneProblemy}

\subsection{Harmonogram prac}

\textbf{Przyrost Pierwszy - szacowany czas ukończenia (25.06.2018r.) }
\renewcommand{\labelenumii}{\textendash}
\begin{enumerate}
	\item Analiza potencjalnych umów do zrealizowania w naszym systemie:
	\begin{enumerate}
		\item analiza umowy A1.1
		\item analiza umowy A1.2
		\item analiza umowy A1.3
		\item analiza umowy A2.1
		\item analiza umowy A2.2
	\end{enumerate}
	\item Propozycja pierwszej architektury systemu
	\item Stworzenie wstępnego harmonogramu 
	\item Rozważanie strategii
	\item Dokument założeń wstępnych
	\item Pierwsza specyfikacja wymagań systemowych
	\item Wstępny diagram przypadków użycia
	\item Nauka obsługi ekosystemu Ethereum
\end{enumerate}

\vspace{6mm}

\textbf{Przyrost drugi - szacowany czas ukończenia (10.10.2018r.)}
\renewcommand{\labelenumii}{\textendash}
	\begin{enumerate}
        \item Implementacja umowy A1.1
        \item Implementacja tokenu ERC20
        \item Implementacja AgreementManagera
        \item Analiza scenariuszy umów:
        \begin{enumerate}
        	\item analiza umowy A3.1
            \item analiza umowy A3.2
            \item analiza umowy A3.3
            \item analiza umowy A3.4
            \item analiza umowy A4.2
            \item analiza umowy A4.3
            \item analiza umowy PA1
            \item analiza umowy PA2
            \item analiza umowy WV1
            \item analiza umowy WV2
        \end{enumerate}
\end{enumerate}
%\newline aby oddzielic przyrost 2 i 3

\newpage
\textbf{Przyrost trzeci - szacowany czas ukończenia (14.01.2019r.)}
\renewcommand{\labelenumii}{\textendash}
\begin{enumerate}
    \item Implementacja umowy A1.2
    \item Implementacja serwisu do wyszukiwania umów:
    \item Implementacja widoków dla systemu:
    \begin{enumerate}    
        \item Prototyp widoku do wyszukiwania umów
        \item Szczegółowy widoku umów
        \item Widok do przeglądania akcji danego adresu portfela
        \item Widok do tworzenia umów
    \end{enumerate}

    \item Rozbudowany routing komponentów
    \item Interakcje z umowami
    \item Pełna implementacja interfejsu użytkownika do wyszukiwania umów
\end{enumerate}

\subsection{Infrastruktura komunikacyjna i dokumentacyjna}

W ramach infrastruktury komunikacyjnej wykorzysztaliśmy platformę \url{https://www.discord.com}, za pomocą której występowała komunikacja między członkami zespołu -- większość pracy została wykonywana zdalnie i był to nasz główny kanał komunikacji.  Do integracji kodu projektu oraz dokumentacji pomiędzy członkami grupy projektowej wykorzystaliśmy system kontroli wersji \url{https://www.github.com}.

\subsection{Podział obowiązków}

\begin{itemize}
	\item Wspólne: \begin{itemize}
		\item analiza scenariuszy umów
	\end{itemize}
	\item Oliver Śliwonik -Basista: \begin{itemize}
		\item analiza ograniczeń i własności Ethereum
		\item przygotowanie wstępnej architektury projektu
		\item omówienie problemu centralizacji Ethereum oraz inncy kryptowalut, czynników wpływających na to zjawisko, jak również potencjalnych rozwiązań tej kwestii
		\item przygotowanie wykazu danych wektorów ataków na kontrakty
		\item zdiagnozowanie wpływu braku możliwości agregacji lub kompozycji kontraktów na sposób projektowania systemów w Ethereum
		\item zdiagnozowanie wpływu modelu pamięci \textit{storage} w języku Solidity na sposób projektowania systemów w Ethereum
		\item rozwiązanie problemu braku kompozycji i agregacji kontraktów w postaci autorskiej architektury maszyny stanowej
		\item rozwiązanie problemu modelu pamięci \textit{storage} w języku Solidity przez skonstruowanie abstrakcji pozwalającej do niej pisać jak do pliku
	\end{itemize}
	\item Paweł Dering: \begin{itemize}
		\item zaimplementowanie umowy A 1.1
		\item pomoc przy implementacji podsystemu zarządzającego umowami
		\item implementacja serwisu do wyszukiwania umów 
		\item stworzenie Virtual Wallet - zastąpione standradem ERC20 w miarę postępu prac
		\item współtworzenie przy implementacji interfejsu graficznego
		\item stworzenie testów do wyszukiwania umów
		\item stworzenie testów przy implementacji umowy A1.1
		\item stworzenie testów do podsystemu zarządzania umowami
	\end{itemize}
	\item Bartosz Rychlewski: \begin{itemize}
		\item stworzenie komponentu panelu bocznego
		\item stworzenie widoku wyszukiwania umów
		\item implementacja routingu w komponentach
		\item stworzenie testów do routingu
		\item udokumentowanie wszystkich tabelek, drugiego przyrostu, proponowanego rozwiązania, planu zapewnienia jakości, infrastruktury komunikacyjnej i dokumentacyjnej
		\item współudokumentowanie opisu klas, raportu końcowego
		\item redagowanie dokumentacji
	\end{itemize}
\newpage
	\item Dominik Dopka: \begin{itemize}
		\item stworzenie komponentu do tworzenia umów
		\item zaprojektowanie i ostylowanie interfejsu graficznego
		\item implementacja routingu w komponentach
		\item stworzenie testów przy implementacji umowy A1.1
		\item stworzenie testów do podsystemu zarządzania umowami
		\item dokonanie analizy biznesowej i społecznej platformy2
	\end{itemize}
\end{itemize}

\newpage
\section{Analiza}

\subsection{Wymagania systemowe}
\subsubsection{Wymagania ogólne i dziedzinowe}

\begin{table}[H]
\begin{tabular}{|l|l|l|l|}
\hline
\multicolumn{4}{|l|}{\textit{\textbf{Karta Wymagania}}}                                                           \\ \hline
\textit{Identyfikator:}        & \textit{WO1}            & \textit{\textbf{Priorytet:}}            & M            \\ \hline
\textit{Nazwa:}                & \multicolumn{3}{l|}{\textit{System ECMarket}}                                    \\ \hline
\textit{Opis:}                 & \multicolumn{3}{m{11cm}|}{\textit{Napisanie kodu dla ECMarket i wypuszczenie go do mainnetu}} \\ \hline
\textit{Wymagania powiązane :} & \multicolumn{3}{l|}{\textit{W02}}                                                   \\ \hline
\end{tabular}
\end{table}

\begin{table}[H]
\begin{tabular}{|l|l|l|l|}
\hline
\multicolumn{4}{|l|}{\textit{\textbf{Karta Wymagania}}}                                                           \\ \hline
\textit{Identyfikator:}        & \textit{WO2}            & \textit{\textbf{Priorytet:}}            & S            \\ \hline
\textit{Nazwa:}                & \multicolumn{3}{l|}{\textit{Podstawowa aplikacja kliencka}}                                    \\ \hline
\textit{Opis:}                 & \multicolumn{3}{m{11cm}|}{\textit{Przygotowanie podstawowej aplikacji klienckiej zapewniający graficzny interfejs do ECMarket}} \\ \hline
\textit{Wymagania powiązane :} & \multicolumn{3}{l|}{\textit{W01}}                                                   \\ \hline
\end{tabular}
\end{table}

\subsubsection{Wymagania funkcjonalne}

\begin{table}[H]
    \begin{tabular}{|l|l|l|l|}
    \hline
    \multicolumn{4}{|l|}{\textit{\textbf{Karta Wymagania}}}                                                           \\ \hline
    \textit{Identyfikator:}        & \textit{F01}            & \textit{\textbf{Priorytet:}}            & M            \\ \hline
    \textit{Nazwa:}                & \multicolumn{3}{l|}{\textit{Dostęp do podstawowych pól umowy}}                                    \\ \hline
    \textit{Opis:}                 & \multicolumn{3}{m{11cm}|}{\textit{Każda umowa stworzona w systemie powinna dawać dostęp do takich podstawowych pól jak: lista uczestników umowy, numer bloku utworzenia, timestamp stworzenia, status umowy}} \\ \hline
    \textit{Kryteria akceptacji :} & \multicolumn{3}{l|}{\textit{Umowy posiadają metody zwracające żądane pola}}     \\ \hline
    \textit{Dane wejściowe :} & \multicolumn{3}{l|}{\textit{Brak}}                                                \\ \hline
    \textit{Warunki początkowe :} & \multicolumn{3}{l|}{\textit{Umowa musi istnieć}}                                                \\ \hline
    \textit{Warunki końcowe :} & \multicolumn{3}{l|}{\textit{Strona wywołująca otrzymuje żądane pole}}                                                \\ \hline
\end{tabular}
\end{table}

\begin{table}[H]
    \begin{tabular}{|l|l|l|l|}
    \hline
    \multicolumn{4}{|l|}{\textit{\textbf{Karta Wymagania}}}                                                           \\ \hline
    \textit{Identyfikator:}        & \textit{F02}            & \textit{\textbf{Priorytet:}}            & M            \\ \hline
    \textit{Nazwa:}                & \multicolumn{3}{l|}{\textit{Możliwość wpłacania, wypłacania i sprawdzenia bilansu etheru}}                                    \\ \hline
    \textit{Opis:}                 & \multicolumn{3}{m{11cm}|}{\textit{Każdy użytkownik sieci ma możliwość wpłacenia dowolnej ilości Etheru do StandardECMTokena oraz wypłacenia dowolnej ilości etheru -- nie większej niż bilans danego użytkownika. Zmiana bilansu pokrywa się jeden-do-jednego z ilością wpłaconego lub wypłaconego przez niego etheru. Jeśli użytkownik nie brał udziału w umowie, ani nie wpłacał etheru jego bilans wynosi 0.  }} \\ \hline
    \textit{Kryteria akceptacji :} & \multicolumn{3}{m{11cm}|}{\textit{Każdy użytkownik może niezależnie od innych wpłacać lub wypłacać ether. Jeśli użytkownik wpłacił pewne x i nie brał udziału w żadnej umowie, może maksymalnie wypłacić x.}}    \\ \hline
    \textit{Dane wejściowe :} & \multicolumn{3}{l|}{\textit{Brak}}                                                \\ \hline
    \textit{Warunki początkowe :} & \multicolumn{3}{l|}{\textit{Posiadanie etheru}}                                                \\ \hline
    \textit{Warunki końcowe :} & \multicolumn{3}{l|}{\textit{Zmiana bilansu}}                                                \\ \hline
    \textit{Sytuacje wyjątkowe :} & \multicolumn{3}{m{11cm}|}{\textit{Jeśli użytkownik żąda wypłaty większej ilości etheru niż posiada go w kontrakcie StandardVirtualToken, zostaje wyrzucony wyjątek.}}                                                \\ \hline
\end{tabular}
\end{table}

\begin{table}[H]
    \begin{tabular}{|l|l|l|l|}
    \hline
    \multicolumn{4}{|l|}{\textit{\textbf{Karta Wymagania}}}                                                           \\ \hline
    \textit{Identyfikator:}        & \textit{F03}            & \textit{\textbf{Priorytet:}}            & M            \\ \hline
    \textit{Nazwa:}                & \multicolumn{3}{l|}{\textit{Możliwość tworzenia umów}}                                    \\ \hline
    \textit{Opis:}                 & \multicolumn{3}{m{11cm}|}{\textit{AgreementManager pozwala na stworzenie umowy z parametrami i typem określonym przez użytkownika. Użytkownik otrzymuje adres stworzonej umowy jako wartość zwrotną z funkcji lub jako wydarzenie o nazwie AgreementCreation z dodatkowym polem zawierającym adres utworzonej umowy. Poprawnie utworzona umowa zostaje zarejestrowana przez AgreementManagera.}} \\ \hline
    \textit{Kryteria akceptacji :} & \multicolumn{3}{m{11cm}|}{\textit{AgreementManager przy poprawnych danych tworzy poprawne umowy. Wydarzenie powinno być wygenerowane, tylko wtedy, gdy tworzenie umowy się powiodło.}}    \\ \hline
    \textit{Dane wejściowe :} & \multicolumn{3}{l|}{\textit{Adres, data stworzenia, data przeterminowania, cena, nazwa, opis}}                                                \\ \hline
    \textit{Warunki początkowe :} & \multicolumn{3}{l|}{\textit{Brak}}                                                \\ \hline
    \textit{Warunki końcowe :} & \multicolumn{3}{m{11cm}|}{\textit{Utworzenie nowej umowy i możliwość zarządzania nią z pomocą innych funkcji AgreementManagera}}                                                \\ \hline
\end{tabular}
\end{table}

\begin{table}[H]
    \begin{tabular}{|l|l|l|l|}
    \hline
    \multicolumn{4}{|l|}{\textit{\textbf{Karta Wymagania}}}                                                           \\ \hline
    \textit{Identyfikator:}        & \textit{F04}            & \textit{\textbf{Priorytet:}}            & M            \\ \hline
    \textit{Nazwa:}                & \multicolumn{3}{l|}{\textit{Możliwość wyszukiwania umów}}                                    \\ \hline
    \textit{Opis:}                 & \multicolumn{3}{m{11cm}|}{\textit{AgreementManager powinien mieć możliwość zwrócenia strony z ostatnio stworzonymi umowami.}} \\ \hline
    \textit{Kryteria akceptacji :} & \multicolumn{3}{m{11cm}|}{\textit{AgreementManager przy poprawnych danych tworzy poprawne umowy. Wydarzenie powinno być wygenerowane, tylko wtedy, gdy tworzenie umowy się powiodło.}}    \\ \hline
    \textit{Dane wejściowe :} & \multicolumn{3}{l|}{\textit{Brak}}                                                \\ \hline
    \textit{Warunki początkowe :} & \multicolumn{3}{l|}{\textit{AgreementManager musi posiadać zarejestrowane umowy}}                                                \\ \hline
    \textit{Warunki końcowe :} & \multicolumn{3}{m{11cm}|}{\textit{Utworzenie nowej umowy i możliwość zarządzania nią z pomocą innych funkcji AgreementManagera}}                                                \\ \hline
    \textit{Sytuacje wyjątkowe :} & \multicolumn{3}{m{11cm}|}{\textit{W przypadku braku jakichkolwiek umów zostaje zwrócona tablica z samymi zerami}} \\ \hline
\end{tabular}
\end{table}

\begin{table}[H]
    \begin{tabular}{|l|l|l|l|}
    \hline
    \multicolumn{4}{|l|}{\textit{\textbf{Karta Wymagania}}}                                                           \\ \hline
    \textit{Identyfikator:}        & \textit{F05}            & \textit{\textbf{Priorytet:}}            & M            \\ \hline
    \textit{Nazwa:}                & \multicolumn{3}{l|}{\textit{Możliwość usunięcia nowo-stworzonej umowy A1.1.}}                                    \\ \hline
    \textit{Opis:}                 & \multicolumn{3}{m{11cm}|}{\textit{Umowy stworzone przez użytkowników mogą być przez nich usuwane}} \\ \hline
    \textit{Kryteria akceptacji :} & \multicolumn{3}{m{11cm}|}{\textit{Tylko wskazana umowa ulega usunięciu}}    \\ \hline
    \textit{Dane wejściowe :} & \multicolumn{3}{l|}{\textit{Brak}}                                                \\ \hline
    \textit{Warunki początkowe :} & \multicolumn{3}{m{11cm}|}{\textit{Umowa jest w stanie New, ważna (nie przekroczyła daty ważności), a stroną wywołująca jest jest twórca}}                                                \\ \hline
    \textit{Warunki końcowe :} & \multicolumn{3}{m{11cm}|}{\textit{Kontrakt umowy zostaje zniszczony, wyrejestrowany z AgreementManagera, a środki wpłacone przez petentów zostają do nich odesłane}}                                                \\ \hline
    \textit{Sytuacje wyjątkowe :} & \multicolumn{3}{m{11cm}|}{\textit{Brak}} \\ \hline
\end{tabular}
\end{table}

\begin{table}[H]
    \begin{tabular}{|l|l|l|l|}
    \hline
    \multicolumn{4}{|l|}{\textit{\textbf{Karta Wymagania}}}                                                           \\ \hline
    \textit{Identyfikator:}        & \textit{F06}            & \textit{\textbf{Priorytet:}}            & M            \\ \hline
    \textit{Nazwa:}                & \multicolumn{3}{l|}{\textit{Możliwość usunięcia przeterminowanej umowy A1.1.}}                                    \\ \hline
    \textit{Opis:}                 & \multicolumn{3}{m{11cm}|}{\textit{Przeterminowane umowy mogą być usunięte przez każdego.}} \\ \hline
    \textit{Kryteria akceptacji :} & \multicolumn{3}{m{11cm}|}{\textit{Tylko wskazana umowa ulega usunięciu}}    \\ \hline
    \textit{Dane wejściowe :} & \multicolumn{3}{l|}{\textit{Brak}}                                                \\ \hline
    \textit{Warunki początkowe :} & \multicolumn{3}{l|}{\textit{Umowa przekroczyła swoją datę ważności}}                                                \\ \hline
    \textit{Warunki końcowe :} & \multicolumn{3}{m{11cm}|}{\textit{Kontrakt umowy zostaje zniszczony, wyrejestrowany z AgreementManagera, a środki wpłacone przez petentów zostają do nich odesłane}}                                                \\ \hline
    \textit{Sytuacje wyjątkowe :} & \multicolumn{3}{m{11cm}|}{\textit{Brak}} \\ \hline
\end{tabular}
\end{table}

\begin{table}[H]
    \begin{tabular}{|l|l|l|l|}
    \hline
    \multicolumn{4}{|l|}{\textit{\textbf{Karta Wymagania}}}                                                           \\ \hline
    \textit{Identyfikator:}        & \textit{F07}            & \textit{\textbf{Priorytet:}}            & M            \\ \hline
    \textit{Nazwa:}                & \multicolumn{3}{l|}{\textit{Możliwość usunięcia zakończonej umowy A1.1.}}                                    \\ \hline
    \textit{Opis:}                 & \multicolumn{3}{m{11cm}|}{\textit{Umowy zakończone mogą być usuwane przez użytkowników}} \\ \hline
    \textit{Kryteria akceptacji :} & \multicolumn{3}{m{11cm}|}{\textit{Tylko wskazana umowa ulega usunięciu}}    \\ \hline
    \textit{Dane wejściowe :} & \multicolumn{3}{l|}{\textit{Umowa A1.1}}                                                \\ \hline
    \textit{Warunki początkowe :} & \multicolumn{3}{m{11cm}|}{\textit{Umowa jest w stanie Done, ważna (nie przekroczyła daty ważności), a stroną wywołująca jest jest twórca}}                                                \\ \hline
    \textit{Warunki końcowe :} & \multicolumn{3}{m{11cm}|}{\textit{Kontrakt umowy zostaje zniszczony, wyrejestrowany z AgreementManagera, a środki wpłacone przez petentów zostają do nich odesłane}}                                                \\ \hline
    \textit{Sytuacje wyjątkowe :} & \multicolumn{3}{m{11cm}|}{\textit{Brak}} \\ \hline
\end{tabular}
\end{table}

\begin{table}[H]
    \begin{tabular}{|l|l|l|l|}
    \hline
    \multicolumn{4}{|l|}{\textit{\textbf{Karta Wymagania}}}                                                           \\ \hline
    \textit{Identyfikator:}        & \textit{F08}            & \textit{\textbf{Priorytet:}}            & M            \\ \hline
    \textit{Nazwa:}                & \multicolumn{3}{l|}{\textit{Wykonanie umowy A1.1}}                                    \\ \hline
    \textit{Opis:}                 & \multicolumn{3}{m{11cm}|}{\textit{Umowa A1.1 posiada funkcje join, accept i conclude. Strony chętne do uczestnictwa w umowie wykonują join jednocześnie przekazując środki określone w polu cena na konto umowy. Właściciel umowy może wykonać accept wskazując z kim chce przeprowadzić transakcję, jednocześnie pobierając środki z konta umowy.}} \\ \hline
    \textit{Kryteria akceptacji :} & \multicolumn{3}{m{11cm}|}{\textit{AgreementManager zarejestrował wykonanie umowy}}    \\ \hline
    \textit{Dane wejściowe :} & \multicolumn{3}{l|}{\textit{Kontrakt został zapoczątkowany}}                                                \\ \hline
    \textit{Warunki początkowe :} & \multicolumn{3}{l|}{\textit{Brak}}                                                \\ \hline
    \textit{Warunki końcowe :} & \multicolumn{3}{m{11cm}|}{\textit{Kontrakt zostaje zakończony, oczekuje na usunięcie przez twórce.}}                                                \\ \hline
    \textit{Sytuacje wyjątkowe :} & \multicolumn{3}{m{11cm}|}{\textit{Brak}} \\ \hline
\end{tabular}
\end{table}

\begin{table}[H]
    \begin{tabular}{|l|l|l|l|}
    \hline
    \multicolumn{4}{|l|}{\textit{\textbf{Karta Wymagania}}}                                                           \\ \hline
    \textit{Identyfikator:}        & \textit{F09}            & \textit{\textbf{Priorytet:}}            & M            \\ \hline
    \textit{Nazwa:}                & \multicolumn{3}{l|}{\textit{Możliwość wypłacania środków z kontraktu}}                                    \\ \hline
    \textit{Opis:}                 & \multicolumn{3}{m{11cm}|}{\textit{Petent powinien mieć możliwość wypłacenia środków}} \\ \hline
    \textit{Kryteria akceptacji :} & \multicolumn{3}{m{11cm}|}{\textit{Brak przecieków środków lub innych problemów podczas wypłaty}}    \\ \hline
    \textit{Dane wejściowe :} & \multicolumn{3}{l|}{\textit{Adres portfela, ilość}}                                                \\ \hline
    \textit{Warunki początkowe :} & \multicolumn{3}{l|}{\textit{Wpłacenie środków do kontraktu.}}                                                \\ \hline
    \textit{Warunki końcowe :} & \multicolumn{3}{m{11cm}|}{\textit{Wypłacone środki}}                                                \\ \hline
    \textit{Sytuacje wyjątkowe :} & \multicolumn{3}{m{11cm}|}{\textit{Brak}} \\ \hline
\end{tabular}
\end{table}

\begin{table}[H]
    \begin{tabular}{|l|l|l|l|}
    \hline
    \multicolumn{4}{|l|}{\textit{\textbf{Karta Wymagania}}}                                                           \\ \hline
    \textit{Identyfikator:}        & \textit{F10}            & \textit{\textbf{Priorytet:}}            & M            \\ \hline
    \textit{Nazwa:}                & \multicolumn{3}{l|}{\textit{Ustawianie czasu ważności umowy przy tworzeniu}}                                    \\ \hline
    \textit{Opis:}                 & \multicolumn{3}{m{11cm}|}{\textit{Użytkownik ma możliwość określenia ile czasu/bloków dana umowa będzie ważna. Przedział dozwolonych czasów ważności jest ograniczony od dołu i od góry i zależny od typu umowy. Po utworzeniu umowy nie jest możliwa modyfikacja tej wartości}} \\ \hline
    \textit{Kryteria akceptacji :} & \multicolumn{3}{m{11cm}|}{\textit{Bezkonfliktowe utworzenie daty kontraktu.}}    \\ \hline
    \textit{Dane wejściowe :} & \multicolumn{3}{l|}{\textit{Ilość bloków}}                                                \\ \hline
    \textit{Warunki początkowe :} & \multicolumn{3}{l|}{\textit{Etap tworzenia kontraktu}}                                                \\ \hline
    \textit{Warunki końcowe :} & \multicolumn{3}{m{11cm}|}{\textit{Została ustawiona ilość bloków do wygaśnięcia umowy}}                                                \\ \hline
    \textit{Sytuacje wyjątkowe :} & \multicolumn{3}{m{11cm}|}{\textit{Brak}} \\ \hline
\end{tabular}
\end{table}

\subsubsection{Wymagania niefunkcjonalne}

\begin{table}[H]
    \begin{tabular}{|l|l|l|l|}
    \hline
    \multicolumn{4}{|l|}{\textit{\textbf{Karta Wymagania}}}                                                           \\ \hline
    \textit{Identyfikator:}        & \textit{NF01}            & \textit{\textbf{Priorytet:}}            & S            \\ \hline
    \textit{Nazwa:}                & \multicolumn{3}{l|}{\textit{Zgodność ze standardem ERC20}}                                    \\ \hline
    \textit{Opis:}                 & \multicolumn{3}{m{11cm}|}{\textit{Token systemowy musi być zgodny ze standardem ERC20. https://eips.ethereum.org/EIPS/eip-20}} \\ \hline
\end{tabular}
\end{table}

\begin{table}[H]
    \begin{tabular}{|l|l|l|l|}
    \hline
    \multicolumn{4}{|l|}{\textit{\textbf{Karta Wymagania}}}                                                           \\ \hline
    \textit{Identyfikator:}        & \textit{NF02}            & \textit{\textbf{Priorytet:}}            & M            \\ \hline
    \textit{Nazwa:}                & \multicolumn{3}{m{11cm}|}{\textit{Wstęp do optymalizacji zużycia gazu}}                                    \\ \hline
    \textit{Opis:}                 & \multicolumn{3}{m{11cm}|}{\textit{Umowy i inne kontrakty powinny być zoptymalizowane tak, aby zużywały jak najmniej gazu, zmniejszając ryzyko niepowodzenia.}} \\ \hline
\end{tabular}
\end{table}

\subsubsection{Wymagania projektowo-wdrożeniowe}

\begin{table}[H]
    \begin{tabular}{|l|l|l|l|}
    \hline
    \multicolumn{4}{|l|}{\textit{\textbf{Karta Wymagania}}}                                                           \\ \hline
    \textit{Identyfikator:}        & \textit{ŚD01}            & \textit{\textbf{Priorytet:}}            & M            \\ \hline
    \textit{Nazwa:}                & \multicolumn{3}{l|}{\textit{Ekosystem Ethereum}}                                    \\ \hline
    \textit{Opis:}                 & \multicolumn{3}{m{11cm}|}{\textit{Wszystkie produkty wygenerowane w czasie trwania projektu opierają się na technologiach pochodzących z ekosystemu Ethereum, takich jak sieć Ethereum, sieć Swarm, sieć Whisper, biblioteki Web3.}} \\ \hline
\end{tabular}
\end{table}

\subsubsection{Wymagania dotyczące procesów wytwarzania}

\begin{table}[H]
    \begin{tabular}{|l|l|l|l|}
    \hline
    \multicolumn{4}{|l|}{\textit{\textbf{Karta Wymagania}}}                                                           \\ \hline
    \textit{Identyfikator:}        & \textit{PW01}            & \textit{\textbf{Status:}}            & M            \\ \hline
    \textit{Nazwa:}                & \multicolumn{3}{l|}{\textit{TDD - Test Driven Development}}                                    \\ \hline
    \textit{Opis:}                 & \multicolumn{3}{m{11cm}|}{\textit{Każda zmiana lub nowa funkcjonalność wprowadzana do systemu, musi być najpierw zamodelowana z pomocą testów. Dopiero po tym testy specyfikują jak docelowy kod ma wyglądać.}} \\ \hline
\end{tabular}
\end{table}

\begin{table}[H]
    \begin{tabular}{|l|l|l|l|}
    \hline
    \multicolumn{4}{|l|}{\textit{\textbf{Karta Wymagania}}}                                                           \\ \hline
    \textit{Identyfikator:}        & \textit{PW02}            & \textit{\textbf{Status:}}            & M            \\ \hline
    \textit{Nazwa:}                & \multicolumn{3}{l|}{\textit{Określony git-flow}}                                    \\ \hline
    \textit{Opis:}                 & \multicolumn{3}{m{11cm}|}{\textit{Każda nowa funkcjonalność lub ich zbiór powinien być opisany w Issue. Branche z implementacją danego Issue powinny mieć postać ‘feature/IS\%numer\_issue\%/\%dodatkowy\_opis\%’. Zmianny do testów i do kodu powinny być zawarte w osobnych commitach. Wiadomość w commitach powinny mieć postać ‘IS\%numer\_issue\% test/code - \%opis\_commitu\%’.}} \\ \hline
    \textit{Kryteria akceptacji :} & \multicolumn{3}{m{11cm}|}{\textit{Brak}}    \\ \hline
\end{tabular}
\end{table}

\subsection{Scenariusze i analiza umów}

\subsubsection{Umowa A1.1}
\begin{figure}[H]
\textbf{Scenariusz:} \\
\begin{table}[H]
    \begin{tabular}{ll}
    \multicolumn{2}{l}{\textit{\textbf{Umowa A1.1 -- umowa prosta}}}                                                           \\ \hline
    0.        & S tworzy
\\ \hline
    1.        & K wpłaca
\\ \hline
    2.        & S potwierdza i wypłaca
\\ \hline
    3.       & K ocenia
\\
	\multicolumn{2}{l}{}
\\        
	2a.        & S odrzuca
\\ \hline   
	3a.        & K wypłaca
\\ \hline     
	4a.        & Koniec
\\                                 
\end{tabular}
\end{table}
\textbf{Legenda:} \\
S -- Sprzedający \\
K -- Kupujący \\ \\
\label{Scenariusz A1.1}
\end{figure}

\newpage
\subsubsection{Umowa A1.2}
\begin{figure}[H]
\textbf{Scenariusz:} \\
\begin{table}[H]
    \begin{tabular}{ll}
    \multicolumn{2}{l}{\textit{\textbf{Umowa A1.2 -- umowa z zaliczką}}}                                                           \\ \hline
    0.        & S tworzy i ustala cenę, zaliczkę oraz czas do wycofania się
\\ \hline
    1.        & K wpłaca cenę
\\ \hline
    2.        & S potwierdza
\\
	\multicolumn{2}{l}{\textit{(teraz >= ostateczny termin wycofania się)}}
\\   
    3.       & S wypłaca i ocenia
\\ \hline
    4.       & K ocenia
\\ 
	\multicolumn{2}{l}{}
\\        
	2a.        & S odrzuca
\\ \hline   
	3a.        & K wypłaca
\\ \hline     
	4a.        & Koniec
\\   
	\multicolumn{2}{l}{}
\\        
	3b.        & K nie potwierdza i żąda zwrotu (cena – zaliczka) jeśli \textit{(teraz < ostateczny termin wycofania się)}
\\ \hline 
	4b.        & S wypłaca i ocenia
\\ \hline     
	5b.        & Koniec
\\                                
\end{tabular}
\end{table}
\textbf{Legenda:} \\
S -- Sprzedający \\
K -- Kupujący \\ \\
\label{Scenariusz A1.2}
\end{figure}

\newpage


\subsubsection{Umowa A1.3}
\begin{figure}[H]
\textbf{Scenariusz:} \\
\begin{table}[H]
    \begin{tabular}{ll}
    \multicolumn{2}{l}{\textit{\textbf{Umowa A1.3 -- umowa z kaucjami}}}                                                           \\ \hline
    0.        & S tworzy
\\ \hline
    1.        & K wpłaca k1 + cena
\\ \hline
    2.        & S wpłaca k2
\\ \hline
    3.        & K potwierdza, ocenia i wypłaca k1
\\ \hline
    4.        & S wypłaca k2 + cena i ocenia
\\ 
	\multicolumn{2}{l}{}
\\        
	2a.        & S odrzuca
\\ \hline   
	3a.        & K wypłaca k1 + cena
\\ \hline     
	4a.        & Koniec
\\   
	\multicolumn{2}{l}{}
\\        
	3b.        & K odrzuca -- nowa\_cena
\\ \hline 
	4ba.        & S ugoda -- wypłaca nowa\_cena + k2 i ocenia
\\ \hline     
	5ba.        & K wypłata cena -- cena\_nowa + k1 i ocenia
\\ \hline  
	6ba.        & Koniec
\\   
	\multicolumn{2}{l}{}
\\        
	4bb.        & S odrzuca -- nowa\_cena\_2
\\ \hline     
	5bb.        & K ugoda -- wypłaca cena - nowa\_cena\_2 + k1 i ocenia
\\ \hline  
	6bb.        & S wypłaca nowa\_cena\_2 + k2 i ocenia   
\\ \hline  
	7bb.		& Koniec     
\\   
	\multicolumn{2}{l}{}
\\   
	\multicolumn{2}{l}{\textit{(liczba odrzuceń > max liczba odrzuceń)}}
\\                
    3+l. odrzuceń.b.*.        & S/K odrzuca -- k1+k2 zostają zabrane przez system
\\ \hline  
	4+l. odrzuceń.b.*.        & S i K oceniają. S wypłaca cenę  
\\ \hline  
	5+l. odrzuceń.b.*.			& Koniec     
\\ 
\end{tabular}
\end{table}
\textbf{Legenda:} \\
S -- Sprzedający \\
K -- Kupujący \\
k1, k2 -- kaucje\\ \\
\label{Scenariusz A1.3}
\end{figure}

\newpage
\subsubsection{Umowa A2.1}
\begin{figure}[H]
\textbf{Abstrakt}\\
Umowa prostej, jawnej aukcji -- podobnie jak umowa A1.1 wymaga wpłacenia środków przez
klienta (dalej nazywanego K[numer]) na portfel umowy, w celu jej zawarcia (może być dodana
opcja zatwierdzenia K przy pierwszej wpłacie). W przeciwieństwie do niej na początku
następuje sekwencja licytacji. W jej trakcie  K1-Km obserwują kwoty wpłacane przez innych
uczestników i przebijają ostatnią najwyższą ofertę. Faza licytacji kończy się pod wpływem decyzji
sprzedającego (dalej nazywany S) lub innego zdarzenia [wymagane jest określenie jakie
wydarzenia mogą zakończyć aukcję i jakie potencjalne pod-umowy mogą generować].\\ \\
\textbf{Scenariusz:} \\
\begin{table}[H]
    \begin{tabular}{ll}
    \multicolumn{2}{l}{\textit{\textbf{Umowa A2.1 -- aukcja prosta jawna}}}                                                           \\ \hline
    0.        & S tworzy umowę
\\ \hline
    1.        & K1 wpłaca kwotę k1
\\ \hline
    2.        & \textit{opcjonalnie S zatwierdza udział K1}
\\ \hline
    3.        & Aukcja się kończy -- S wypłaca k1
\\ \hline
    4.        & K1 ocenia S
\\ 
	\multicolumn{2}{l}{}
\\        
	2a.        & K2 wpłaca kwotę k2 (k2 > k1)
\\ \hline   
	3a.        &\textit{opcjonalnie S zatwierdza udział K2}
\\ \hline     
	4a.        & K1 wypłaca k1
\\   
	\multicolumn{2}{l}{}
\\        
	Na.        & Km wpłaca kwotę kN (kN > k(N-1))
\\ \hline 
	(N+1)a.        & Aukcja się kończy -- S wypłaca kN
\\ \hline     
	5ba.        & Km ocenia S
\\ 
	\multicolumn{2}{l}{}
\\        
	2b.        & \textit{Opcjonalne -- S odrzuca K1}
\\ \hline     
	3b.        & \textit{Opcjonalne -- K1 wypłaca k1}
\\ 
\end{tabular}
\end{table}
\textbf{Legenda:} \\
K1, K2, …, Km – klienci biorący udział w aukcji (m <= N) \\
S – sprzedawca \\ \\
Zatwierdzanie powinno nastąpić tylko raz przy pierwszej wpłacie. W przeciwnym wypadku sprzedający może mieć za dużą kontrolę nad aukcją (potencjalne skutki wymagają weryfikacji). Aukcja się kończy, pod wpływem decyzji S lub innego zdarzenia \\ \\
\label{Scenariusz A2.1}
\end{figure}
\textbf{Analiza}\\
Jawna aukcja posiada tę zaletę, że dają możliwość K1-Km między sobą konkurować, a
przede wszystkim obserwować jak kształtuje się ostateczna cena, jednak tylko wtedy, gdy
sprzedawca jest uczciwy. Nieuczciwy sprzedawca nie może bezpośrednio ingerować w przebieg
aukcji, aczkolwiek z pomocą multi-konta lub wspólników może windować cenę, nieproporcjonalnie do
wartości oferowanego produktu. Jedyną ochroną w tej sytuacji dla Kn jest system ocen,
pozwalający sprawdzić sprzedawcę przed zawiązaniem umowy lub go ukarać po jej zakończeniu.

Przy ocenianiu ,,sprawiedliwości'' transakcji należy mieć na uwadze, że cała idea
aukcji daje sprzedającemu przewagę, tzn. klienci konkurują między sobą
oferując wyższe kwoty za towar. Ten problem może być rozwiązany z pomocą grup handlowych,
które mogą kontrolować i narzucać dodatkowe zasady na przebieg umowy i jej kontrahentów.

Rola złośliwych użytkowników jest ograniczona, głównie przez ich portfel (windowanie ceny wymaga posiadania
funduszy, poza tym troll ryzykuje ,,wygranie'' aukcji) i jest tym bardziej ograniczona im jest
większa cena wejścia do aukcji, a pozostali gracze dysponują dużą ilością środków. Jednak nie
eliminuje to całkowicie problemu i prosta aukcja jawna w obecnej postaci może nie być
odpowiednia do transakcji, które wymagają dużej odporności na ataki zamożnych adwersarzy.

\newpage
\subsubsection{Umowa A2.2}
\textbf{Abstrakt}\\
Umowa niejawnej prostej aukcji składa się z trzech faz: licytacja, odsłanianie, wypłaty. W
trakcie licytacji klienci (oznaczeni dalej jako K1-KM) wpłacają depozyty k1-kM (większe równe od
opcjonalnej minimalnej ceny) razem z kryptograficznymi haszami h1-hM. Kolejność w jakiej są
dokonywane wpłaty jest dowolna. Wpłaty mogą wymagać opcjonalnego zatwierdzenia przez
sprzedającego. Faza licytacji jest kończona predefiniowanym zdarzeniem [wymagane jest
określenie jakie wydarzenia mogą zakończyć aukcję i jakie potencjalne pod-umowy mogą
generować], np. decyzją sprzedającego (nazywanego dalej S).

Kolejnym krokiem jest odsłanianie, w trakcie którego K1-KM wysyłają p – faktyczną
kwotę, f – znacznik prawdziwości i s – sól. Na podstawie tych wartości jest wyliczany
kryptograficzny hasz – hasz(p.f.s) i jest on porównywany z wysłaną poprzednio wartością h.
Wpłaty spełniające równość h == hasz(p.f.s), są oznaczane jako poprawne, pozostałe jako
niepoprawne. Wpłaty spełniające dodatkowo warunek f \& k >= p (1.) są nominowane do udziału w
licytacji. Ta faza, podobnie jak poprzednia, kończy się pod wpływem wydarzenia [wymagane jest
określenie jakie wydarzenia mogą zakończyć aukcję i jakie potencjalne pod-umowy mogą
generować].

Ostatnim etapem są wypłaty, w trakcie których:

\begin{enumerate}
	\item Poprawna wpłata spełniająca 1. z najwyższym p, należąca do pewnego KI, jest wypłacana S
	\item KI wypłaca kI – pI i ocenia S
	\item Poprawne wpłaty są wypłacane przez odpowiednich K
	\item Niepoprawne wpłaty są rekwirowane przez system
\end{enumerate}
Faza wypłat jest kończona wraz z wypłaceniem wszystkich poprawnych wpłat. \newpage
\begin{figure}[H]
\textbf{Scenariusz:} \\
\begin{table}[H]
    \begin{tabular}{ll}
    \multicolumn{2}{l}{\textit{\textbf{Umowa A2.2 --  aukcja prosta niejawna}}}                                                           \\ \hline
    0.        & S tworzy umowę
\\ \hline
    1. async1       & K1 wpłaca depozyt k1 i wysyła kryptograficzny hasz h1
\\ \hline
    2. async1        & \textit{opcjonalnie S zatwierdza udział K1 przy pierwszej wpłacie}
\\ \hline
    3. async2        & K2 wpłaca depozyt k2 i wysyła kryptograficzny hasz h2
\\ \hline
    4. async2        & \textit{opcjonalnie S zatwierdza udział K2 przy pierwszej wpłacie} 
\\ 
	\multicolumn{2}{l}{}
\\        
	N. async N        & KM wpłaca depozyt kM i wysyła kryptograficzny hasz hM 
\\ \hline   
	N+1. async N        & \textit{opcjonalnie S zatwierdza udział KM przy pierwszej wpłacie}
\\ \hline     
	N+2.        & Faza licytacji się kończy 
\\    \hline   
	N+3. async1         & K1 wysyła w postaci jawnej wartość p1, sól s1 i znacznik prawdziwości f1
\\ \hline     
	N+4. async2         & K2 wysyła w postaci jawnej wartość p2, sól s2 i znacznik prawdziwości f2

\\   
	\multicolumn{2}{l}{}
\\        
	N+M. async M .        & KM wysyła w postaci jawnej wartość pM, sól sM i znacznik prawdziwości fM
\\ \hline 
	N+M+1.        & Koniec aukcji – S wypłaca środki w ilości pI = max(S), gdzie
\\ 
			&  S = \{pK: 1=<K=<N \& wysłane(pK) \& fK \& pK <= kK \& hK ==
\\ 
			& ==  hasz(pK.fK.sK)\}
\\ \hline     
	N+M+2.        &  KI ocenia S i wypłaca kI – pI 
\\ \hline  
	N+M+3.        & Klienci ze zbioru
\\ 
			& R = \{KK: 1=<K=<N \& wysłane(pK) \& hK ==
\\ 
			& == hash(pK.fK.sK)\} wypłacają wpłacone k 
\\   \hline  
	N+M+4.        & Klienci spoza zbioru R tracą k na rzecz systemu
\\ 
\end{tabular}
\end{table}
\textbf{Legenda:} \\
K1-KM – klienci \\
async[numer] – sąsiadujące kroki oznaczone async o różnych numerach wykonują się w dowolnej kolejności \\
Klienci mogą dokonać więcej niż jednej wpłaty \\ \\
\label{Scenariusz A2.2}
\end{figure}

\textbf{Analiza}\\
W aukcji niejawnej sprzedawca i klienci mogą jedynie dowiedzieć się jaki jest górny limit
potencjalnych ofert, bez gwarancji, że wszystkie będą brać udział w wyznaczaniu ostatecznej ceny.
Taki system utrudnia konkurowanie między K, ale też ogranicza nieuczciwe praktyki ze strony S. K
muszą polegać głównie na własnej ocenie produktu i swoich możliwości przy składaniu propozycji
ceny. S nie zna prawdziwych kwot obstawianych przez K, więc nie ma jak ich skutecznie
windować. Jedyne co może zrobić to podwyższać potencjalny górny limit ofert, co nie gwarantuje,
że K podniosą prawdziwe stawki.

Działania złośliwych użytkowników w tym rodzaju aukcji nie są szkodliwe, wręcz przeciwnie, mogą być
pożądane, np. K wystawiają fałszywe oferty, by wystraszyć innych kupców.

Klienci w fazie odsłaniania mają już wgląd do szczegółów ofert innych K, którzy ujawnili o
nich dane, co może rodzić chęć do nie ujawniania części swoich ofert, np. pewien KX złożył dwie
propozycje, jedna opiewająca na relatywnie niską kwotę, a drugą na znacznie większą. Po
odczekaniu, aż pozostali uczestnicy aukcji ujawnią swoje oferty, odkrył, że pierwsza niższa jest
wystarczająca do zwycięstwa, co może zrodzić chęć nie ujawnienia drugiej mniej opłacalnej. 
Takie działanie kłóci się z ideą tej umowy i by zniechęcać do takich działań, wpłaty które nie
zostały poprawnie odsłonięte, będą rekwirowane przez system.

Problemem do dalszej analizy jest dokładne określenie ograniczeń na zdarzenie kończące
fazę odsłaniania. Jest możliwe, że złośliwy S lub źle skonfigurowane zdarzenie doprowadzi do
sytuacji, gdzie większość K nie zdąży odsłonić swoich ofert, przez co zostaną one pochłonięte przez
system. Można to rozwiązać nakładając na sztywno dodatkowe ograniczenie czasowe, ale dokłada
to kolejny szczegół implementacyjny na wyższym poziomie abstrakcji.

\newpage
\subsubsection{Umowa A3.1}
\begin{figure}[H]
\textbf{Scenariusz:} \\
\begin{table}[H]
    \begin{tabular}{ll}
    \multicolumn{2}{l}{\textit{\textbf{Umowa A3.1 --  prosta umowa z świadkami (pasywna)}}}                                                           \\ \hline
    0.        & S tworzy umowę
\\ \hline
    1.       & K zawiera umowę i wpłaca cenę – zgłasza zbiór ŚK 2t+1 świadków 
\\ \hline
    2.        & S zatwierdza K i zgłasza zbiór SŚ 2t+1 świadków, taki, że ŚK  n ŚS =/= O
\\ \hline
    3.        & K potwierdza poprawność 
\\ \hline
    4.        &  S wypłaca i ocenia K 
\\ \hline
    5.        & K ocenia S
\\  \hline
            & KONIEC
\\ 
	\multicolumn{2}{l}{}
\\        
	3a.        & K nie potwierdza  
\\ \hline   
	4a.        & ŚK U ŚS oraz K i S głosują na K lub S – procedura głosowania 
\\ \hline     
	5a.        & Większość za K – K wypłaca i ocenia S i ŚS 
\\    \hline   
	6a.         & S ocenia ŚK i K
\\    
	\multicolumn{2}{l}{}
\\        
	5aa.        & Większość za S – S wypłaca i ocenia K i ŚK 
\\ \hline 
	6aa.        & K ocenia ŚS i S
\\ 
	\multicolumn{2}{l}{}
\\        
		     & if(nieparzysta ilość nie zagłosuje i jest remis) 
\\ \hline
	4ab.        & S wypłaca i ocenia ŚS 
\\ \hline 
	5ab.        & K ocenia S i ŚK
\\ 
	\multicolumn{2}{l}{}
\\        
	7a.	     & ŚK U ŚS (ci co zagłosowali) ocenia S i K (wypłacają prowizje)
\\
\end{tabular}
\end{table}
\label{Scenariusz A3.1}
\end{figure}

\newpage
\subsubsection{Umowa A3.2}
\begin{figure}[H]
\textbf{Scenariusz:} \\
\begin{table}[H]
    \begin{tabular}{ll}
    \multicolumn{2}{l}{\textit{\textbf{Umowa A3.2 --  prosta umowa z losowaniem świadków (pasywna)}}}                                                           \\ \hline
    0.        & S tworzy umowę i zgłasza swoją połowę zbioru świadków 
\\ \hline
    1.        &  K zawiera umowę, wpłaca cenę
\\ \hline
		& i zgłasza swoją połowę zbioru świadków 
\\ \hline
    2.        &  S zatwierdza K 
\\ \hline
    3.        &  K potwierdza poprawność 
\\ \hline
    4.        & S wypłaca i ocenia K
\\ \hline
    5.	    &  K ocenia S 
\\ \hline
	    & KONIEC
\\
	\multicolumn{2}{l}{}
\\        
	3a.        & K nie potwierdza 
\\ \hline   
	4a.        & System losuje zbiór świadków Ś 2t+1 z sumy zgłoszonych zbiorów 
\\ \hline     
	5a.        & Ś oraz K i S głosują na K lub S – procedura głosowania 
\\   \hline     
	6a.        & Większość za K – K wypłaca i ocenia S i Ś
\\   \hline     
	7a.        & S ocenia Ś i K
\\   
	\multicolumn{2}{l}{}
\\        
	6ab.        & Większość za S – S wypłaca i ocenia K i Ś 
\\ \hline 
	7ab.        &  K ocenia Ś i S
\\
	\multicolumn{2}{l}{}
\\        
	       & if(nieparzysta ilość nie zagłosuje i jest remis) 
\\ \hline     
	5ac.        &  S wypłaca i ocenia Ś
\\ \hline  
	6ac.        &  K ocenia S i Ś 
\\  
	\multicolumn{2}{l}{}
\\   
	8a.		& Ś (ci co zagłosowali) ocenia S i K (wypłacają prowizje)     
\\   
\end{tabular}
\end{table}
\label{Scenariusz A3.2}
\end{figure}

\newpage
\subsubsection{Umowa A3.3}
\begin{figure}[H]
\textbf{Scenariusz:} \\
\begin{table}[H]
    \begin{tabular}{ll}
    \multicolumn{2}{l}{\textit{\textbf{Umowa A3.3 --  umowa z zaliczką i świadkami (pasywna)}}}                                                           \\ \hline
    0.        & S tworzy umowę oraz ustala cenę i zaliczkę 
\\ \hline
    1.        &   K zawiera, wpłaca cenę i zgłasza zbiór ŚK 2t+1 świadków 
\\ \hline
    2.        & S zatwierdza K i zgłasza zbiór SŚ 2t+1 świadków, taki że ŚK n ŚS =/= O
\\ \hline
    3.        & K potwierdza poprawność 
\\ \hline
    4.        & S wypłaca i ocenia K
\\ \hline
    5.	    &  K ocenia S 
\\ \hline
	    & KONIEC
\\
	\multicolumn{2}{l}{}
\\        
	3a.        &  K nie potwierdza i żąda zwrotu (cena – zaliczka) jeśli 
\\ \hline
		  & (teraz < ostateczny termin wycofania się) 
\\ \hline   
	4a.        &  S wypłaca zaliczkę i ocenia K
\\ \hline     
	5a.        &  K wypłaca (cena – zaliczka) 
\\   \hline     
	       & KONIEC
\\   
	\multicolumn{2}{l}{}
\\        
	3b.        & K nie potwierdza i żąda zwrotu całej kwoty 
\\ \hline 
	4b.        & ŚK U ŚS oraz K i S głosują na K lub S – procedura głosowania 
\\ \hline 
	5b.        &   Większość za K – K wypłaca i ocenia S i ŚS 
\\ \hline 
	6b.        &  S ocenia ŚK i K
\\
	\multicolumn{2}{l}{}
\\        
	5ba.        & Większość za S – S wypłaca i ocenia K i ŚK 
\\ \hline  
	6ba.        &   K ocenia ŚS i S
\\  
	\multicolumn{2}{l}{}
\\   
			& if(nieparzysta ilość nie zagłosuje i jest remis) 
\\ \hline     
	4bb.		& S wypłaca cenę i ocenia ŚS
\\   \hline  
	5bb.        & K ocenia S i ŚK 
\\  \hline  
	7b.        &   ŚK U ŚS (ci co zagłosowali) ocenia S i K (wypłacają prowizje)
\\  
\end{tabular}
\end{table}
\label{Scenariusz A3.3}
\end{figure}

\newpage
\subsubsection{Umowa A3.4}
\begin{figure}[H]
\textbf{Scenariusz:} \\
\begin{table}[H]
    \begin{tabular}{ll}
    \multicolumn{2}{l}{\textit{\textbf{Umowa A3.4 --   prosta umowa z świadkami (aktywna)}}}                                                           \\ \hline
    0.        &  S tworzy umowę 
\\ \hline
    1.        &    K zawiera umowę i wpłaca cenę – zgłasza zbiór ŚK 2t+1 świadków 
\\ \hline
    2.        & S zatwierdza K i zgłasza zbiór SŚ 2t+1 świadków, taki że ŚK n ŚS =/= O
\\ \hline
    3.        & ŚK U ŚS oraz K i S głosują na K lub S – procedura głosowania
\\ \hline
    4.        & Większość za S – S wypłaca i ocenia K i ŚK 
\\ \hline
    5.	    &  K ocenia ŚS i S 
\\ \hline
    6.	    &  ŚK U ŚS (ci co zagłosowali) ocenia S i K (wypłacają prowizje) 
\\ \hline
		& KONIEC
\\
	\multicolumn{2}{l}{}
\\        
	4a.        &   Większość za K – K wypłaca i ocenia S i ŚS 
\\ \hline   
	5a.        &  S ocenia ŚK i K 
\\ \hline     
	        & powrót do 6.
\\  
	\multicolumn{2}{l}{}
\\   
			& if(nieparzysta ilość nie zagłosuje i jest remis) 
\\ \hline     
	4b.		&  S wypłaca i ocenia ŚS 
\\   \hline  
	5b.        & K ocenia S i ŚK  
\\  \hline  
	        &   powrót do 6.
\\  
\end{tabular}
\end{table}
\label{Scenariusz A4.1}
\end{figure}

\newpage
\subsubsection{Umowa A4.2}
\begin{figure}[H]
\textbf{Scenariusz:} \\
\begin{table}[H]
    \begin{tabular}{ll}
    \multicolumn{2}{l}{\textit{\textbf{Umowa A4.2 --   aukcja prosta niejawna ze świadkami (pasywna)}}}                                                           \\ \hline
    0.        & S tworzy umowę i zgłasza zbiór świadków ŚS o rozmiarze 2t+1 
\\ \hline
    1. async1       & K1 wpłaca depozyt k1 i wysyła kryptograficzny hasz h1
\\ 
		& Jeśli nie ma świadka, to zgłasza zbiór ŚK1 2t+1 świadków,
\\ 
	& taki że ŚK1 n ŚS =/= O
\\ \hline
    2. async1        &  S zatwierdza udział K1 przy pierwszej wpłacie
\\ \hline
    3. async2        &  K2 wpłaca depozyt k2 i wysyła kryptograficzny hasz h2 
\\ 
		& Jeśli nie ma świadka, to zgłasza zbiór ŚK2 2t+1 świadków,
\\
	&  taki że ŚK2 n ŚS =/= O
\\ \hline
    4. async2        & S zatwierdza udział K2 przy pierwszej wpłacie
\\ 
	\multicolumn{2}{l}{}
\\        
	N. async N        & KM wpłaca depozyt kM i wysyła kryptograficzny hasz hM
\\ \hline   
	N+2. async N         & S zatwierdza udział KM przy pierwszej wpłacie 
\\ \hline     
	N+3.        & Faza licytacji się kończy 
\\    \hline   
	N+4. async1         & K1 wysyła w postaci jawnej wartość p1, sól s1 i znacznik prawdziwości f1
\\ \hline     
	N+5. async2         & K2 wysyła w postaci jawnej wartość p2, sól s2 i znacznik prawdziwości f2
\\   
	\multicolumn{2}{l}{}
\\        
	N+M+3. async M .        & KM wysyła w postaci jawnej wartość pM, sól sM i znacznik prawdziwości fM
\\ \hline 
	N+M+4.        & Koniec aukcji – wybieramy ŚKI i KI, takie że pI == max(S), gdzie 
\\ 
			&  S = \{pK: 1=<K=<N \& wysłane(pK) \& fK \& pK <= kK \& hK ==
\\
	& ==  hasz(pK.fK.sK)\}. 
\\ \hline     
	N+M+5.        & KI potwierdza poprawność 
\\ \hline  
	N+M+6.        &  S wypłaca środki w ilości pI i ocenia KI 
\\ \hline  
	N+M+7.        &  KI ocenia S i wypłaca kI – pI
\\  
	\multicolumn{2}{l}{}
\\        
	N+M+5.a.        & KI nie potwierdza
\\ \hline 
	N+M+6.a.        & ŚKI U ŚS oraz K i S głosują na KI lub S – procedura głosowania 
\\ \hline     
	N+M+7.a.        &  Większość za KI – KI wypłaca kI i ocenia S i ŚS 
\\ \hline  
	N+M+8.a.        &  S ocenia ŚKI i KI
\\ 
	\multicolumn{2}{l}{}
\\       
	N+M+7.aa.        &  Większość za S – S wypłaca pI i ocenia KI i ŚKI 
\\ \hline  
	N+M+8.aa.        &   KI ocenia ŚS i S i wypłaca kI – pI
\\ 
	\multicolumn{2}{l}{}
\\      
		& if(nieparzysta ilość nie zagłosuje i jest remis) 
\\ \hline
	N+M+6.ab.        &  S wypłaca pI i ocenia ŚS 
\\ \hline  
	N+M+7.ab.        & KI ocenia S i ŚKI i wypłaca kI – pI [następny N+M+9.a.]
\\ 
	\multicolumn{2}{l}{}
\\  
	N+M+9.a.        & ŚKI U ŚS (ci co zagłosowali) ocenia S i K (wypłacają prowizje)
\\ 
	&	[następny N+M+9.]
\\ 
	\multicolumn{2}{l}{}
\\  
	N+M+9.        &  Klienci ze zbioru 
\\ 
			&  R=\{KK: 1=<K=<N\& wysłane(pK)\& hK== hash(pK.fK.sK)\} 
\\ 
	& 	wypłacają wpłacone k 
\\ 
	\multicolumn{2}{l}{}
\\  
	N+M+10.        &  Klienci spoza zbioru R tracą k na rzecz systemu
\\ 
\end{tabular}
\end{table}
\end{figure}
\begin{figure}[H]
\textbf{Legenda:} \\
K1-KM – klienci \\
async[numer] – sąsiadujące kroki oznaczone async o różnych numerach wykonują się w dowolnej kolejności \\
Klienci mogą dokonać więcej niż jednej wpłaty \\ \\
\label{Scenariusz A4.2}
\end{figure}

\subsubsection{Umowa A4.3}
\begin{figure}[H]
\textbf{Scenariusz:} \\
\begin{table}[H]
    \begin{tabular}{ll}
    \multicolumn{2}{l}{\textit{\textbf{Umowa A4.3 --   aukcja prosta niejawna ze świadkami (aktywna)}}}                                                           \\ \hline
    0.        & S tworzy umowę i zgłasza zbiór świadków ŚS o rozmiarze 2t+1 
\\ \hline
    1. async1       &  K1 wpłaca depozyt k1 i wysyła kryptograficzny hasz h1 (opcjonalnie opłaty dla Ś)
\\ 
		& Jeśli nie ma świadka, to zgłasza zbiór ŚK1 2t+1 świadków,
\\
		& taki, że ŚK1 n ŚS =/= O
\\ \hline
    2. async1        &   S zatwierdza udział K1 przy pierwszej wpłacie 
\\ \hline
    3. async2        &  K2 wpłaca depozyt k2 i wysyła kryptograficzny hasz h2 (opcjonalnie opłaty dla Ś)
\\ 
		& Jeśli nie ma świadka, to zgłasza zbiór ŚK2 2t+1 świadków,
\\ 
		& taki, że ŚK2 n ŚS =/= O
\\ \hline
    4. async2        & S zatwierdza udział K2 przy pierwszej wpłacie
\\ 
	\multicolumn{2}{l}{}
\\        
	N. async N        &  KM wpłaca depozyt kM i wysyła kryptograficzny hasz hM
\\ 
			&  (opcjonalnie opłaty dla Ś)
\\ 
			& Jeśli nie ma świadka, to zgłasza zbiór ŚKM 2t+1 świadków, 
\\
			& taki, że ŚKM n ŚS =/= O
\\ \hline  
	N+2. async N         & S zatwierdza udział KM przy pierwszej wpłacie 
\\ \hline     
	N+3.        & Faza licytacji się kończy 
\\    \hline   
	N+4. async1         & K1 wysyła w postaci jawnej wartość p1, sól s1 i znacznik prawdziwości f1
\\ \hline     
	N+5. async2         & K2 wysyła w postaci jawnej wartość p2, sól s2 i znacznik prawdziwości f2
\\   
	\multicolumn{2}{l}{}
\\        
	N+M+3. async M .        & KM wysyła w postaci jawnej wartość pM, sól sM i znacznik prawdziwości fM
\\ \hline 
	N+M+4.        & Koniec aukcji – wybieramy ŚKI i KI, takie że pI == max(S), gdzie 
\\ 
			&  S = \{pK: 1=<K=<N \& wysłane(pK) \& fK \& pK <= kK \& hK ==
\\ 
			& == hasz(pK.fK.sK)\}.  
\\ \hline     
	N+M+5.        &  Głosowanie – większość za 
\\ \hline  
	N+M+6.        &  S wypłaca środki w ilości pI i ocenia KI i ŚKI 
\\ \hline  
	N+M+7.        &  KI ocenia S i ŚS i wypłaca kI – pI 
\\  \hline  
	N+M+8.        &  ŚKI U ŚS (ci co zagłosowali) oceniają KI i S 
\\ 
			& (opcjonalnie wypłacają swoje prowizje) 
\\  \hline  
	N+M+9.        & Klienci ze zbioru
\\ 
			&  R=\{KK: 1=<K=<N\& wysłane(pK)\& hK== hash(pK.fK.sK)\}
\\ 
			& wypłacają wpłacone k 
\\  \hline  
	N+M+10.        &  Klienci spoza zbioru R tracą k na rzecz systemu
\\  
	\multicolumn{2}{l}{}
\\        
	N+M+5.a.        & Większość przeciw 
\\ \hline 
	N+M+6.a.        &  KI wypłaca i ocenia S i ŚS 
\\ \hline     
	N+M+7.a.        &  S ocenia KI i ŚKI 
\\ \hline  
	N+M+8.a.        &  ŚKI U ŚS (ci co zagłosowali) (wypłaca i) ocenia KI i S 
\\ \hline  
	N+M+9.a.        &  powrót do N+M+9.
\\ 
\end{tabular}
\end{table}
\end{figure}

\begin{figure}[H]
\textbf{Legenda:} \\
K1-KM – klienci \\
async[numer] – sąsiadujące kroki oznaczone async o różnych numerach wykonują się w dowolnej kolejności \\
Klienci mogą dokonać więcej niż jednej wpłaty \\
Klient może zgłosić tylko jednego U. Robi to przy pierwszej zaakceptowanej wpłacie.\\ \\
\label{Scenariusz A4.2}
\end{figure}

\newpage
\subsubsection{Umowa PA1}
\begin{figure}[H]
\textbf{Scenariusz:} \\
\begin{table}[H]
    \begin{tabular}{ll}
    \multicolumn{2}{l}{\textit{\textbf{Umowa PA1 --  proste wiązanie świadków prowizją}}}                                                           \\ \hline
    1.       &  Udziałowiec U tworzy umowę 
\\ \hline
    2.       &    Świadek Ś1 zgłasza się 
\\ \hline
    3.        &  U go akceptuje i wpłaca prowizję (lub nie) 
\\ \hline
    4.        &  Ś2 zgłasza się 
\\ \hline
    5.		&   U go akceptuje  i wpłaca prowizję (lub nie) 
\\
	\multicolumn{2}{l}{}
\\        
	N+1.        &   ŚN zgłasza się 
\\ \hline  
	N+2.        &  U go akceptuje  i wpłaca prowizję (lub nie)  
\\ \hline     
	N+3.        & U zgłasza A1  
\\    \hline   
	N+4.         &  A1 raportuje uczestnictwo świadków 
\\   
	\multicolumn{2}{l}{}
\\        
	N+M+2.        & U zgłasza AM 
\\ \hline 
	N+M+3.        & AM raportuje uczestnictwo świadków 
\\ \hline     
	N+M+4.        &   Świadkowie, którzy uczestniczyli (głosowali poprawnie) we wszystkich 
\\
			& umowach, do których zostali wytypowani, wypłacają prowizje 
\\ \hline  
	N+M+5.        &   U wypłaca resztę 
\\ 
	\multicolumn{2}{l}{}
\\        
	        & if(teraz > T) 
\\ \hline 
	1a.        &  Świadkowie wypłacają prowizje jeśli uczestniczyli we wszystkich 
\\
		& umowach, do których zostali wytypowani 
\\ \hline     
	2a.        &  U wypłaca resztę
\\ 
\end{tabular}
\end{table}
\textbf{Legenda:} \\
dane wejściowe: prowizja na świadka P, ilość świadków N, Ilość umów A po której jest wypłata M, czas wygaśnięcia T \\
dane dodatkowe: licznik zgłoszonych umów I \\ \\
\label{Scenariusz PA1}
\end{figure}

\newpage
\subsubsection{Umowa PA2}
\begin{figure}[H]
\textbf{Scenariusz:} \\
\begin{table}[H]
    \begin{tabular}{ll}
    \multicolumn{2}{l}{\textit{\textbf{Umowa PA2 --  proste wiązanie świadków prowizją i kaucją}}}                                                           \\ \hline
    1.       &  Udziałowiec U tworzy umowę 
\\ \hline
    2.       &    Świadek Ś1 zgłasza się i wpłaca kaucję
\\ \hline
    3.        &  U go akceptuje i wpłaca prowizję (lub nie) 
\\ \hline
    4.        &  Ś2 zgłasza się i wpłaca kaucję
\\ \hline
    5.		&   U go akceptuje  i wpłaca prowizję (lub nie) 
\\
	\multicolumn{2}{l}{}
\\        
	N+1.        &   ŚN zgłasza się i wpłaca kaucję
\\ \hline  
	N+2.        &  U go akceptuje  i wpłaca prowizję (lub nie)  
\\ \hline     
	N+3.        & U zgłasza A1  
\\    \hline   
	N+4.         &  A1 raportuje uczestnictwo świadków 
\\   
	\multicolumn{2}{l}{}
\\        
	N+M+2.        & U zgłasza AM 
\\ \hline 
	N+M+3.        & AM raportuje uczestnictwo świadków 
\\ \hline     
	N+M+4.        &   Świadkowie, którzy uczestniczyli (głosowali poprawnie) we wszystkich 
\\
			& umowach, do których zostali wytypowani, wypłacają prowizja + kaucja 
\\ \hline  
	N+M+5.        &   U wypłaca resztę 
\\ 
	\multicolumn{2}{l}{}
\\        
	        & if(teraz > T) 
\\ \hline 
	1a.        &  Świadkowie wypłacają prowizja + kaucja jeśli uczestniczyli we wszystkich
\\
		& umowach, do których zostali wytypowani. 
\\ 
		& Nie wytypowani do żadnych wypłacają samą kaucję
\\ \hline     
	2a.        &  U wypłaca resztę. Kaucje świadków, którzy nie wywiązali się, 
\\
				& zostają zabrane przez system.
\\ 
\end{tabular}
\end{table}
\textbf{Legenda:} \\
dane wejściowe: prowizja na świadka P, ilość świadków N, Ilość umów A po której jest wypłata M, czas wygaśnięcia T, kaucja K \\
dane dodatkowe: licznik zgłoszonych umów I \\ \\
\label{Scenariusz PA2}
\end{figure}

\newpage
\subsubsection{Głosowanie świadków}
\begin{figure}[H]
\textbf{Głosowanie WV1} \\
\textbf{Scenariusz:} \\
wejście: zbiór Ś, K, S, maksymalny czas trwania, zdarzenie kończące\\ \\
1. oddawanie głosów na K lub S wyjście: wynik, kto oddał głos \\ \\
\textbf{Głosowanie WV2} \\
\textbf{Scenariusz:} \\
wejście: zbiór Ś, K, S, maksymalny czas trwania, zdarzenie kończące fazę głosowania, zdarzenie kończące fazę odsłaniania\\ \\
1. wysyłanie haszy kryptograficznych h \\
2. wysyłanie głosów v i soli s \\
wyjście: wynik, kto oddał ważny głos (h == hash(v.s))\\ \\
\label{Scenariusz glosowania swiadkow}
\end{figure}

\subsection{Specyfikacja przypadków użycia}
\subsubsection{Diagram przypadków uzycia}

\includegraphics[width=1.0\textwidth]{Use_Case_Diagram}

\subsubsection{Diagram sekwencji}
\includegraphics[width=1.0\textwidth]{DiagramSekwencji}
\newpage
\subsubsection{Maszyny stanowe}

Umowa A1.1\\
\includegraphics[width=1.0\textwidth]{SM-A1_1}

\newpage
\section{Analiza i krytyka zastosowanej architektury}

\subsection{Ekosystem Ethereum}
\label{sec:EkosystemEthereum}

\subsubsection{Siec Ethereum}

Ethereum jest projektem pragnącym stworzyć uogólnioną technologię, pozwalającą na zrealizowanie dowolnego rozwiązania opierającego się na transakcyjnej maszynie stanowej. Co więcej, jednym z jego celów jest dostarczenie deweloperom pełnego, zwartego systemu do budowy aplikacji, we wcześniej szerzej nie znanym paradygmacie ,,trustful object messaging compute framework''.

Innym kluczowym celem przyświecającym Ethereum, jest zestawienie kanału do interakcji między agentami, którzy w innych okolicznościach nie mieliby możliwości sobie zaufać, co do poprawności procedury. Źródłem nieufności może być geograficzna odległość, brak odpowiedniej wiedzy, niechęć, niepewność lub korupcja istniejącego systemu prawnego.\cite{work:ethereum2018yellowpaper}

\subsubsection{Własności sieci}
\label{sec:WlasnosciSieci}

Kontrakty budowane w takim systemie, posiadają rzadko spotykane w rzeczywistym świecie własności. Bezstronność algorytmicznego interpretera, w naturalny sposób gwarantuje odporne na korupcję wykonanie transakcji. Przejrzystość rozstrzygnięcia kontraktu pochodząca z historii interakcji, jak również z zasad rządzących kodem, nie istniejąca w systemach opartch o języki naturalne i ludzkie działania, które w swojej naturze są nieprecyzyjne\cite{work:ethereum2018yellowpaper}.

Kontrakty są uruchamiane na EVM - Ethereum Virtual Machine, maszynie wykonującej ich byte-code (W Ethereum 2.0 będzie używany eWASM), której model programowy jest uważany za zupełny w sensie Turinga (nieformalnie, formalnie jest automatem liniowo ograniczonym). Pozwala to na oprogramowanie dowolnego algorytmu, który wykonałby domowy komputer\cite{website:EthHubCFTC,website:EthTuringMedium,work:ethereum2018yellowpaper}.

Oprócz ekspresywnego zestawu instrukcji, kontrakty mają możliwość zapisywania lub odczytania swojego stanu bezpośrednio z blockchaina, bez potrzeby angażowania w ten proces użytkownika. Posiadają one również prawie pełną swobodę, co do modyfikacji treści swojego stanu, oraz ograniczoną mozliwość zmiany stanu innych kontraktów. Pozwala to na budowanie rozwiązań, które są niemożliwe lub trudne do zbudowania z pomocą bezstanowych UXTO\cite{website:EthHubCFTC,website:EthTuringMedium, work:ethereum2018yellowpaper}.


Klienci wchodzą w interakcję z siecią, poprzez rozgłaszanie transakcji do znanych węzłów. Poprawnie uformowane i podpisane transakcje, są zbierane i grupowane, przez wyróżnioną grupę węzłów nazywaną górnikami (w Ethereum 2.0 walidatorami), w bloki. Mają oni pewną swobodę w tym ile transakcji będzie się znajdować w bloku i w jaki sposób będą one w nim uporządkowane. To ile maksymalnie transakcji może być w jednym bloku jest określane przez zmienną \textit{block gas limit}.
Kiedy górnik zbierze i przygotuje kolejkę transakcji, wykonuje w pętli funkcję tranzycji stanu, która przyjmuje obecny stan sieci i pierwszą transakcję z kolejki, a zwraca nowy stan sieci, dopóki kolejka nie jest pusta. W czasie wykonania pętli zbiera on informacje o tym jakie transakcje zakończyły się powodzeniem i jakie logi one wygenerowały i zapisuje je w \textit{transaction receipts tree}. Po wykonaniu wszystkich wybranych transakcji, górnik zapisuje nowy stan sieci w \textit{world state tree}, wykonane transakcje w \textit{transaction tree} i umieszcza je w bloku. Na tak przygotowanej strukturze wykonuje on algorytm Ethash (w Ethereum 2.0 ten mechanizm inaczej wygląda), czyli kopanie bloków. Po znalezieniu liczby spełniającej wymogi algorytmu, dołącza ją do nowo wykopanego bloku i rozgłasza go w sieci.
Zostaje on zaakceptowany przez resztę węzłów, jeśli nowy stan jest poprawny i wynika z transakcji zawartych w bloku, wskazuje na historyczny poprawny blok, a łańcuch do którego należy spełnia warunki protokołu GHOST (w uproszczeniu wybieramy ten łańcuch, nad którym wykonano najwięcej pracy).\cite{work:ethereum2018yellowpaper,website:EthHubCFTC} Jest jeszcze kilka innych warunków poprawności bloku, ale nie są one bardzo istotne z punktu widzenia tej pracy.


\clearpage
\KOMAoptions{paper=A3,pagesize}
\KOMAoptions{paper=landscape}
\recalctypearea

\begin{center}

\begin{figure}[h]
\caption[Caption for LOF]{Formowanie bloków w Ethereum\footnotemark}
\centering
\includegraphics[width=1.0\textwidth, height=0.742\textheight]{blockchain}
\end{figure}
\footnotetext{Pobrano z \url{https://ethereum.stackexchange.com/questions/268/ethereum-block-architecture/6413#6413}}

\end{center}

\clearpage
\KOMAoptions{paper=portrait}
\KOMAoptions{paper=A4,pagesize}
\recalctypearea


Mechanizmem zachęcającym górników do zbierania i wykonywania transakcji jest obecność opłat za nie. Wykonanie każdej z nich kosztuje \textit{gas}; jednostkę wykonanej pracy, istniejącą tylko w kontekście danej transakcji. Użytkownik, w rozgłaszanej przez siebie transakcji, określia cenę w etherze za jednostkę gazu oraz maksymalną ilość gazu jaką może ona pobrać. Inną zachętą jest nagroda za wyprodukowanie poprawnego bloku, którą górnik może sam sobie przyznać. Takie mechanizmy wspierają żywotność sieci, jednak na ile chronią przed cenzurą i wzmacniają bezpieczeństwo sieci jest spornym tematem\cite{work:ethereum2018yellowpaper}.

Ethereum 1.0 gwarantuje wykonanie kontraktów i poprawne przeprowadzenie transakcji, czyli bezpieczeństwo i żywotność sieci, przy założeniu, że więcej niż 50\% mocy hashującej działa zgodnie z protokołem, a wiadomości o stanie sieci docierają do innych klientów synchronicznie (istnieje znane maksymalne opóźnienie wiadomości). Jeśli, któreś z powyższych założeń jest niespełnione, to tranasakcje lub wykonanie kodu może być cenzurowane (zignorowane, niewykonane), różni uczestnicy sieci mogą widzieć jej odmienny stan, kontrakty i mechanizmy nie opierające się na podpisach kryptograficznych mogą paść ofiarą ,,double spending attack'' oraz historia już wykonanych transakcji może być zmieniana \cite{work:AnonByzConsBit}.


\subsubsection{Środowisko wykonawcze kontraktów}

Kontrakty są bytami posiadającymi kod, jak również stan, nazywany też pamięcią \textit{storage}. Są one zapisane pod pewnym adresem w przestrzeni kont. Interakcję z kontraktem przeprowadza się poprzez wysłanie pod jego adres transakcji. Kontrakt dotknięty nią uruchamia swój kod, z zawartością transakcji jako danymi wejściowymi.

Uruchomienie transakcji, jak i każdej wykonanej instrukcji w kodzie kontraktu pochłania pewną określoną ilość gazu. Inicjator transakcji określa cenę w etherze za jednostkę gazu i limit zużycia gazu. Cena za uruchomienie jest naliczana natychmiast, a kolejne opłaty, tylko w momencie wykonania instrukcji. Jeśli incjatorowi zabraknie funduszy na opłacenie gazu, zostanie osiągnięty limit zużycia gazu lub limit gazu na blok, wykonanie kończy się wyjątkiem \textit{out of gas} i zmiany wprowadzone w stanie sieci zostają wycofane. Jednak bez względu na wynik obliczeń, koszty zakupu gazu nie zostają cofnięte, jako że górnicy (lub walidatorzy) ponieśli już koszty z wiązane z ich wykonaniem. Poprawne wykonanie transakcji, czyli bez zwrócenia do inicjatora wyjątku, utrwala zmiany wprowadzone w sieci, a niezużyty gaz zostaje automatycznie odsprzedany i koszty jego zakupu zostają zwrócone do inicjatora, po takiej samej cenie jak zakupu.

Uruchomienie transakcji tworzy kontekst transakcji, czli zbiór zmiennych środowiskowych dostępnych w kontrakcie istniejących, ale tylko dla tego jednego wywołania. Należą do nich adres wywołującego, adres docelowy, ilość przekazanego etheru, ilość przekazanego gazu, cena za jednostkę gazu, adres opłacający transakcję oraz tablicę dodatkowych danych\cite{work:ethereum2018yellowpaper,website:solidityDocs}.

Szczególną uwagę należy poświęcić instrukcjom \textit{REVERT}, \textit{INVALID}, \textit{CALL}, \textit{DELEGATECALL}, \textit{STATICCALL}, \textit{CREATE}, \textit{SELFDESTRUCT} i \textit{SSTORE}.
\begin{itemize}
	\item \textit{REVERT} -- kończy wykonanie wyjątkiem określonym w kodzie przez jego autora, wycofuje wprowadzone zmiany w procedurze i zwraca niezużyty gaz do nadprocedury lub inicjatora transakcji. Zwykle używany jako reakcja na niepoprawne dane wejściowe.
	\item \textit{INVALID} -- kończy wykonanie wyjątkiem i ,,spala'' cały dostępny gaz w procedurze. Zazwyczaj używany jako reakcja na błąd wewnętrzny lub złamanie niezmienników.
	\item \textit{CALL} -- pozwala kontraktowi na uruchomienie podprocedury. Procedura wywołująca określa adres docelowy (może to być ten sam lub inny kontrakt), ile etheru chce przekazać, ile gazu chce dostarczyć oraz tablicę dodatkowych danych. Wyjątki wyrzucone w podprocedurze \textbf{nie kończą} automatycznie procedury wywołującej, jedynie zostaje do niej zwrócony kod błedu. Procedura może wysłać tylko tyle etheru, ile w danym momencie posiada kontrakt, podobnie w przypadku gazu; może przekazać maksymalnie 63/64 gazu, którego posiadała w momencie wykonania instrukcji \textit{CALL}. Wywołanie podprocedury, podobnie jak uruchomienie transakcji, tworzy nowy kontekst, jednak w przeciwieństwie do tej drugiej cena za jednostkę gazu i adres opłacającego nigdy nie ulega zmianie. Wynika to z tego, że w obecnej wersji Ethereum kontrakty nie mogą same opłacać swojego wykonania.
	\item \textit{DELEGATECALL} -- przyjmuje takie same argumenty jak \textit{CALL} jednak nie zmienia w pełni kontekstu tzn.\ podprocedura ma dostęp do tych samych zmiennych środowiskowych i pamięci \textit{storage} co nadprocedura, z wyjątkiem wysłanego etheru i dostępnego gazu. Ta instrukcja jest najczęsciej wykorzystywana w implementacji bibliotek, kontraktów z możliwością aktualizacji, jak również w naszym modelu maszyny stanowej
	\item \textit{STATICCALL} -- działa podobnie jak \textit{CALL}, jednka nie pozwala na przesyłanie etheru, a wszelkie próby wywołania instrukcji zmieniających stan sieci w podprocedurze, konczą ją i zwracają do nadprocedury kod błędu.
	\item \textit{CREATE} -- tworzy nowy kontrakt
	\item \textit{SELFDESTRUCT} -- usuwa z sieci kontrakt (zeruje kod i stan), w którego kontekście została ona wywołana, a cały ether jaki był w nim zawarty przesyła pod wskazany adres. Jej wywołanie refunduje część zużytego gazu, co ma stanowić zachętę, by transakcje, jeśli to możliwe, usuwały nie potrzbne już kontrakty.
	\item \textit{SSTORE} -- zapisuje 256-bitowe słowo we wskazane miejsce \textit{storage} kontraktu, w którego kontekście została wywołana. Jeśli w jej wyniku niezerowa wartość zostaje wyzerowana, to refunduje część gazu, jako zachętę by transakcje usuwały nie potrzebne fragmenty stanu sieci
\end{itemize}

\subsubsection{Klienci Ethereum}

\begin{itemize}
	\item Geth
	\item Parity
	\item aleth
\end{itemize}

\subsubsection{Technologie wspierające}

\begin{itemize}
	\item Solidity -- język programowania wysokiego poziomu, zorientowany kontraktowo. Kompiluje się do byte-codu EVM. Zawiera w sobie mechanizmy zarządzania pamięcią \textit{storage} i pamięcią \textit{memory}.
	\item Web3 -- biblioteki dla różnych języków programowania, pomagające aplikacjom współpracującym z siecią Ethereum na wchodzenie w interakcje kontraktami.
\end{itemize}

\subsubsection{Ograniczenia technologii}
\label{sec:OgraniczeniaTechnologii}

Złożoność kontraktów jest ograniczona przez limit gazu na blok, jak również maksymalny rozmiar kodu pojedynczego kontraktu, czyli ~24kB. Możliwe jest obejście tych problemów, przez rozbijanie dużych transakcji i kontraktów na mniejsze, lecz komplikuje to interakcje z tak zbudowanym systemem\cite{work:ethereum2018yellowpaper,website:solidityDocs}.

Kontrakty ,,widzą'' jedynie stan sieci. O jakiekolwiek informacje o świecie zewnętrznym muszą prosić swoich klientów, co może być wektorem ataku na nieodpowiednio przemyślany i zaprojektowany system\cite{work:ethereum2018yellowpaper,website:solidityDocs}.

Obecna wersja Ethereum oferuje średnią przepustowość na poziomie 15 tx/s, co może utrudniać działanie aplikacji, w przypdaku dużego zainteresowania\cite{website:ethereum2018whitepaper,website:EthHubCFTC}.

Problem prywatności w Ethereum jest dalej tematem intensywnych badań i choć są proponowane działające rozwiązania, w obecnej wersji należy zakładać, że każdy może sprawdzić jaki adres wchodził w interkację z jakim kontraktem. Również umieszczając jakiekolwiek dane w kontraktach, należy zakładać, że są w pełni publiczne, jeśli nie zostały wcześniej zaszyfrowane\cite[Problems]{website:ethereum2018whitepaper}.

\subsubsection{Znane podatności kontraktów}
\label{sec:ZnanePodatnościKontraktów}
\begin{itemize}
	\item Re-Entrancy -- klasa podatności opierająca się o race-condition, która występuje kiedy kontrakt A wywołuje (np.\ poprzez wysłanie etheru) kontrakt B, a ten nieoczekiwanie wywołuje z powrotem A. Takie działanie może wprowadzić A w niepoprawny stan, pozwalający np.\ na kradzież zawartego w nim etheru lub jego zamrożenie\cite{website:TheDAOHack}. Obrona przed tymi atakami polega głównie na użyciu mutexów lub użycie wzorca \textit{checks-effects-interactions}. 
	\item Nieoczekiwany ether -- kiedy kontrakt otrzymuje dowolną transakcję/wywołanie, również taką/ie z etherem, może ją/go odrzucić, np.\ przez wywołanie instrukcji \textit{REVERT} lub \textit{INVALID}. Jednak istnieją sposoby wysłania ethereu nie pozwalające kontraktowi na reakcję, należą do nich wysłanie ethereu z pomocą \textit{SELFDESTRUCT}, wskazanie adresu kontraktu jako beneficjenta nagrody za wykopany blok i wysłanie etheru na adres, przed umieszczeniem w nim kodu. Główną strategią ochronną jest nie używanie stanu konta kontraktu, np.\ do kontrolowania ile ethereu użytkownicy wpłacili do kontraktu zgodnie z jego zastosowaniem.
	\item Wywołanie funkcji bibliotecznej z \textit{SELFDESTRUCT} -- kontrakty używają \textit{DELEGATECALL}, by używać bibliotek, mieć możliwość aktualizacji swojego kodu lub by zmniejszyć jego redundancję. W szczególności dwa ostatnie przypdaki mogą posiadać różne instrukcje zmieniające stan, jak choćby \textit{SELFDESTRUCT}. Rozważmy przypadek,który w poprawnym wykonaniu przebiega naspępująco. Kontrakt beneficjenta A wykonuje \textit{DELEGATECALL} do kontraktu dostarczającego kod B z chęcią wywołania funkcji, która uruchamia instrukcję \textit{SELFDESTRUCT}, jako że zostaje ona wykonana w kontekście A, to on ulega usunięciu. Jeśli teraz wywołamy bezpośrednio B z pomocą transakcji lub \textit{CALL}, a on sam nie posiada mechanizmów pozwalających mu sprawdzić, w jaki został wykonany, to \textit{SELFDESTRUCT} zostanie wykonany w kontekście B i on ulegnie usunięciu. Wynikiem takiej akcji będzie najprawdopodobniej DoS systemów korzystających z B. Rozwiązaniem tego problemu jest unikanie instrukcji zmieniających stan w kontraktach wywoływanych z pomocą \textit{DELEGATECALL} lub przygotowanie innych zabezpieczeń, które pozwolą rozróżnić sposób wywołania kontraktu.
	\item Hash bloku jako źródło entropii -- klasa ataków wynikająca z błędnego założenia, że hash bloków jest dobrym generatorem liczb pseudolosowych. Nie istnieją proste strategie pozwalające ominąc ten problem. Na obecną chwilę najlepszym rozwiązaniem jest korzystanie z kontraktów z rodziny RANDAO, które z pewnymi ograniczeniami, pozwalają generować liczby pseudolosowe w Ethereum.
	\item Skrócony adres -- jest to błąd po stronie klientów Ethereum, który wnika w ogólności z błędów w przekazywaniu argumentów do kontraktów.
	\item Front-Running -- klasa ataków z rodziny race-condition. Występuje ona, kiedy wcześniej niewybrani użytkownicy mają przekazać do kontraktu jakiś sekret, by uzyskać np.\ ether z gry hazardowej. Polega ona na tym, że górnik lub inny użytkownik przechwytuje transakcję z sekretem i sam, we własnym imieniu, ją wysyła do kontraktu. Najprostrzą metodą ochrony przed tym rodzajem ataków, jest konstruowanie kontraktów by najpierw zbierały hashe sekretów, a dopiero poźniej same sekrety.
	\item DoS -- najprostrzym przykładem ataku z tej klasy jest sytuacja, kiedy kontrakt napisany w Solidity rozsyła ether, w pętli do pewnej grupy użytkowników, z pomocą funkcij \textit{transfer}, która to wykonuje \textit{REVERT}, jeśli przesyłanie zakończy się nie powodzeniem. Niech jeden z użytkowników będzie złośliwym kontraktem, który przy próbie wysłania mu etheru wykonuje \textit{REVERT} lub \textit{INVALID}. Wtedy każda próba rozesłania etheru do użytkowników zakończy się niepowodzeniem, czyli DoSem. Najprostszą metodą ochrony przed tym atakiem jest użycie wzorca \textit{withdraw}, który wymusza na użytkownikach samodzielne pobranie należnych im środków.
\end{itemize}
\nocite{website:ethereumVulnerabilities}

\subsubsection{Problem centralizacji}
\label{sec:ProblemCentralizacji}

Ethereum składa swoim użytkownikom obietnicę platformy do tworzenia zdecentralizowanych aplikacji\cite{website:ethereum2018whitepaper, website:EthHubCFTC}, która jest obwarowana pewnymi ,,oczywistymi'', już omawianymi warunkami i ograniczeniami (patrz rozdział \ref{sec:WlasnosciSieci} i \ref{sec:OgraniczeniaTechnologii}). Jednak okazuje się, że ,,oczywiste'' warunki mają swoją głębszą naturę, a oprócz nich odkrywane są nowe problemy, których wpływ na Ethereum i inne kryptowaluty jest przedmiotem intensywnych debat.

Celem tego podrozdziału będzie określenie pojęcia decentralizacji, oraz omówienie zagadnień potencjalnie zagrażających jej zapewnieniu w Ethereum, jak również dyskusja, na ile przedstawione zjawiska są niebezpieczne, bez budowania formalnej miary.

Pojęcie decentralizacji będziemy rozważać pod kątem definicji przedstawionych w artykule Vitalika Buterina \cite{website:decentralizationMeaning}, jak i innch źródłach \cite{work:Decentralization}. Dlatego będziemy ją analizować na płaszczyźnie architektury, zarządzania i struktury logicznej.

Przed analizą wspomnianych problemów, należy zwrócić uwagę jakie rodzaje decentralizacji są pożadane w ekosystemie Ethereum i jakie są potencjalne konsekwencje ich zatracenia. Platforma ta promuje rozproszenie architektury, rozumiane jako rozproszenie wykonania kodu między różne maszyny, w idealnym wypadku w różnych lokacjach na świecie, rozproszenie zarządzania/podejmowania decyzji, definiowane jako podejmowanie decyzji, jaka wersja kodu będzie wykonywana na sprzęcie, przez niezależnych, rozsianych po świecie aktorów. Ethereum, w najbardziej pożądanej sytuacji, ulega centralizacji tylko na płaszczyźnie struktury logicznej, określanej jako spójny stan i interfejs sieci, czli identyczny dla wszystkich jej uczestników.

Utrata decentralizacji architektonicznej, więc sytuacja, w której stan i obliczenia wykonywane są na jednej lub kilku maszynach, wiąże się przede wszystkim ze spadkiem niezawodności, co może mieć też wpływ na dostępność. Zanik rozproszenia zarządzania, czyli jeden lub kilku aktorów decyduje, co może być zapisane w sieci i co inni widzą, może uderzać w dostępność, gwarancje poprawności stanu sieci, jak również jej spójność. Decentralizacja struktury logicznej może być rozumiana, w kontekście Ethereum, jako rozłam sieci, czyli utrata spójności stanu.\\

PoW sam w sobie stanowi tylko mechanizm zabezpieczający przed spamem i atakami Sybili, jednak w kryptowalutach, poruszanych w tej pracy, jest połączny z mechanizmem wyboru bloków, tworząc algorytm rozproszonego konsensusu. Wynika z tego, że górnicy posiadający więcej mocy obliczeniowej, mają większą szansę na zadecydowanie jaki blok będzie następnym, z czym zazwyczaj wiąże się możliwość wypłacania sobie honorarium. Kreuje to rynek, w którym górnicy konkurują z sobą, próbując uzyskać jak największą ilość hashy na sekundę, jak najmniejszym kosztem. Liczne obserwacje i badania wskazują, że kopanie podlega zjawisku ,,korzyści skali''(\textit{Economies of scale}), które pozwala zmniejszyć koszty przez tworzenie dedykowanych centrów obliczeniowych do wydobycia kryptowalut. Takie mechanizmy sprzyjają centralizacji architektonicznej i decyzyjnej. Jednak czy te zjawiska doprowadzą do wystąpenia spadku niezawodności, czy nawet złośliwego przejęcia sieci oraz jakie kryptowaluty zostaną nimi dotknięte, jest tematem licznych sporów\cite{website:PoWDatacenter, website:PoolCentralization, work:DecentralizationInBitAndEth}.

Teoretycznym limitem zanim kryptowaluta oparta o PoW utraci decentralizację zarządzania, jest przejęcie przez jednego aktora (lub kilku współpracujących) ponad 50\% mocy obliczeniowej\cite{work:AnonByzConsBit}. Praktyczny limit jest jednak niższy i zależy od czasu powstawania nowych bloków, ich rozmiaru, szybkości propagacji danych, ilości agentów, których praca została wzięta pod uwagę przy wyborze nowego bloku, wysokości wypłacanych honorariów oraz ich wartość poza systemem. Symulacje potencjalnych ataków (\textit{Double-spending}) i złośliwych zachowań górników (\textit{eclipse attacks}, \textit{selfish mining}), wskazują że koszty ataku na Ethereum są niższe niż na Bitcoina, jednak inne badania sugerują, że obecnie używany mechanizm wyboru bloków GHOST z ,,wujkami'', podnosi ten poziom do porównywalnego z Bitcoinem\cite{work:PoWSecurity, work:GHOST, website:EthGHOSTAndUncles}.\\

W obszarze krypto-ekonomii od kilku lat trwają badania i próby implementacji nowego mechanizmu rozproszonego konsensusu, nazywanego \textit{Proof-of-Stake} (dowód udziałów) lub w skróconej wersji \textit{PoS}. Istnieje wiele wariacji na temat tej klasy protokołów, a różnice między poszczególnymi realizacjami mogą być zasadnicze, jednak ich elementem wspólnym jest użycie wewnętrznych tokenów/kryptowaluty, do ustalenia stanu sieci lub wybrania węzłów, które będą odpowiedzialne za jego ustalenie. Omówienie wszystkich typów algorytmów PoS wykracza dalece poza zakres tej pracy, dlatego skoncentrujemy się na tych najmocniej związanych z Ethereum, czyli \textit{Tendermint}\cite{website:Tendermint} i \textit{Casper Proof of Stake}\cite{website:CasperPoS}.

Tendermint był jednym z pierwszych protokołów PoS, który był odporny na ataki \textit{Nothing-at-stake} i \textit{Long-Range Attacks}. Pierwszy z nich został rozwiązany przez wymóg wpłacenia depozytów bezpieczeństwa przez twórców bloków oraz wbudowanie w mechanizm konsensusu kar za niepoprawne zachowania, drugi przez finalizację odpowiednio starych bloków oraz wprowadzenie wydłużonego czasu ,,rozmrażania'' wypłacanych depozytów, co oznacza że twórca bloków po zgłoszeniu chęci odzyskania środków, musi odczekać pewien czas zanim będzie mógł nimi ponownie dysponować. Z tymi własnościami stawiane są twierdzenia, że bezpieczeństwo i odporność na centralizację protokołu Tendermint są porównywalne z Bitcoinem\cite{website:PoSHistory2c, website:TendermintVsCasper, website:LongRangeAttacks, website:StakeInPoS}. 

Casper Proof of Stake jest mechanizmem konsensusu rozwijanym równolegle do Tendermint, czerpiącym z rozwiązań tego pierwszego. Jednym z istonych usprawnień na płaszczyźnie decentralizacji wprowadzonych przez Casper PoS było zmienienie modeli agresorów, ze złośliwego walidatora ze stałym procentem depozytów bezpieczeństwa oraz zewnętrznego agenta rozdającego ,,łapówki'' (\textit{bribing attacker model}), na zbiór ,,karteli'' i oligopoli, czyli zbiorów bogatych walidatorów zdolnych do złożonej kooperacji. Ta zmiana była uzasadniona tym, że w protokole Tendermint, jak i też protokołach opartych o PoW, bogaci/o dużej mocy obliczeniowej walidatorzy/górnicy mogą czerpać korzyści z formowania oligopoli i cenzurowania uczestników protokołu nie należących do ich organizacji, co skutkuje zmniejszeniem liczby niezależnych walidatorów/górników, a co za tym idzie zwiększeniem centralizacji. Odpowiedź na taki model adwersarza, polegała na dodaniu nowych warunków zniszczenia depozytów bezpieczeństwa. Dokładniej jeśli jakaś część aktywnych walidatorów nie jest w stanie wystarczająco często podpisywać nowych bloków, wtedy wszyscy aktywni walidatorzy zostają ukarani (można to uznać za pewien analog \textit{Anti-trust law}). Twórcy Ethereum przedstawiają ten model, jako znacznie bardziej odporny na centralizację decyzyjną niż obecne systemy PoW, jednak ostateczna ocena tego systemu nastapi nie wcześniej niż po forku mainnetu, wprowadzającego te innowacje\cite{website:PoSHistory4c, website:PoSHistory5c, website:CasperPoS}.


%   Ethereum PoS a centralizacja
%   Infura
%   System rządów Ethereum

%   selfish
%   https://arxiv.org/pdf/1311.0243v2.pdf

%   eclipse
%   https://eprint.iacr.org/2015/263.pdf


%   https://www.coindesk.com/bitcoin-mining-detente-ghash-io-51-issue

%   https://forum.blockstack.org/t/pos-blockchains-require-subjectivity-to-reach-consensus/762?u=muneeb

%   https://bitcoinexchangeguide.com/bringing-true-decentralization-to-ethereum-eth-infura-has-got-to-go/

%   https://www.journaltranscript.com/2018/12/infura-is-solely-contradicting-ethereums-decentralization-reputation/

%   https://medium.com/@VitalikButerin/the-meaning-of-decentralization-a0c92b76a274

%   https://cointelegraph.com/news/6-myths-about-ethereum-decentralization

%   jeszcze więcej rzeczy

%   https://www.ccn.com/dev-ethereum-may-fail-if-it-relies-on-infura-to-run-nodes-potential-solution/
%   https://hackernoon.com/the-ethereum-blockchain-size-has-exceeded-1tb-and-yes-its-an-issue-2b650b5f4f62

\subsubsection{Zgodność z Ethereum 2.0}

Znaczna część zaproponowanych przez nas rozwiązań, będzie musiała być ponownie przeanalizowana, a nawet zmieniona, w kolejnych wersjach Ethreum, głównie z powodu dodania mechanizmu naliczania ,,czynszu'' za używanie pamięci \textit{storage}, jak również pamięci, w której przechowywany jest kod kontraktów. Jedną z już zdiagnozowaych zmian, konieczną do wprowadzenia wraz z wprowadzeniem ,,czynszu'', będzie zwiększenie modularyzacji, w szczególności w obszarze \textit{AgreementManager} oraz tokenu ERC20 (patrz \ref{OMclass2} i \ref{OMclass4}).

Kolejną istotną zmianą będzie wprowadzenie \textit{shardingu}, który pozwoli zwiększyć przepustowość sieci, jak również doda nowe mechanizmy asynchronicznego, wzajemnego wywoływania się kontraktów. Dodanie obsługi tych funkcjonalności będzie kluczowe, by zapewnić poprawne działanie systemu i zapewnić jak najwyższą wydajność naszych usług.

\subsection{Zidentyfikowane problemy}
\label{sec:ZidentyfikowaneProblemy}

\subsubsection{Model pamięci Solidity}
\label{sec:ModelPamięciSolidity}

Nasz projekt ze swojej natury wymaga implementacji podstawowych elementów bazo-danowych i zapisywania stanu. Najbardziej naturalnym rodzajem pamięci w kontraktach, który nadaje się do tego zadania jest \textit{storage}. Jednak Solidity nie pozwala traktować tej pamięci jak analogu pliku, do którego można arbitralnie pisać i odczytywać, tylko jak pola obiektu, o zdefiniowanym typie, lokalizacji i zasadach dostępu. Takie rozwiązanie, nie wątpliwie pomaga zaznajomić się z językiem i ułatwia pisanie niedużych projektów, ale utrudnia konstruowanie bardziej skomplikowanych modeli stanu kontraktów.

\subsubsection{Modularyzacja systemu}
\label{sec:ModularyzacjaSystemu}

Kontrakty nie pozwalają na agregację lub kompozycję innych kontraktów, czyli nie jest możliwe utworzenie jednego wewnątrz drugiego, tak by pierwszy nie był w pełni widoczny i dostępny dla całej reszty sieci. Ten fakt sprawia, że pomimo obiektowego charakteru Solidity, nie jest możliwe zastosowanie części wzorców projektowych, takich jak np.\ kompozyt, strategia, czy stan. Brak prostych odpowiedników, znacząco skomplikował budowę rozwiązań w naszym systemie.

\subsection{Zaimplementowane rozwiązania}

\subsubsection{Generyczna maszyna stanowa}
\label{sec:GenerycznaMaszynaStanowa}

Jako odpowiedź na brak prostego odpowiednika wzorca stanu, zaprojektowaliśmy i zaimplementowaliśmy autorską generyczną maszynę stanową. Jej mechanizm działania opiera się na zbieraniu wywołań na głownym kontrakcie i przekazywaniu ich z pomocą \textit{DELEGATECALL} do innch kontraktów, w zależności od obecnego stanu.

Jej implementacja obejmuje kontrakty:
\begin{itemize}
	\item \textit{StateMachine} -- centralny w tym rozwiązaniu kontrakt, który przechowuje obecny stan, adresy kolejnych stanów oraz inne dane zapisane do pamięci \textit{storage}. Wykorzystuje on \textit{fallback}, czyli funkcję bez nazwy, do zbierania wszystkich wywołań, a następnie używa mechanizmu \textit{DELEGATECALL}, by przekazać sterowanie do odpowiedniego kodu.
	\item \textit{AbstractState} -- abstrakcyjny kontrakt, po którym dziedziczą stany używane w maszynie stanowej. Kod pochodnych do niego kontraktów jest ładowany w czasie wywołania \textit{DELEGATECALL}.
	\item \textit{IState} -- interfejs stanów.
	\item \textit{IStateMachineBase} -- interfejs maszyny stanowej.
	\item\textit{IStateMachineBase} -- wspólny interfejs dla maszyn i stanów.
\end{itemize}

 , \textit{AbstractState}, 

\subsubsection{Abstrakcja nad pamięcią \textit{storage}}
\label{sec:AbstrakcjaNadPamięciąStorage}

Zestaw bibliotek i abstrakcyjnych kontraktów, które konstruują warstwę abstrakcji, omijającą mechanizmy zarządzania \textit{storage} przez Solidity. Pozwalają one traktować omawianą pamięc, jak plik do którego można swobodnie pisać. Pozwala to konstruować kontrakty wywoływane z pomocą \textit{DELEGATECALL}, które mogą bezpiecznie posiadać stan.

Implementacja jest zawarta w:
\begin{itemize}
	\item \textit{StorageUtils} -- biblioteka z zestawem funkcji pozwalających na zapis i odczyt arbitralnych danych oraz struktur implementujących ,,wskaźnik'' na elementy \textit{storage}.
	\item \textit{StorageManagement} -- biblioteka implementująca ,,strony'' pamięci \textit{storage}, które moga być bezpiecznie używane przez kontrakty wywoływane z pomocą \textit{DELEGATECALL}.
	\item \textit{StorageController} -- kontrakt abstrakcyjny, po którym dziedziczą kontrakty będące donorami pamięci \textit{storage}.
	\item \textit{StorageClient} -- kontrakt abstrakcyjny, dziedziczą po nim kontrakty klienckie, które używają \textit{storage} \textit{StorageControllera}, przy wywołaniu z pomocą \textit{DELEGATECALL}.
\end{itemize}

\subsubsection{Umowa jako maszyna stanowa}
\label{sec:UmowaJakoMaszynaStanowa}

Połączenie powyższych rozwiązań, pozwoliło stworzyć nam zestaw kontraktów realizującycj umowę, który jest modularny i względnie łatwy w rozszerzaniu. Nie mniej rozwiąznie wymaga przetestowania, z większą ilością zaimplementowanych scenariuszy, by w pełni zdiagnozować inne potencjalne problemy takiej architektury.

\clearpage
\KOMAoptions{paper=A3,pagesize}
\KOMAoptions{paper=landscape}
\recalctypearea

\subsection{Model obiektowy systemu}
\subsubsection{Back-end}
\includegraphics[width=1.0\textwidth, height=0.9\textheight]{Contract_Class_Diagram_Class_Diagram}

\clearpage
\KOMAoptions{paper=portrait}
\KOMAoptions{paper=A4,pagesize}
\recalctypearea

\subsubsection{Opis kontraktów}

\begin{figure}[H]
\centering
\includegraphics[width=.4\textwidth]{OMclass2}
\caption{Interfejs AgreementManagera}
\begin{table}[H]
    \begin{tabular}{|l|l|l|l|}
    \hline
    \textit{\textbf{Klasa}}                & \multicolumn{2}{m{12cm}|}{\textit{IAgreementManager}}                                    \\ \hline
    \textit{\textbf{Opis klasy}}                 & \multicolumn{2}{m{12cm}|}{\textit{Odpowiada za metody tworzenia, usuwania, szukania, potwierdzania i ustawiania umów w fabryce}} \\ \hline
    \multirow{ 2}{*}{\textit{\textbf{Metody}}} & \multicolumn{2}{m{12cm}|}{\textit{\textbf{create()} --- tworzy nowy kontrakt}}    \\ 
		&  \multicolumn{2}{m{12cm}|}{\textit{\textbf{remove()} --- usuwa istniejący kontrakt}} \\ 
		&  \multicolumn{2}{m{12cm}|}{\textit{\textbf{search()} --- szuka istniejącego kontraktu}} \\
		&  \multicolumn{2}{m{12cm}|}{\textit{\textbf{setAgreementFactory()} --- przypisuje fabrykę do managera}} \\
		&  \multicolumn{2}{m{12cm}|}{\textit{\textbf{checkReg()} ---  sprawdza czy kontrakt znajduje się na liście managera}} \\ \hline
\end{tabular}
\end{table}
\label{OMclass2}
\end{figure}


\begin{figure}[H]
\centering
\includegraphics[width=.4\textwidth]{OMclass1}
\caption{AgreementManager}
\begin{table}[H]
    \begin{tabular}{|l|l|l|l|}
    \hline
    \textit{\textbf{Klasa}}                & \multicolumn{2}{m{12cm}|}{\textit{AgreementManager}}                                    \\ \hline
    \textit{\textbf{Opis klasy}}                 & \multicolumn{2}{m{12cm}|}{\textit{Odpowiada za nadzór fabryki kontraktów jak i tworzenia, usuwania i szukania ich za pomocą interfejsu IAgreementManager}} \\ \hline
    \multirow{ 2}{*}{\textit{\textbf{Zmienne}}} & \multicolumn{2}{m{12cm}|}{\textit{\textbf{agreements} --- zbiór dostępnych konktraktów}}    \\ 
		&  \multicolumn{2}{m{12cm}|}{\textit{\textbf{iAgreementFactory} --- interfejs fabryki umów}} \\ \hline
\end{tabular}
\end{table}
\label{OMclass1}
\end{figure}


\begin{figure}[H]
\centering
\includegraphics[width=.4\textwidth]{OMclass3}
\caption{Kontrakt Agreement}
\begin{table}[H]
    \begin{tabular}{|l|l|l|l|}
    \hline
    \textit{\textbf{Klasa}}                & \multicolumn{2}{m{12cm}|}{\textit{Agreement}}                                    \\ \hline
    \textit{\textbf{Opis klasy}}                 & \multicolumn{2}{m{12cm}|}{\textit{Kontrakt, z którego można utworzyć umowy.}} \\ \hline
    \multirow{ 2}{*}{\textit{\textbf{Zmienne}}} & \multicolumn{2}{m{12cm}|}{\textit{\textbf{NAME\_SIZE} ---  stała, która mówi ile 256-bitowych słów zajmuje string}}    \\ 
		&  \multicolumn{2}{m{12cm}|}{\textit{\textbf{DESCRIPTION\_SIZE} --- stała, która mówi ile 256-bitowych słów zajmuje string}} \\ 
		&  \multicolumn{2}{m{12cm}|}{\textit{\textbf{SHARED\_STORAGE\_SIZE} --- zarezerwowana wielkość dla kontraktu}} \\ 
		&  \multicolumn{2}{m{12cm}|}{\textit{\textbf{isInitialized} --- zmienna przechowująca informację czy Agreement został zapoczątkowany }} \\ \hline
    \multirow{ 2}{*}{\textit{\textbf{Metody}}} & \multicolumn{2}{m{12cm}|}{\textit{\textbf{init()} --- metoda rozpoczynająca AgreementManagera i standard IERC20}}    \\ 
		&  \multicolumn{2}{m{12cm}|}{\textit{\textbf{getSharedStoragePointer} --- metoda pobiera aktualny wskaźnik dla SharedStorage}} \\ \hline
\end{tabular}
\end{table}
\label{OMclass3}
\end{figure}


\begin{figure}[H]
\centering
\includegraphics[width=.4\textwidth]{OMclass4}
\caption{Standard tokenów IERC20}
\begin{table}[H]
    \begin{tabular}{|l|l|l|l|}
    \hline
    \textit{\textbf{Klasa}}                & \multicolumn{2}{m{12cm}|}{\textit{AgreementManager}}                                    \\ \hline
    \textit{\textbf{Opis klasy}}                 & \multicolumn{2}{m{12cm}|}{\textit{Standard tokenów dla Ethereum}} \\ \hline
    \multirow{ 2}{*}{\textit{\textbf{Metody}}} & \multicolumn{2}{m{12cm}|}{\textit{\textbf{totalSupply()} --- zbiór dostępnego zapasu tokenów}}    \\ 
		&  \multicolumn{2}{m{12cm}|}{\textit{\textbf{balanceOf()} --- sprawdza stan konta innego adresu}} \\ 
		&  \multicolumn{2}{m{12cm}|}{\textit{\textbf{transfer()} --- przesyła wskazaną ilość tokenów na podany adres}} \\
		&  \multicolumn{2}{m{12cm}|}{\textit{\textbf{transferFrom()} ---  przesyła wskazaną ilość tokenów z podanego pierwszego adresu na podany drugi adres}} \\ 
		&  \multicolumn{2}{m{12cm}|}{\textit{\textbf{approve()} --- pozwala wydającemu token wycofać się z konta wskazaną ilość razy. Jeśli funkcja jest wywołana ponownie nadpisuje ilość pozwoleń na wycofanie się.}} \\
		&  \multicolumn{2}{m{12cm}|}{\textit{\textbf{allowance()} --- zwraca ilość jaką wydający wciąż może pobrać od właściciela.}} \\ \hline
\end{tabular}
\end{table}
\label{OMclass4}
\end{figure}


\begin{figure}[H]
\centering
\includegraphics[width=.4\textwidth]{OMclass6}
\caption{Maszyna stanowa}
\begin{table}[H]
    \begin{tabular}{|l|l|l|l|}
    \hline
    \textit{\textbf{Klasa}}                & \multicolumn{2}{m{12cm}|}{\textit{StateMachine}}                                    \\ \hline
    \textit{\textbf{Opis klasy}}                 & \multicolumn{2}{m{12cm}|}{\textit{ }} \\ \hline
    \multirow{ 2}{*}{\textit{\textbf{Zmienne}}} & \multicolumn{2}{m{12cm}|}{\textit{\textbf{abstractStates} --- stany abstrakcyjne maszyny stanowej}}    \\ 
		&  \multicolumn{2}{m{12cm}|}{\textit{\textbf{FORWARD\_GAS\_LIMIT} --- stała, która określa ile maszyna ma sobie zostawić gazu na funkcjonowanie}} \\ 
		&  \multicolumn{2}{m{12cm}|}{\textit{\textbf{currentState} --- aktualny proces stanu maszyny stanowej.}} \\
		&  \multicolumn{2}{m{12cm}|}{\textit{\textbf{hasBeenRegisteredForStateTransition} ---  przechowuje informację o dostępności tranzycji na inny stan.}} \\ \hline
	\textit{\textbf{Metody}} & \multicolumn{2}{m{12cm}|}{\textit{\textbf{fallback()} --- }}    \\ \hline
\end{tabular}
\end{table}
\label{OMclass6}
\end{figure}


\begin{figure}[H]
\centering
\includegraphics[width=.4\textwidth]{OMclass7}
\caption{Interfejs stanu}
\begin{table}[H]
    \begin{tabular}{|l|l|l|l|}
    \hline
    \textit{\textbf{Klasa}}                & \multicolumn{2}{m{12cm}|}{\textit{IState}}                                    \\ \hline
    \textit{\textbf{Opis klasy}}                 & \multicolumn{2}{m{12cm}|}{\textit{Interfejs stanu maszyny stanowej }} \\ \hline
    \multirow{ 2}{*}{\textit{\textbf{Metody}}} & \multicolumn{2}{m{12cm}|}{\textit{\textbf{getMachineState()} --- zwraca identyfikator obecnego stanu}}    \\ 
		&  \multicolumn{2}{m{12cm}|}{\textit{\textbf{getMachine()} --- zwraca identyfikator obecnej maszyny stanowej}} \\ 
		&  \multicolumn{2}{m{12cm}|}{\textit{\textbf{getMachineReachableState()} --- zwraca identyfikatory osiągalnych stanów}} \\ \hline
\end{tabular}
\end{table}
\label{OMclass7}
\end{figure}


\begin{figure}[H]
\centering
\includegraphics[width=.4\textwidth]{OMclass8}
\caption{Interfejs maszyny stanowej}
\begin{table}[H]
    \begin{tabular}{|l|l|l|l|}
    \hline
    \textit{\textbf{Klasa}}                & \multicolumn{2}{m{12cm}|}{\textit{IStateMachine}}                                    \\ \hline
    \textit{\textbf{Opis klasy}}                 & \multicolumn{2}{m{12cm}|}{\textit{Interfejs maszyny stanowej.}} \\ \hline
    \multirow{ 2}{*}{\textit{\textbf{Metody}}} & \multicolumn{2}{m{12cm}|}{\textit{\textbf{getNewState()} --- zwraca identyfikator nowego stanu}}    \\ 
		&  \multicolumn{2}{m{12cm}|}{\textit{\textbf{getCurrentState()} --- zwraca identyfikator obecnego stanu maszyny stanowej}} \\ 
		&  \multicolumn{2}{m{12cm}|}{\textit{\textbf{getListOfReachableStates()} --- zwraca listę osiągalnych stanów}} \\ \hline
\end{tabular}
\end{table}
\label{OMclass8}
\end{figure}


\begin{figure}[H]
\centering
\includegraphics[width=.4\textwidth]{OMclass9}
\caption{Interfejs bazowej maszyny stanowej}
\begin{table}[H]
    \begin{tabular}{|l|l|l|l|}
    \hline
    \textit{\textbf{Klasa}}                & \multicolumn{2}{m{12cm}|}{\textit{IStateMachineBase}}                                    \\ \hline
    \textit{\textbf{Opis klasy}}                 & \multicolumn{2}{m{12cm}|}{\textit{ }} \\ \hline
    \textit{\textbf{Metody}} & \multicolumn{2}{m{12cm}|}{\textit{\textbf{amIMachine()} --- zwraca true, jeśli metoda jest aktywna w kontekście maszyny stanowej, a false w przeciwnym wypadku.}}  \\ \hline
\end{tabular}
\end{table}
\label{OMclass9}
\end{figure}


\begin{figure}[H]
\centering
\includegraphics[width=.4\textwidth]{OMclass10}
\caption{Stan wejściowy}
\begin{table}[H]
    \begin{tabular}{|l|l|l|l|}
    \hline
    \textit{\textbf{Klasa}}                & \multicolumn{2}{m{12cm}|}{\textit{EntryState}}                                    \\ \hline
    \textit{\textbf{Opis klasy}}                 & \multicolumn{2}{m{12cm}|}{\textit{ }} \\ \hline
    \multirow{ 2}{*}{\textit{\textbf{Metody}}} & \multicolumn{2}{m{12cm}|}{\textit{\textbf{join()} --- zwraca identyfikator aplikanta próbującego dołączyć}}    \\ 
		&  \multicolumn{2}{m{12cm}|}{\textit{\textbf{accept()} --- akceptuje aplikanta \_suplicant}} \\ 
		&  \multicolumn{2}{m{12cm}|}{\textit{\textbf{getStatus()} --- zwraca stan aplikanta}} \\ \hline
\end{tabular}
\end{table}
\label{OMclass10}
\end{figure}


\begin{figure}[H]
\centering
\includegraphics[width=.4\textwidth]{OMclass11}
\caption{Stany abstrakcyjne}
\begin{table}[H]
    \begin{tabular}{|l|l|l|l|}
    \hline
    \textit{\textbf{Klasa}}                & \multicolumn{2}{m{12cm}|}{\textit{AbstractState}}                                    \\ \hline
    \textit{\textbf{Opis klasy}}                 & \multicolumn{2}{m{12cm}|}{\textit{}} \\ \hline
    \multirow{ 2}{*}{\textit{\textbf{Metody}}} & \multicolumn{2}{m{12cm}|}{\textit{\textbf{isMachine()} ---  zwraca true, jeśli metoda jest aktywna w kontekście maszyny stanowej, a false w przeciwnym wypadku.}}    \\ 
		&  \multicolumn{2}{m{12cm}|}{\textit{\textbf{setMachineNextState()} --- ustawia maszynę stanową na następny stan uprzednio przygotowany}} \\ 
		&  \multicolumn{2}{m{12cm}|}{\textit{\textbf{getMachineState()} --- zwraca identyfikator obecnego stanu}} \\ 
		&  \multicolumn{2}{m{12cm}|}{\textit{\textbf{getMachine()} --- zwraca identyfikator obecnego stanu}} \\ 
		&  \multicolumn{2}{m{12cm}|}{\textit{\textbf{getMachineReachableState()} --- zwraca identyfikator obecnego stanu}} \\ 
		&  \multicolumn{2}{m{12cm}|}{\textit{\textbf{amIMachine()} --- zwraca identyfikator obecnego stanu}} \\ \hline
\end{tabular}
\end{table}
\label{OMclass11}
\end{figure}


\begin{figure}[H]
\centering
\includegraphics[width=.4\textwidth]{OMclass12}
\caption{Stan aktywny kontraktu}
\begin{table}[H]
    \begin{tabular}{|l|l|l|l|}
    \hline
    \textit{\textbf{Klasa}}                & \multicolumn{2}{m{12cm}|}{\textit{RunningState}}                                    \\ \hline
    \textit{\textbf{Opis klasy}}                 & \multicolumn{2}{m{12cm}|}{\textit{}} \\ \hline
    \multirow{ 2}{*}{\textit{\textbf{Metody}}} & \multicolumn{2}{m{12cm}|}{\textit{\textbf{conclude()} ---  zamyka aktywny kontrakt}}    \\ 
		&  \multicolumn{2}{m{12cm}|}{\textit{\textbf{getStatus()} --- zwraca status stanu maszyny stanowej}} \\ \hline
\end{tabular}
\end{table}
\label{OMclass12}
\end{figure}


\begin{figure}[H]
\centering
\includegraphics[width=.4\textwidth]{OMclass13}
\caption{Stan usuwania kontraktu}
\begin{table}[H]
    \begin{tabular}{|l|l|l|l|}
    \hline
    \textit{\textbf{Klasa}}                & \multicolumn{2}{m{12cm}|}{\textit{RemovingState}}                                    \\ \hline
    \textit{\textbf{Opis klasy}}                 & \multicolumn{2}{m{12cm}|}{\textit{}} \\ \hline
    \multirow{ 2}{*}{\textit{\textbf{Metody}}} & \multicolumn{2}{m{12cm}|}{\textit{\textbf{remove()} ---  usuwa zamknięty kontrakt}}    \\ 
		&  \multicolumn{2}{m{12cm}|}{\textit{\textbf{getStatus()} --- zwraca status stanu maszyny stanowej}} \\ \hline
\end{tabular}
\end{table}
\label{OMclass13}
\end{figure}


\begin{figure}[H]
\centering
\includegraphics[width=.4\textwidth]{OMclass14}
\caption{?}
\begin{table}[H]
    \begin{tabular}{|l|l|l|l|}
    \hline
    \textit{\textbf{Klasa}}                & \multicolumn{2}{m{12cm}|}{\textit{StateCommons}}                                    \\ \hline
    \textit{\textbf{Opis klasy}}                 & \multicolumn{2}{m{12cm}|}{\textit{}} \\ \hline
    \multirow{ 2}{*}{\textit{\textbf{Metody}}} & \multicolumn{2}{m{12cm}|}{\textit{\textbf{withdraw()} ---  }}    \\ 
		&  \multicolumn{2}{m{12cm}|}{\textit{\textbf{expirationSensitive()} --- }} \\ 
		&  \multicolumn{2}{m{12cm}|}{\textit{\textbf{getStatus()} --- zwraca status stanu maszyny stanowej}} \\ 
		&  \multicolumn{2}{m{12cm}|}{\textit{\textbf{getSharedStoragePointer()} --- zwraca wskaźnik na wspólnej pamięci}} \\ \hline
\end{tabular}
\end{table}
\label{OMclass14}
\end{figure}


\begin{figure}[H]
\centering
\includegraphics[width=.4\textwidth]{OMclass15}
\caption{?}
\begin{table}[H]
    \begin{tabular}{|l|l|l|l|}
    \hline
    \textit{\textbf{Klasa}}                & \multicolumn{2}{m{12cm}|}{\textit{StorageUtil}}                                    \\ \hline
    \textit{\textbf{Opis klasy}}                 & \multicolumn{2}{m{12cm}|}{\textit{}} \\ \hline
    \textit{\textbf{Zmienne}} & \multicolumn{2}{m{12cm}|}{\textit{\textbf{SPOINTER\_SIZE} ---  ustawia wielkość wskaźnika}}    \\ \hline
    \multirow{ 2}{*}{\textit{\textbf{Metody}}} & \multicolumn{2}{m{12cm}|}{\textit{\textbf{setSlots()} ---  }}    \\ 
		&  \multicolumn{2}{m{12cm}|}{\textit{\textbf{getSlots()} --- }} \\ 
		&  \multicolumn{2}{m{12cm}|}{\textit{\textbf{setBytes32()} --- }} \\ 
		&  \multicolumn{2}{m{12cm}|}{\textit{\textbf{getBytes32()} --- }} \\ 
		&  \multicolumn{2}{m{12cm}|}{\textit{\textbf{setBytes()} --- }} \\ 
		&  \multicolumn{2}{m{12cm}|}{\textit{\textbf{getBytes()} --- }} \\ 
		&  \multicolumn{2}{m{12cm}|}{\textit{\textbf{setMapping()} --- }} \\ 
		&  \multicolumn{2}{m{12cm}|}{\textit{\textbf{getMapping()} --- }} \\ 
		&  \multicolumn{2}{m{12cm}|}{\textit{\textbf{setStoragePointer()} --- }} \\ 
		&  \multicolumn{2}{m{12cm}|}{\textit{\textbf{getStoragePointer()} --- }} \\ 
		&  \multicolumn{2}{m{12cm}|}{\textit{\textbf{getStoragePointerMapping()} --- }} \\ 
		&  \multicolumn{2}{m{12cm}|}{\textit{\textbf{setPositionAt()} --- }} \\ 
		&  \multicolumn{2}{m{12cm}|}{\textit{\textbf{mapSPointerTo()} --- }} \\ 
		&  \multicolumn{2}{m{12cm}|}{\textit{\textbf{relativeMove()} --- }} \\ 
		&  \multicolumn{2}{m{12cm}|}{\textit{\textbf{getAbsoluteSlotLocation()} --- }} \\ \hline
\end{tabular}
\end{table}
\label{OMclass15}
\end{figure}


\begin{figure}[H]
\centering
\includegraphics[width=.4\textwidth]{OMclass16}
\caption{Client przechowywania pamięci}
\begin{table}[H]
    \begin{tabular}{|l|l|l|l|}
    \hline
    \textit{\textbf{Klasa}}                & \multicolumn{2}{m{12cm}|}{\textit{StorageClient}}                                    \\ \hline
    \textit{\textbf{Opis klasy}}                 & \multicolumn{2}{m{12cm}|}{\textit{}} \\ \hline
    \textit{\textbf{Zmienne}} & \multicolumn{2}{m{12cm}|}{\textit{\textbf{storageManagers} --- }}    \\ \hline
	\textit{\textbf{Metody}} & \multicolumn{2}{m{12cm}|}{\textit{\textbf{getStoragePointer()} --- zwraca id wskaźnika }}     \\ \hline
\end{tabular}
\end{table}
\label{OMclass16}
\end{figure}

\clearpage
\KOMAoptions{paper=A3,pagesize}
\KOMAoptions{paper=landscape}
\recalctypearea

\subsubsection{Front-end}
\includegraphics[width=1.0\textwidth, height=0.9\textheight]{FrontEnd_Class_Diagram}

\clearpage
\KOMAoptions{paper=portrait}
\KOMAoptions{paper=A4,pagesize}
\recalctypearea
\newpage
\section{Przyrosty}

\subsection{Szablon przyrostów}

Każdy z przyrostów zawiera krótki opis celów jakie chcielismy osiągnąć w czesie jego trwania, analizę zadań i wymagań oraz wytworzone produkty lub zidentyfikowane problemy. Ich gównym zadaniem jest chronologiczna prezentacja wcześniejszych rozdziałów, w których został zawarty pełny opis analizy, architektury, problemów i ich rozwiązań, dlatego zawierają one tylko odnośniki i zwięzłe komentarze.

\subsection{Przyrost pierwszy}
\subsubsection{Cele przyrostu}

Podczas pierwszego przyrostu przygotowaliśmy analizę, jak powinny być zbudowane umowy, by wspierać możliwość wymiany dowolnymi dobrami oraz jakie gwarancje prezentują takie modele. Dokonaliśmy wyboru frameworków do testowania i wgrywania kontraktów do sieci oraz klienta sieci deweloperskiej. Przygotowaliśmy krótkie materiały szkoleniowe z pisania kontraktów w \textit{Solidity} ze wsparciem wybranych technologii. Kolejnym postawionym celem było przygotowanie wstępnej architektury naszego rozwiązania. Oprócz tego, wykonaliśmy wstępny harmonogram prac do monitorowania postępów. 

\subsubsection{Analiza}

Wynikiem rozważań na temat konstrukcji umów, są scenariusze \ref{Scenariusz A1.1}, \ref{Scenariusz A1.2}, \ref{Scenariusz A1.3}, \ref{Scenariusz A2.1}, \ref{Scenariusz A2.2}.

Jako framework do testowania i wgrywania kontraktów wybraliśmy \textit{Truffle}, ze względu na to, że było to najbardziej rozwinięte i udokumentowane tego typu rozwiązanie, jak na początek roku 2018.

Klientem sieci deweloperskiej zostało \textit{Ganache}, ze względu na prostotę obsługi oraz silne gwarancje interoperacyjności z \textit{Truffle} (są tworzone przez tę samą organizację). Niektóre z naszych rozwiązań testowaliśmy z pomocą klienta \textit{Parity}, ale ze względu na wcześniej wymienione kwestie, w pełni przenieśliśmy proces wytwórczy na \textit{Ganache}.

W czasie trwania tego przyrostu przygotowaliśmy ćwiczenia pozwalające dokonać rozpoznania możliwości wybranych przez nas technologii, wstępnego zapoznania się z językiem \textit{Solidity}. Więcej szczegółów znajduje się w załączniku \textit{SmartContracts}.

Na wstępny szkic architektury składał się \textit{AgreementManager} służacy do tworzenia i nadzorowania umów oraz autorskiego portfela z tokenami. Również ustaliliśmy wstepny model danych przechowywanych przez umowy. Przeprowadziliśmy również rozpoznanie niektórych ataków na kontrakty oraz sposobów zabezpieczania się przed nimi (patrz rodział \ref{sec:ZnanePodatnościKontraktów} punkt \textit{Re-Entrancy} i \textit{Nieoczekiwany ether}).


\subsection{Przyrost drugi}
\subsubsection{Cele przyrostu}
Zadaniami postawionymi w przyroście drugim była, implementacja pierwszej umowy A1.1 (\ref{Scenariusz A1.1}), AgreementManagera, odpowiedzialnego za rejestrowanie i tworzenie umów w systemie oraz kontraktu tokenów zgodnych ze standardem ERC20. Została przeprowadzona także analiza scenariuszy innych, bardziej złożonych typów umów.

\subsubsection{Analiza}

Pierwotne rozwiązanie by umowy wprost pobierały depozyt etheru i same nim zarządzały, nie było zgodne z dobrymi praktykami programowania, jak również wprowadzało problemy z modularnością i przenośnością systemu, dlatego wprowadziliśmy token zgodny ze standardem ERC20, który wymienia się w stosunku 1:1 z etherem oraz prowadzi listę bilansów każdego adresu w sieci Ethereum. Pozwoliło to przenieść krytyczną funkcjonalnośc przesyłania tokenów z umów do osobnego kontraktu, który można było niezależnie testować pod kątem bezpieczeństwa.

W tym przyroście prezentujemy dalszy ciąg naszych rozważań na temat konstrukcji umów, których wynikiem są scenariusze umów \ref{Scenariusz A3.1}, \ref{Scenariusz A3.2}, \ref{Scenariusz A3.3}, \ref{Scenariusz A4.1}, \ref{Scenariusz A4.2}, \ref{Scenariusz PA1}, \ref{Scenariusz PA2}, \ref{Scenariusz glosowania swiadkow}

\subsubsection{Testy}

\textbf{Środowisko testów}

Aplikacja była testowana za pomocą środowiska developerskiego \textit{Truffle} w którym testy tworzymy za pomocą javascriptowego środkowska testowego \textit{Mocha} oraz \textit{Chai} do tworzenia assercji.
Do testów potrzebujemy środowisko w którym możemy uruchomić kontrakty. Do tych zadań używamy środowiska \textit{Ganache}.\\
\textbf{Scenariusze testowe}
\begin{table}[H]
    \begin{tabular}{|l|l|}
\hline
    	\multicolumn{1}{|m{6,5cm}|}{\textit{Umowa A1.1: Standardowa ścieżka}}                 & \multicolumn{1}{m{9cm}|}{\textit{Sprawdzamy stworzenie umowy A1.1 z daną ceną w początkowym stanie. Użytkownik wpłaca pieniądze do kontraktu. Twórca wypłaca pieniądze. Każdy z użytkowników zamyka umowę. Umowa zostaje zakończona.}} \\ \hline
    	\multicolumn{1}{|m{6,5cm}|}{\textit{Umowa A1.1: Właściwości dołączania}} & \multicolumn{1}{m{9cm}|}{\textit{Sprawdzamy różne scenariusze dołączania do umowy. Sprawdzamy czy umowa posiada do maksymalnie trzech użytkowników. Sprawdzamy czy twórca lub suplikant nie próbuje dołączyć dwa razy do tej samej umowy.}}    \\ \hline
    	\multicolumn{1}{|m{6,5cm}|}{\textit{Umowa A1.1: Ograniczenia akecptacji użytkowników}} & \multicolumn{1}{m{9cm}|}{\textit{Sprawdzamy możliwość akceptacji użytkowników. Sprawdzamy czy twórca nie próbuje akceptować samego siebie do umowy. Sprawdzamy, czy tylko twórca ma możliwość akceptacji próśb o dołączenie do umowy. Sprawdzamy również czy nie ma możliwości akceptacji próśb, gdy umowa jest w trakcie realizacji.}}                                                \\ \hline
   	\multicolumn{1}{|m{6,5cm}|}{\textit{Umowa A1.1: Właściwości akceptacji zależne od stanu}} & \multicolumn{1}{m{9cm}|}{\textit{Sprawdzamy czy twórca nie ma możliwości podwójnie zaakceptować prośby o dołączenie od tej samej osoby. Sprawdzamy również, czy nie ma możliwości dołączania do umowy gdy jest ona zamknięta.}}                                                \\ \hline
	\multicolumn{1}{|m{6,5cm}|}{\textit{Umowa A1.1: właściwości zamknięcia umowy}} & \multicolumn{1}{m{9cm}|}{\textit{Testujemy, czy obce adresy nie mają możliwości zakończenia umowy. Sprawdzamy również, czy niezaakceptowany adres nie ma możliwości zamknięcia umowy oraz czy nie ma możliwości podówjnie zakończyć umowy.}} \\ \hline
	\multicolumn{1}{|m{6,5cm}|}{\textit{Interkacja umowy z funkcją remove}}    & \multicolumn{1}{m{9cm}|}{\textit{Sprawdzamy czy różne funkcje takie jak join czy accept nie mają wpływu na usunięcie całego kontratktu. Sprawdzamy równiez czy nie ma możliwości usunąć umowy gdy jest w stanie realizacji.}} \\ \hline
	\multicolumn{1}{|m{6,5cm}|}{\textit{Umowa A1.1: Właściwości wycofania}}  & \multicolumn{1}{m{9cm}|}{\textit{W tym scenraiuszu testujemy wycofywanie się użytkownika z umowy. Sprawdzamy, czy sprzedawca nie może wycofać się z umowy. Wraz z wycofywaniem się, sprawdzamy czy jest możliwość wypłacenia tokenów. Sprawdzamy czy po staniu się kupującym nie może wycofać się z umowy.}}                                          \\ \hline
	\multicolumn{1}{|m{6,5cm}|}{\textit{AgreementManager: Tworzenie umów}}  & \multicolumn{1}{m{9cm}|}{\textit{Sprawdzamy czy AgreementManager pomyślnie tworzy umowę. Sprawdzamy również czy umowa istnieje pod danym adresem.}}                                          \\ \hline
	\multicolumn{1}{|m{6,5cm}|}{\textit{AgreementManager: Właściwości wyszukiwania}}  & \multicolumn{1}{m{9cm}|}{\textit{Sprawdzamy właściwości wyszukiwania AgreementManagera. Sprawdzamy czy AgreementManager pomyślnie wyszukuje pomyślnie daną umowę.}}                                          \\ \hline
\end{tabular}
\end{table}

\subsubsection{Implementacja}

Produkty:
\begin{itemize}
	\item \textit{StandardECMToken} -- klasa w \ref{OMclass4}
	\item \textit{AgreementManager} -- klasy w \ref{OMclass1}, \ref{OMclass2}
	\item \textit{Agreement} -- klasa w \ref{OMclass3}, jednak z czasem uległa zmianie
\end{itemize}


\subsection{Przyrost trzeci}
\subsubsection{Cele przyrostu}

Głównym zadaniem było stworzenie UI do wygodnego wyszukiwania umów oraz różnych widoków potrzebnych do tworzenia, czy też przeglądania umów.
W tym przyroście również przygotowaliśmy system do dodania umowy A1.2.

\subsubsection{Analiza}

W celu bardziej przystępnej prezentacji naszych rozwiązań, przygotowaliśmy prosty graficzny interfejs użytkownika, możliwy do uruchomienia lokalnie na maszynie użytkownika, który z pomocą biblioteki \textit{Web3js} komunikuje się z klientem Ethereum, a do celowo z naszym systemem.

W celu zwiększenia możliwości naszego systemu rozpoczeliśmy implementację umowy A1.2 (\ref{Scenariusz A1.2}).

\subsubsection{Napotkane problemy}

Próba implementacji umowy A1.2, przy zapewnieniu modularności i rozszerzalności systemu, skonfrontowała nas z nigdzie nie poruszanymi własnościami języka Solidity i kontraktami Ethereum. Problemy wynikające z nich orbitowały one w okół narzucanego przez Solidity modelu pamięci \textit{storage} i niemożliwości kompozycji lub agregacji kontraktów. Zostały one opisane w rozdziałach \ref{sec:ModelPamięciSolidity} i \ref{sec:ModularyzacjaSystemu},a ich rozwiązania w \ref{sec:GenerycznaMaszynaStanowa}, \ref{sec:AbstrakcjaNadPamięciąStorage} i \ref{sec:UmowaJakoMaszynaStanowa}.

\subsubsection{Testy}
\begin{table}[H]
    \begin{tabular}{|l|l|}
\hline
    \multicolumn{1}{|m{6,5cm}|}{\textit{Standard ECM Token: transfer}}                 & \multicolumn{1}{m{9cm}|}{\textit{Sprawdzamy różne scenariusze przepływu tokenów pomiędzy kontami. Sprawdzamy poprawność transakcji.Sprawdzamy również czy nie ma możliwości wysyłać tokenów do adresu danego tokenu.}} \\ \hline
    \multicolumn{1}{|m{6,5cm}|}{\textit{Standard ECM Token: wpłata oraz wypłata pieniędzy}} & \multicolumn{1}{m{9cm}|}{\textit{Sprawdzamy poprawność wpłat i wypłat tokenów. Funkcje testowane są również na kilku użytkownikach oraz samego zdarzenia transferu.}}    \\ \hline
    \multicolumn{1}{|m{6,5cm}|}{\textit{Maszyna stanowa}} & \multicolumn{1}{m{9cm}|}{\textit{W tym scenariuszu sprawdzamy poprawaność przejścia stanu.}}                \\ \hline
    \multicolumn{1}{|m{6,5cm}|}{\textit{Maszyna stanowa z pamięcią}} & \multicolumn{1}{m{9cm}|}{\textit{W tym scenariuszu sprawdzamy poprawność zapisywania różnych stanów w pamięci}}                                                \\ \hline
    \multicolumn{1}{|m{6,5cm}|}{\textit{Obiekt Storage}} & \multicolumn{1}{m{9cm}|}{\textit{Testujemy inicjowanie obiektu Storage}} \\ \hline
    \multicolumn{1}{|m{6,5cm}|}{\textit{Zakres pamięci}}    & \multicolumn{1}{m{9cm}|}{\textit{Testujemy zachowanie pamięci poza zakresem.}} \\ \hline
    \multicolumn{1}{|m{6,5cm}|}{\textit{Pamięć: podstawy działania}}    & \multicolumn{1}{m{9cm}|}{\textit{Sprawdzamy poprawność funkcji, których celem jest zapisywanie do danego slotu w pamięci.}} \\ \hline
                                    
\end{tabular}
\end{table}

\subsubsection{Implementacja}
Produkty:
\begin{itemize}
	\item Abstrakcja na pamięcią \textit{storage} -- rozdział \ref{sec:AbstrakcjaNadPamięciąStorage}
	\item Generyczna maszyna stanowa -- rozdział \ref{sec:GenerycznaMaszynaStanowa}
	\item Umowa jako maszyna stanowa -- rozdział \ref{sec:UmowaJakoMaszynaStanowa}
	\item Widok i serwis wyszukiwania umów
	\item Widok i serwis do komunikacji z umowami
	\item widok domowy
	\item widok do tworzenia umów
\end{itemize}

\newpage
\section{Instrukcja obsługi}

\subsection{Konfiguracja środowiska deweloperskiego}

\subsubsection{Back-end}

Środowisko deweloperskie składa się z:
\begin{itemize}
	\item \textbf{npm} w wersji 6.1.0
	\item \textbf{nodejs} w wersji 8.15.0
	\item \textbf{truffle} w wersji 4.1.14
	\item \textbf{solidity} w wersji 0.4.24
	\item \textbf{ganache} w wersji 1.2.0
\end{itemize}

Po zainstalowaniu wymaganych programów, uruchomieniu klienta Ganache i przejściu
do głównego katalogu projektu, mamy do dyspozycji następujące funkcje:
\begin{itemize}
	\item \textbf{truffle test} -- uruchamia zestaw testów
	\item \textbf{truffle compile} -- kompiluje kontrakty do byte-codu i generuje ich ABI
	\item \textbf{truffle migrate} -- wykonuje skrypt deployujący skompilowane kontrakty do wskazanej w konfiguracji sieci
\end{itemize}

\subsubsection{Front-end}
Front-end tutaj przedstwaiony jest tylko przykładem który w przyszłości różni developerzy będą mogli modyfikowa i udoskanalac
Środowsiko developerskie składa się z:
\begin{itemize}
	\item \textbf{npm} w wersji 6.1.0
	\item \textbf{angular} w wersji 6.0.8
\end{itemize}

Po zainstalowaniu wymaganych programów i przejściu do głównego katalogu by korzystać z aplikacji potrzebujemy:
\begin{itemize}
	\item \textbf{npm install} -- komenda która pozwala do zainstalowana paczek oraz paczek zależnych od niej w celu uruchumienia apliakcji
	\item \textbf{ng serve} -- buduje oraz dostarcza aplikację gotową do przeglądania
	\item Po pomyślnym zbudowaniu aplikacji, otwieramy przeglądarke internetową np. \textbf{Google Chrome}.
	\item W pasku adresu przeglądarki wpisujemy /textbf{http://localhost:4200/} w celu przejścia do naszej aplikacji.
	\item \textbf{ng test} -- uruchamia zestaw testów. W celu poprawnego działania zalecane jest używanie jako domyślnej przeglądarki {Google Chrome}
\end{itemize}

\subsection{Obsługa aplikacji}

Pierwszy z dostępnych paneli jest panel wyszukiwnia w którym możemy bez problemu wyszuka bez problemu interesującą nas umowę.
W tym wydaniu możem wyszukać umowę wpisując rózne parametry tj.
\begin{itemize}
	\item \textbf{name:} aby wyszukać umowę po nazwie
	\item \textbf{price:} aby wyszukać umowę po cenie
	\item \textbf{expirationDate:} aby wyszukać po umowę po dacie wygaśniecia umowy
\end{itemize}

Drugim z dostępnych paneli jest panel użytkownika. W tym panelu możemy znaleźć adres portfelu na którym będziemy operować podczas użytkownia aplikacji. 
Możemy tu również znaleźć własne stworzone umowy oraz umowy w których uczestniczymy.

Trzecim panel jest panel tworzenia umów. W tym panelu możemy stworzyć umowę danego typu. W tym wydaniu dostępna będzie umowa A1.1. By stworzyć umowę należy podać datę,nazwę,opis oraz cenę.
Następnie klikamy przycisk \textbf{Submit} aby dodać umowę do systemu.

Przy każdym wyszukaniu umowy mamy możliwość spojrzeć na szczegółową stronę umowy. 
Ta strona pozwala nam przejrzeć wszystkie detale umowy, widziwmy również adresy twórcy oraz petenta.

\newpage
\section{Raport końcowy}

W ramach tej pracy udało nam się przygotować:
\begin{itemize}
	\item analizę scenariuszy umów
	\item implementację podsystemu zarządzającą umowami
	\item implementację umowy A1.1
	\item implementacja prostego interfejsu graficznego do prezentacji możliwości naszego systemu
	\item Implementacja serwisu do wyszukiwania umów
	\item Stworzenie testów dla umowy A1.1
	\item Stworzenie testów dla serwisu do wyszukiwania umów
	\item Stworzenie testów dla podsystemu zarządzającego umowami
	\item analiza jak system ECMarket może przyczynić się do stworzenia bardziej egalitarnego społeczeństwa
	\item omówienie problemu centralizacji Ethereum oraz inncy kryptowalut, czynników wpływających na to zjawisko, jak również potencjalnych rozwiązań tej kwestii
	\item przygotowanie wykazu zanych wektorów ataków na kontrakty
	\item zdiagnozowanie wpływu braku możliwości agregacji lub kompozycji kontraktów na sposób projektowania systemów w Ethereum
	\item zdiagnozowanie wpływu modelu pamięci \textit{storage} w języku Solidity na sposób projektowania systemów w Ethereum
	\item rozwiązanie problemu braku kompozycji i agregacji kontraktów w postaci autorskiej architektury maszyny stanowej
	\item rozwiązanie problemu modelu pamięci \textit{storage} w języku Solidity przez skonstruowanie abstrakcji pozwalającej do niej pisać jak do pliku
\end{itemize}

 

\newpage
\section{Wkład własny w projekt}

% w osobnym branchu dajcie swójw wkład

\newpage
\section{Podsumowanie}

Zestaw cech i własności oferowanych przez technologię Ethereum, w szczególności takich jak, wielopłaszczyznowa decentralizacja połączona z wirtualną maszyną zdolną do wykonania dowolnego kodu, pozwala na eksplorację nowych ideii i konstrukcji, trudnych do realizacji w świecie sprzed wschodu krypto-ekonomii. Jedną z nich jest możliwość tworzenia bardziej egalitarnych aplikacji i organizacji, do których grona aspiruje nasz system -- ECMarket, niż istniejące scentralizowane rozwiązania.

Jednak egzotyka tej technologii i jej nie w pełni zbadana natura, jest źródłem nietrywialnych problemów. Jedną z omawianych w tej pracy, krytycznych dla naszej aplikacjii kwestii, jest siła gwarancji decentralizacji Ethereum. Przedstawiamy od czego w obecnej chwili zależy ta własności, jak wygląda ona na tle innnych kryptowalut oraz jakie środki zaradcze proponuje środowisko Ethereum. Również w ramach naszego projektu odkryliśmy nierozpoznane wcześniej problemy, dotyczące sposobów projektowania i implementacji kontraktów oraz przedstawiliśmy ich rozwiązania.

Pomimo licznych przeszkód i spornych kwestii, optymistycznie podchodzimy do możliwości rozwoju naszego systemu w ramach ekosystemu Ethereum i dostrzegamy znaczący potencjał tej technologii.

\newpage
\section{Słownik pojęć}
\begin{itemize}
	\item Platforma Handlowa -- aplikacje agregujące oferty handlowe, pośredniczące w wykonaniu opłat i dostarczające narzędzia od oceny kupujących i sprzedających. 
	\item Ekosystem Ethereum (zespół technologii Ethereum) -- Ethereum jako środowisko wykonawcze oparte na blockchainie, rozproszona przestrzeń dyskowa Swarm, system komunikatów Whisper, biblioteki web3 i aplikacje klienckie Ethereum.
	\item  Ethereum -- środowisko wykonawcze oparte na blockchainie dla inteligentnych kontraktów.
	\item  Blockchain -- struktura danych mogąca przechowywać dowolne dane w postaci bloków, skonstruowana na zasadzie listy (lub w pewnym stopniu drzewa), gdzie wskaźnikami do poprzednich bloków są ich kryptograficzne hasze.
	\item Patricia Merkle Trie -- struktura danych mogąca przechowywać dowolne dane, skonstruowana na zasadzie drzewa, gdzie wskaźnikami na rodziców są ich kryptograficzne hasze.
	\item Mainnet -- główna sieć Ethereum o id równym 1. Na niej odbywają się transakcje z użyciem etheru posiadającym wartość dla użytkowników.
	\item Swarm -- zdecentralizowana przestrzeń dyskowa oparta na drzewach Merkle, zintegrowana z Ethereum. 
	\item Whisper -- zdecentralizowany, wysoce anonimowy system komunikatów zintegrowany z Ethereum.
	\item Wirtualna Maszyna Ethereum -- część Ethereum wykonująca bytecode skompilowanych kontraktów.
	\item Ether -- jednostka płatności/możliwości wykonania kodu. Jest podstawową walutą w ekosystemie Ethereum. 1 ether dzieli się na $10$\textsuperscript{18} wei.
	\item Solidty -- jezyk zorientowany kontraktowo. W nim została wykonana gółwna cześć systemu wgrywana do sieci Ethereum.
	\item Krytowaluty -- rozproszony system księgowy bazujący na kryptografii, przechowujący informację o stanie posiadania w umownych jednostkach.
	\item Zupełność w sensie Turinga -- cecha maszyny lub języka programowania, polegająca na tym, że można za jego pomocą rozwiązać identyczną klasę problemów obliczeniowych, jak na uproszczonym modelu programowalnego komputera zwanego maszyną Turinga.
	\item Korzyści skali -- korzyści płynące z produkcji masowej. Charakteryzują się malejącym kosztem produkcji w przeliczeniu na jeden egzemplarz
	\item Krypto-ekonomia -- W swojej najprostszej formie kryptoekonomia odnosi się do zastosowań ekonomii (poprzez zachęcanie) i kryptografii (poprzez szyfrowanie) do zaprojektowania bezpiecznego systemu lub sieci z predefiniowanymi pożądanymi właściwościami.
\end{itemize}

\newpage
\section{Załączniki}

\begin{enumerate}
	\item SmartContracts.pdf - ćwiczenia z Solidity
\end{enumerate}

\newpage
\bibliographystyle{IEEEtranS}
\bibliography{bibliografia}
\end{document}