\documentclass[11pt,oneside]{article}
\usepackage[utf8]{inputenc}
\usepackage[polish]{babel}
\usepackage[T1]{fontenc}
\usepackage[a4paper,width=170mm,top=18mm,bottom=22mm,includeheadfoot]{geometry}
\usepackage{graphicx}
\usepackage{comment}
\usepackage{multirow}
\usepackage{courier}
\usepackage{float}
\usepackage{array}
\usepackage{amsmath}
\usepackage{url}
\graphicspath{ {./images/} }

\begin{document}

\begin{figure}[H]

\centering
\includegraphics[width=1.0\textwidth]{logopjwstk}

\begin{Huge}
\begin{center}
	\texttt{\textbf {KARTA PROJEKTU}}
\end{center}
\end{Huge}

\includegraphics[width=0.94\textwidth]{cheat1}
\includegraphics[width=0.94\textwidth]{cheat2}
\end{figure}

\begin{figure}[H]
\includegraphics[width=.4\textwidth]{cheat3}
\end{figure}

\setcounter{tocdepth}{2}
\tableofcontents

\newpage

\section{Wstęp}
Na fali pojawiających się nowych zdecentralizowanych aplikacji,
protokołów komunikacj p2p, algorytmów rozproszonego konsensusu i ideii Internetu 3.0,
narodziła się w wśród nas koncepcja systemu wspierającego i nadzorującego szeroko rozumianą wymianę obiektów,
w szczególności dóbr fizycznych, który wpasowywałby się do tego nowego, wspaniałego świata.
Porwani przez tę wizję, poświęcamy tę pracą pierwszemu podstawowemu wydaniu systemu ECMarket,
ktore będzie wstępem do stworzenia analogu do serwisów aukcyjnych typu Allegro czy Ebay w uniwersum
aplikacji zdecentralizowanych.

\newpage
\section{Podziękowania}



\newpage
\section{Opis Problemu}

\subsection{Prezentacja Problemu}
W związku z rosnącą centralizacją Internetu, problemem cenzury oraz spadkiem zaufania do
instytucji państwa\cite{work:trustPOL,website:trustUSA} jak i korporacji, powstało wiele projektów mających
na celu rozproszenie tej koncentracji władzy, jednak przez długi czas rozwiązania te nie zyskiwały
duzej popularności poza specjalistycznymi dziedzinami.\\
Odwilż w tej kwestii przyniósł protokół Kademila\cite{work:Kademila}, na której oparte zostały sieci wymiany plików takie
jak BitTorrent, czy Kad Network. Ich funkcjonalność ograniczała się do wymiany plików, jednak nie była
pozbawiona wad takich jak niska żywotność niepopularnych plików, egoistyczne zachowania klientów sieci
(leeching), brak odporności na ataki DDoS (Distributed Denial-of-Service).\\
Przełom nastąpił w 2009 w raz z prezentacją technologi Bitcoin\cite{work:BitcoinPaper}, która oferowała roziązanie wymienionych problemów, z którymi borykały się poprzednie rozwiązania. Zastosowanie mechanizmu PoW (Proof-of-Work), jako zabezpieczenia przed atakami DDos i Sybili, użycie nowej struktury --- blockchainu, jako rozproszonej bazy danych, a przede wszystki, innowacyjne jak na tamte czasy, zastosowanie zachęty finansowej dla klientów zbierających transakcje i przechowujących blockchain, by zachować żywotność sieci, storzyło mieszankę, która zagwarantowała sukces tej technologii. Dowodem tego jest osiągnięcie przez Bitcoina pod koniec listopada 2013 wartości 1200 USD/BTC.\\
Te wydarzenia dały początek nowej dziedzinie wiedzy nazawanej krypto-ekonomią oraz zapoczątkowały eksplozję nowych kryptowalut i rozwiązań pochodnych. Jednym z nowych rozwiązań, które pojawiło się na horyzoncie na przełomie 2013 i 2014 jest Ethereum\cite{website:ethereum2018whitepaper}. Rozwiązanie zaproponowane przez Vitalika Buterina, Gavina Wooda oraz Josepha Lubina, oferuje rozproszoną platformę obliczeniową z systemem operacyjnym, obsługującym inteligentne kontrakty (smart contracts). Podobnie jak Bitcoin oferuje możliwość przesyłania wewnętrznych tokenów (w przypadku Ethereum nazywanych Etherem) między użytkownikami, a dzięki zmodyfikowanemu konsensusowi Nakamoto (Ethereum 2.0 będzie używać innych mechanizmów), daje podobne, a nawet silniejsze gwarancje bezpieczeństawa. Najbardziej istotna różnica występuje jednak w systemie skryptów (inteligentnych kontraktów), te oferowane przez Ethereum, w przeciwieństwie do Bitcoina, są kompletne w sensie Turinga i mogą posiadać bardziej skomplikowany stan.

\subsection{Proponowane Rozwiązanie}
Dzięki takiemu zestawowi cech, Ethereum stanowi dobrą platformę, do budowania własnych, wyspecjalizowanych systemów, które są bezpieczne i niezależne od potencjalnie nieuczciwych stron trzecich. Szczególnym przypadkiem się tutaj platformy handlowe, które agregują oferty sprzedających, pośredniczą w wykonaniu opłat i dostarczają narzędzia od oceny kupujących i sprzedających. W przypadku scentralizowanych wersji takich platform, bardzo częstym problemem jest faworyzowanie pewnych grup sprzedawców, różne stopy opłat dla różnych uczestników i stronniczy moderatorzy.\\
Mając na uwadze wymienione wcześniej fakty, proponowanym przez nas rozwiązaniem problemów z centralizacją i nieuczciwymi pośrednikami, przynajmniej w sferze platform handlowych, jest stworzenie takiej aplikacji w ekosystemie Ethereum i zapewnienie, że w późniejszych etapach rozwoju, będzie ona rozwijana w ścisłej współpracy ze społecznością sieci. Z tego powodu chcielibyśmy zaprezentować nasze autorskie podejście w postaci projektu ECMarket.

\subsection{Rich Picture}

\subsection{Konkurencja}
ECMarket ma na chwilę obecną kilka wartych uwagi konkurentów, którzy operują w trochę innych, ale pokrywających się z naszym obszarach.
\begin{itemize}
	\item \textit{uPort} Pozwala użytkownikom tylko zarejestrować własną tożsamość w Ethereum i zarządzać własnymi kluczami i danymi w bezpiecznym portfelu \cite{website:uPort}.
	\item \textit{Ujo Music} Handluje tylko utworami muzycznymi, nie posiada również rozbudowanego systemu konktraktów \cite{website:UjoMusic}.
	\item \textit{IDEX} Oferuje tylko usługi wymiany tokenów ERC20 \cite{website:IDEX}.
	\item \textit{OpenBazaar} Oferuje tylko jeden rozdaj umowy (umowa między dwiema stronami z depozytem)\cite{website:openbazaar}. 
\end{itemize}
Nasza oferta stanowi swoiste połączenie i wariację niektórych własności oferowanych przez wyżej wymienionych konkurentów, dlatego sądzimy, że mamy szansę wypromować się na tym rynku.


\newpage
\section{Planowanie}

\subsection{Cele i Zakres Projektu}

Celem projektu jest dostarczenie odpornej na cenzurę, ogólnodostępnej, otwartej platformy, w ramach której możliwe jest przeprowadzanie kupna oraz sprzedaży dóbr w sposób niezależny od osób trzecich lub w ramach grup handlowych, tworzonych przez użytkowników systemu z określanymi przez nich zasadami.
W pierwszym stadium systemu, realizowanym w ramach tego projektu, planujemy pobierać mały, stały procent od kwoty przesyłanej w transakcjach między użytkownikami, by zapewnić utrzymanie i dalszy rozwój projektu.
Naszą platformę kierujemy do wszystkich pragnących niezależnego i otwartego systemu do handlu.


\subsection{Kontekst Projektu}

System będzie docelowo uruchomiony na Testnecie Ethereum i będzie dostarczał rozproszoną, odporną na cenzurę, na ile pozwala na to protokół\cite{website:TheDAOHack, website:ethereum2018whitepaper}, platformę do handlu.
Użytkownicy w pierwszym stadium rozwoju platformy będą się dzielić na Deweloperów i na Klientów. Zadaniem Deweloperów będzie dostarczanie kodu i jego deployment oraz ponoszenie kosztów tych operacji. Deweloperzy będą również beneficjentami oprocentowania transakcji. Klientami nazywamy wszystkich użytkowników, którzy dokonują transakcji. Klienci mogą zarządzać swoimi portfelami, oraz mogą wchodzić w interakcje z dostępnymi dla nich umowami.Liczba Klientów nie jest w żaden sposób ograniczona.

\subsection{Analiza Biznesowa}
Pierwsze wydanie systemu dostarczy klientom zestaw umów gotowych umów do użycia i system ocen, który będzie dawać nam przewagę nad konkurencją w takich kwestiach jak:
\begin{itemize}
\item Wiarygodny system ocen --- stanowiący kluczową wartość oferowaną przez nasze rozwiązanie. W naszym systemie nie występuje administrator, to też niezwykle istotne było stworzenie środowiska promującego bycie uczciwym. Podobne rozwiązania możemy zaobserwować w przeróżnych serwisach aukcyjnych, gdzie stanowią pewnego rodzaju informację dla kupującego o tym czy może ufać sprzedającemu. Nasze rozwiązanie pozwala na ocenienie drugiej strony umowy --- takowej oceny dokonuje także sprzedający, a nie tylko kupujący. Dodatkowo ocenom tym zostają nadane odpowiednie wagi tak by zapobiec próbom wpłynięcia na system ocen np.\ poprzez wchodzenie w małe transakcje z dużą ilością kont powiązanych ze sprzedającym. Wagi przyznawane są na podstawie ocen posiadanych przez stronę wystawiającą takową ocenę, a także jej wiarygodności dla społeczności platformy wyrażoną poprzez oceny pozytywne oraz ogólną ilość ocen.
\item Niezależność od walut FIAT --- na platformie płatności odbywać się będą przy pomocy Ether'u (ETH), kryptowaluty, której wartość jest niezależna od żadnego centralnego organu finansowego. Naszych środków nie da się zamrozić tak jak np.\ w przypadku Peso Argentyńskiego (ARS) w 2001 roku podczas kryzysu finansowego. Nie grozi nam także hiperinflacja, pieniędzy nie da się po prostu dodrukować przez specyfikę pozyskiwania Ether'u (ETH).
\item Niezależność od stron trzecich --- transakcje odbywają się w ramach zdecentralizowanej sieci Ethereum, gdzie nie występuje żaden administrator czy istytucja będąca w stanie zablokować bądź cofnąć naszą transakcję. Gwarantuje to poczucie bezpieczeństwa i niezależności wyróżniające nas na tle konkurencyjnych serwisów nieopartych na technologii blockchain, gdzie strona trzecia już występuje. Zwiększa to także bezpieczeństwo naszych danych osobowych zachowując większą anonimowość.
\item Bezpieczne wykonanie umów --- ???
\item Różnorodność umów --- sposobów przeprowadzenia transakcji - Przygotowaliśmy 11 przykładowych modeli umów z czego 2 znajdą się w pierwszym wydaniu systemu. Gwarantuje to dostosowanie systemu do potrzeb użytkowników. Dodatkowo ilość ta mogłaby się powiększyć wraz z powiększeniem bazy użytkowników. 
\item Globalny zasięg ofert --- sieć Ethereum wiąże się z brakiem ograniczenia co do lokalizacji na kuli ziemskiej. Korzystać z niej można z dowolnego zakątku, wystarczy jedynie połączenie z internetem. Ether może nabyć każdy co stanowi jedyny wymóg by korzystać z naszej platformy. Daje to o wiele większą wszechstronnoć od większości platform do handlu ograniczonych jedynie do rynku lokalnego. Konkurować z nami pod tym względem mogliby jedynie giganci tacy jak np.\ Amazon czy eBay.
\item Budowanie nowych rozwiązań --- unikalna cecha wyróżniająca nas na rynku. Jestemy otwarci na społeczność tworząca się wokół naszej platformy w znacznie szerszy sposób niż nieblockchainowa konkurencja. Po wypuszczeniu produktu żyłby on własnym życiem kontrolowany w całości przez społeczność posiadającą możliwość wprowadzania modyfikacji, które mogłyby zostać przez większość zaakceptowane bądź odrzucone co stanowi gwarancję dostosowania ECMarket'u do potrzeb potencjalnego użytkownika. Jednakże dodanie funkcjonalności związanych z tą cechą, jest zaplanowane na przyszłe wydania
\end{itemize}


\subsection{Rozważane Strategie}
Przy określaniu paradygmatu, metodologii i procesu wytwarzania oprogramowania, które zastosujemy w naszym projekcie rozważaliśmy następujące strategie

\begin{itemize}
	\item Ewolucyjno-przyrostowy \\
	Cechy:
	\begin{itemize}
		\item nie wymaga pełnej specyfikacji problemu
		\item można regulować intensywność i czas przyrostów
		\item pozwala na zmianę wymagań i dostosowywanie poprzednich przyrostów
		\item funkcjonalności systemu muszą dawać rozbić się na podfunkcjonalności
		\item proces zajmuje dużo czasu
	\end{itemize}
	\item Scrum \\
	Cechy:
	\begin{itemize}
		\item szybka adaptacja do zmian wymagań
		\item wymagana stała komunikacja w zespole
		\item nieduże przyrosty w małych odstępach
		\item wymaga podobnych umiejętności wszystkich członków zespołu
		\item samozarządzanie się zespołu
		\item szybko tworzone oprogramowanie
	\end{itemize}
	\item Prototypowanie \\
	Cechy:
	\begin{itemize}
		\item wspomaga identyfikację wymagań
		\item może przeszkodzić w kompleksowej analizie problemu
		\item duże nakłady czasu na przygotowanie prototypu
		\item logiczna walidacja problemu
		\item najlepiej się sprawdza przy budowaniu UI
	\end{itemize}
\end{itemize}

\nocite{academy:BYT, wiki:Scrum, wiki:SoftPrototyping}

\subsection{Wybór Strategii i Argumentacja}

Metodologią, którą rozważaliśmy na początku był Scrum, jednak wymóg stałej komunikacji i intensywności pracy, zdyskwalifikował ją dość szybko. Problemem również była bardzo mocno zróżnicowana wśród zespołu znajomość technologii (Ethereum). Wahaliśmy się, czy nie użyć protypowania, jednakże mieliśmy już na samym początku, w pewnym stopniu określone wymagania, więc zdecydowaliśmy się na bardziej znaną nam metodę ewolucyjno-przyrostową.

\subsection{Plan Zapewnienia Jakości}
Jakość w naszym produkcie jest zapewniona dzięki wykorzystaniu metodyki zwinnej TDD (Test Driven Development). Zdecydowaliśmy się na taką metodykę, ponieważ nasz system operuje na kryptowalutach oraz na dobrach każdego użytkownika. Sama idea zdecentralizowanej platformy handlowej jest mało popularnym zagadnieniem, więc aby zapewnić użytkowników do zaufania nowej technologii i platformy, musimy dostarczyć wysoką jakość naszego produktu.

\subsection{Udziałowcy}


\begin{table}[H]
\begin{tabular}{|l|m{11cm}|}
\hline
\multicolumn{2}{|l|}{\textit{\textbf{Karta udziałowca}}}                               \\ \hline
\textit{Identyfikator:}  & \textit{UNP 01}                                             \\ \hline
\textit{Nazwa:}          & \textit{PJATK}                                              \\ \hline
\textit{Opis:}           & \textit{PJATK umożliwiło nam wykonanie pracy inżynierskiej} \\ \hline
\textit{Typ udziałowca:} & \textit{Nieożywiony pośredni}                               \\ \hline
\textit{Punkt widzenia:} & \textit{Ekonomicznej}                                       \\ \hline
\textit{Ograniczenia:}   & \textit{PJATK nie może narzucać rozwiązań systemowych}      \\ \hline
\textit{Wymagania:}      & \textit{}                                                   \\ \hline
\end{tabular}
\end{table}

\begin{table}[H]
\begin{tabular}{|l|m{11cm}|}
\hline
\multicolumn{2}{|l|}{\textit{\textbf{Karta udziałowca}}}                               \\ \hline
\textit{Identyfikator:}  & \textit{UNB 02}                                             \\ \hline
\textit{Nazwa:}          & \textit{Sieć Ethereum}                                              \\ \hline
\textit{Opis:}           & \textit{Zdecentralizowana sieć która pozwala nam wykonywać smart contracty.} \\ \hline
\textit{Typ udziałowca:} & \textit{Nieożywiony bezpośredni}                               \\ \hline
\textit{Punkt widzenia:} & \textit{ekonomiczny, techniczny}                                       \\ \hline
\textit{Ograniczenia:}   & \textit{brak}      \\ \hline
\textit{Wymagania:}      & \textit{}                                                   \\ \hline
\end{tabular}
\end{table}

\begin{table}[H]
\begin{tabular}{|l|m{11cm}|}
\hline
\multicolumn{2}{|l|}{\textit{\textbf{Karta udziałowca}}}                               \\ \hline
\textit{Identyfikator:}  & \textit{UOB 01}                                             \\ \hline
\textit{Nazwa:}          & \textit{Zespół Projektowy}                                              \\ \hline
\textit{Opis:}           & \textit{Członkowie zespołu wykonującego system ECMarket.} \\ \hline
\textit{Typ udziałowca:} & \textit{Ożywiony bezpośredni}                               \\ \hline
\textit{Punkt widzenia:} & \textit{techniczny, wykonawczy}                                       \\ \hline
\textit{Ograniczenia:}   & \textit{brak}      \\ \hline
\textit{Wymagania:}      & \textit{}                                                   \\ \hline
\end{tabular}
\end{table}

\begin{table}[H]
\begin{tabular}{|l|m{11cm}|}
\hline
\multicolumn{2}{|l|}{\textit{\textbf{Karta udziałowca}}}                               \\ \hline
\textit{Identyfikator:}  & \textit{UOB 02}                                             \\ \hline
\textit{Nazwa:}          & \textit{Klienci}                                              \\ \hline
\textit{Opis:}           & \textit{Grupa która dzięki systemowi ECMarket kupuje i sprzedaje dobra.} \\ \hline
\textit{Typ udziałowca:} & \textit{Ożywiony bezpośredni}                               \\ \hline
\textit{Punkt widzenia:} & \textit{Operator systemu}                                       \\ \hline
\textit{Ograniczenia:}   & \textit{Klienci nie mogą narzucać wymagań systemowych.}      \\ \hline
\textit{Wymagania:}      & \textit{}                                                   \\ \hline
\end{tabular}
\end{table}

\begin{table}[H]
\begin{tabular}{|l|m{11cm}|}
\hline
\multicolumn{2}{|l|}{\textit{\textbf{Karta udziałowca}}}                               \\ \hline
\textit{Identyfikator:}  & \textit{UNP 02}                                             \\ \hline
\textit{Nazwa:}          & \textit{Regulacje prawne}                                              \\ \hline
\textit{Opis:}           & \textit{Prawo ma duży wpływ na możliwości naszego systemu.} \\ \hline
\textit{Typ udziałowca:} & \textit{Nieożywiony pośredni}                               \\ \hline
\textit{Punkt widzenia:} & \textit{prawny}                                       \\ \hline
\textit{Ograniczenia:}   & \textit{}      \\ \hline
\textit{Wymagania:}      & \textit{}                                                   \\ \hline
\end{tabular}
\end{table}


\subsection{Ograniczenia}
Głównym ograniczeniem jest czas na wykonanie projektu, który wynosi niepełny rok.
Każda instrukcja wykonana na Wirtualnej Maszynie Ethereum (EVM), kosztuje pewną ilość etheru, co stanowi ograniczenie jak bardzo skomplikowany kod może być na niej wykonywany. Ten problem można częściowo rozwiązać delegując wykonanie niektórych zadań do klientów, takich, które nie wymagają wiarygodności lub takie, dla których można skonstruować dowód poprawności obliczeń, którego weryfikacja jest tańsza niż sam kod.
Pewnym ograniczeniem jest fakt, że nie opieramy się na żadnej stronie trzeciej, która byłaby w stanie stanowić autorytet i potwierdzać, że jakieś wydarzenie miało miejsce lub nie miało (np.\ dostarczenie produktu, który jednocześnie spełniałby pewne wymagania klienta lub innych stron). Te gwarancje próbujemy zbudować na podłożu teorii gier oraz społecznej teorii gier, by doprowadzić do sytuacji, gdzie strony nie wywiązujące się z umów tracą pieniądze lub/i ocenę, więc efektywnie zniechęcane do takich działań.
Istotnym problemem, który chcemy przynajmniej od Siebie odsunąć są kwestie prawne, a więc rosnąca kontrola władzy państwowej nad różnymi aspektami życia. Chcielibyśmy rozwiązać ten problem po przez wydanie platformy na licencji z rodziny GPL, jak również zbudować w ramach naszej platformy system repozytoriów zarządzanych przez klientów. Jednak realizacja tego drugiego odbędzie się nie wcześniej, niż w drugim wydaniu aplikacji.



\subsection{Zagrożenia}

\subsection{Harmonogram Prac}

\textbf{Przyrost Pierwszy}
\renewcommand{\labelenumii}{\textendash}
\begin{enumerate}
	\item analiza potencjalnych umów do zrealizowania w naszym systemie:
	\begin{enumerate}
		\item analiza umowy A1.1
		\item analiza umowy A1.2
		\item analiza umowy A1.3
		\item analiza umowy A2.1
		\item analiza umowy A2.2
		\item analiza umowy A3.1
	\end{enumerate}
	\item pierwsza analiza systemu ocen --- szkic własności
	\item propozycja pierwszej architektury systemu
	\item stworzenie wstępnego harmonogramu 
	\item rozważanie strategii
	\item dokument założeń wstępnych
	\item pierwsza specyfikacja wymagań systemowych
	\item wstępny diagram przypadków użycia
	\item nauka obsługi ekosystemu Ethereum
\end{enumerate}

%\newline aby oddzielic przyrost 1 i 2
\textbf{Przyrost Drugi}
\renewcommand{\labelenumii}{\textendash}
\begin{enumerate}
	\item Implementacja umowy A1.1
	\item Migracja tokenów systemowych do formatu zgodnego z ERC20
	\item Analiza metody zachęcania użytkowników do usuwania zakończonych / przeterminowanych umów
	\item Implementacja serwisu do wyszukiwania umów:
	\item Implementacja widoków dla systemu:
	\begin{enumerate}	
		\item Prototyp widoku do wyszukiwania umów
		\item Szczegółowy widoku umów
		\item Widok do przeglądania akcji danego adresu portfela
		\item Widok do tworzenia umów
	\end{enumerate}

	\item Rozbudowany routing komponentów
	\item Interakcje z umowami
	\item Pełna implementacja interfejsu użytkownika do wyszukiwania umów
\end{enumerate}
%\newline aby oddzielic przyrost 2 i 3

\subsection{Przebieg Etapów Pracy}

\subsection{Charakterystyka Zespołu}

\subsection{Infrastruktura Komunikacyjna i Dokumentacyjna}
W ramach infrastruktury komunikacyjnej wykorzysztaliśmy platformę https://www.discord.com, za pomocą której występowała komunikacja między członkami zespołami, ponieważ większość pracy została wykonywana zdalnie i był to nasz główny kanał komunikacji.  Do integracji kodu projektu oraz dokumentacji pomiędzy członkami grupy projektowej wykorzystaliśmy system kontroli wersji https://www.github.com.

\subsection{Aspekty Społeczne}


\newpage
\section{Analiza}

\subsection{Wymagania Systemowe}
\subsubsection{Wymagania ogólne i dziedzinowe}

\begin{table}[H]
\begin{tabular}{|l|l|l|l|}
\hline
\multicolumn{4}{|l|}{\textit{\textbf{Karta Wymagania}}}                                                           \\ \hline
\textit{Identyfikator:}        & \textit{WO1}            & \textit{\textbf{Priorytet:}}            & M            \\ \hline
\textit{Nazwa:}                & \multicolumn{3}{l|}{\textit{System ECMarket}}                                    \\ \hline
\textit{Opis:}                 & \multicolumn{3}{m{11cm}|}{\textit{Napisanie kodu dla ECMarket i wypuszczenie go do mainnetu}} \\ \hline
\textit{Wymagania powiązane :} & \multicolumn{3}{l|}{\textit{brak}}                                                   \\ \hline
\end{tabular}
\end{table}

\begin{table}[H]
\begin{tabular}{|l|l|l|l|}
\hline
\multicolumn{4}{|l|}{\textit{\textbf{Karta Wymagania}}}                                                           \\ \hline
\textit{Identyfikator:}        & \textit{WO2}            & \textit{\textbf{Priorytet:}}            & S            \\ \hline
\textit{Nazwa:}                & \multicolumn{3}{l|}{\textit{Podstawowa aplikacja kliencka}}                                    \\ \hline
\textit{Opis:}                 & \multicolumn{3}{m{11cm}|}{\textit{Przygotowanie podstawowej aplikacji klienckiej zapewniający graficzny interfejs do ECMarket}} \\ \hline
\textit{Wymagania powiązane :} & \multicolumn{3}{l|}{\textit{brak}}                                                   \\ \hline
\end{tabular}
\end{table}

\subsubsection{Wymagania funkcjonalne}

\begin{table}[H]
    \begin{tabular}{|l|l|l|l|}
    \hline
    \multicolumn{4}{|l|}{\textit{\textbf{Karta Wymagania}}}                                                           \\ \hline
    \textit{Identyfikator:}        & \textit{F01}            & \textit{\textbf{Priorytet:}}            & M            \\ \hline
    \textit{Nazwa:}                & \multicolumn{3}{l|}{\textit{Dostęp do podstawowych pól umowy.}}                                    \\ \hline
    \textit{Opis:}                 & \multicolumn{3}{m{11cm}|}{\textit{Każda umowa stworzona w systemie powinna dawać każdemu dostęp do takich podstawowych pól jak: lista uczestników umowy, numer bloku utworzenia, timestamp stworzenia, status umowy}} \\ \hline
    \textit{Kryteria akceptacji :} & \multicolumn{3}{l|}{\textit{Umowy posiadają metody zwracające żądane pola}}     \\ \hline
    \textit{Dane wejściowe :} & \multicolumn{3}{l|}{\textit{brak}}                                                \\ \hline
    \textit{Warunki początkowe :} & \multicolumn{3}{l|}{\textit{Umowa musi istnieć}}                                                \\ \hline
    \textit{Warunki końcowe :} & \multicolumn{3}{l|}{\textit{strona wywołująca otrzymuje żądane pole}}                                                \\ \hline
\end{tabular}
\end{table}

\begin{table}[H]
    \begin{tabular}{|l|l|l|l|}
    \hline
    \multicolumn{4}{|l|}{\textit{\textbf{Karta Wymagania}}}                                                           \\ \hline
    \textit{Identyfikator:}        & \textit{F02}            & \textit{\textbf{Priorytet:}}            & M            \\ \hline
    \textit{Nazwa:}                & \multicolumn{3}{l|}{\textit{Możliwość wpłacania, wypłacania i sprawdzenia bilansu etheru}}                                    \\ \hline
    \textit{Opis:}                 & \multicolumn{3}{m{11cm}|}{\textit{Każdy użytkownik sieci ma możliwość wpłacenia dowolnej ilość Etheru do “Virtual Wallet”-u oraz wpłacenia dowolnej ilości etheru, nie większej niż bilans danego użytkownika . Zmiana bilansu danego użytkownika, pokrywa się jeden-do-jednego z ilością wpłaconego lub wypłaconego przez niego etheru. Jeśli użytkownik nie brał udziału w umowie, ani nie wpłacał etheru jego bilans wynosi 0.  }} \\ \hline
    \textit{Kryteria akceptacji :} & \multicolumn{3}{m{11cm}|}{\textit{Każdy użytkownik może niezależnie od innych wpłacać lub wypłacać ether. Jeśli użytkownik wpłacił pewne x i nie brał udziału w żadnej umowie, może maksymalnie wypłacić x.}}    \\ \hline
    \textit{Dane wejściowe :} & \multicolumn{3}{l|}{\textit{brak}}                                                \\ \hline
    \textit{Warunki początkowe :} & \multicolumn{3}{l|}{\textit{posiadanie etheru}}                                                \\ \hline
    \textit{Warunki końcowe :} & \multicolumn{3}{l|}{\textit{zmiana bilansu}}                                                \\ \hline
    \textit{Sytuacje wyjątkowe :} & \multicolumn{3}{m{11cm}|}{\textit{Jeśli użytkownik żąda wypłaty większej ilości etheru niż posiada go w VirtualWallet, zostaje wyrzucony wyjątek.}}                                                \\ \hline
\end{tabular}
\end{table}

\begin{table}[H]
    \begin{tabular}{|l|l|l|l|}
    \hline
    \multicolumn{4}{|l|}{\textit{\textbf{Karta Wymagania}}}                                                           \\ \hline
    \textit{Identyfikator:}        & \textit{F03}            & \textit{\textbf{Priorytet:}}            & M            \\ \hline
    \textit{Nazwa:}                & \multicolumn{3}{l|}{\textit{Możliwość stworzenia umowy}}                                    \\ \hline
    \textit{Opis:}                 & \multicolumn{3}{m{11cm}|}{\textit{AgreementManager pozwala na stworzenie umowy z parametrami i typem określonym przez użytkownika. Użytkownik otrzymuje adres stworzonej umowy jako wartość zwrotną z funkcji lub jako wydarzenie o nazwie AgreementCreation z dodatkowym polem zawierającym adres utworzonej umowy. Poprawnie utworzona umowa zostaje zarejestrowana przez AgreementManagera.}} \\ \hline
    \textit{Kryteria akceptacji :} & \multicolumn{3}{m{11cm}|}{\textit{AgreementManager przy poprawnych danych tworzy poprawne umowy. Wydarzenie powinno być wygenerowane, tylko wtedy, gdy tworzenie umowy się powiodło.}}    \\ \hline
    \textit{Dane wejściowe :} & \multicolumn{3}{l|}{\textit{brak}}                                                \\ \hline
    \textit{Warunki początkowe :} & \multicolumn{3}{l|}{\textit{brak}}                                                \\ \hline
    \textit{Warunki końcowe :} & \multicolumn{3}{m{11cm}|}{\textit{utworzenie nowej umowy i możliwość zarządzania nią z pomocą innych funkcji AgreementManagera}}                                                \\ \hline
\end{tabular}
\end{table}

\begin{table}[H]
    \begin{tabular}{|l|l|l|l|}
    \hline
    \multicolumn{4}{|l|}{\textit{\textbf{Karta Wymagania}}}                                                           \\ \hline
    \textit{Identyfikator:}        & \textit{F04}            & \textit{\textbf{Priorytet:}}            & M            \\ \hline
    \textit{Nazwa:}                & \multicolumn{3}{l|}{\textit{Możliwość wyszukania umowy}}                                    \\ \hline
    \textit{Opis:}                 & \multicolumn{3}{m{11cm}|}{\textit{AgreementManager powinien mieć możliwość zwrócenia strony z ostatnio stworzonymi umowami.}} \\ \hline
    \textit{Kryteria akceptacji :} & \multicolumn{3}{m{11cm}|}{\textit{AgreementManager przy poprawnych danych tworzy poprawne umowy. Wydarzenie powinno być wygenerowane, tylko wtedy, gdy tworzenie umowy się powiodło.}}    \\ \hline
    \textit{Dane wejściowe :} & \multicolumn{3}{l|}{\textit{brak}}                                                \\ \hline
    \textit{Warunki początkowe :} & \multicolumn{3}{l|}{\textit{AgreementManager musi posiadać zarejestrowane umowy}}                                                \\ \hline
    \textit{Warunki końcowe :} & \multicolumn{3}{m{11cm}|}{\textit{utworzenie nowej umowy i możliwość zarządzania nią z pomocą innych funkcji AgreementManagera}}                                                \\ \hline
    \textit{Sytuacje wyjątkowe :} & \multicolumn{3}{m{11cm}|}{\textit{W przypadku braku jakichkolwiek umów zostaje zwrócona tablica z samymi zerami}} \\ \hline
\end{tabular}
\end{table}

\begin{table}[H]
    \begin{tabular}{|l|l|l|l|}
    \hline
    \multicolumn{4}{|l|}{\textit{\textbf{Karta Wymagania}}}                                                           \\ \hline
    \textit{Identyfikator:}        & \textit{F05}            & \textit{\textbf{Priorytet:}}            & M            \\ \hline
    \textit{Nazwa:}                & \multicolumn{3}{l|}{\textit{Możliwość usunięcia nowo-stworzonej umowy A1.1.}}                                    \\ \hline
    \textit{Opis:}                 & \multicolumn{3}{m{11cm}|}{\textit{Umowy stworzone przez użytkowników mogą być przez nich usuwane}} \\ \hline
    \textit{Kryteria akceptacji :} & \multicolumn{3}{m{11cm}|}{\textit{Tylko wskazana umowa ulega usunięciu}}    \\ \hline
    \textit{Dane wejściowe :} & \multicolumn{3}{l|}{\textit{brak}}                                                \\ \hline
    \textit{Warunki początkowe :} & \multicolumn{3}{m{11cm}|}{\textit{Umowa jest w stanie New, ważna (nie przekroczyła daty ważności), a stroną wywołująca jest jest twórca}}                                                \\ \hline
    \textit{Warunki końcowe :} & \multicolumn{3}{m{11cm}|}{\textit{Kontrakt umowy zostaje zniszczony, wyrejestrowany z AgreementManagera, a środki wpłacone przez petentów zostają do nich odesłane}}                                                \\ \hline
    \textit{Sytuacje wyjątkowe :} & \multicolumn{3}{m{11cm}|}{\textit{brak}} \\ \hline
\end{tabular}
\end{table}

\begin{table}[H]
    \begin{tabular}{|l|l|l|l|}
    \hline
    \multicolumn{4}{|l|}{\textit{\textbf{Karta Wymagania}}}                                                           \\ \hline
    \textit{Identyfikator:}        & \textit{F06}            & \textit{\textbf{Priorytet:}}            & M            \\ \hline
    \textit{Nazwa:}                & \multicolumn{3}{l|}{\textit{Możliwość usunięcia przeterminowanej umowy A1.1.}}                                    \\ \hline
    \textit{Opis:}                 & \multicolumn{3}{m{11cm}|}{\textit{Przeterminowane umowy mogą być usunięte przez każdego.}} \\ \hline
    \textit{Kryteria akceptacji :} & \multicolumn{3}{m{11cm}|}{\textit{Tylko wskazana umowa ulega usunięciu}}    \\ \hline
    \textit{Dane wejściowe :} & \multicolumn{3}{l|}{\textit{brak}}                                                \\ \hline
    \textit{Warunki początkowe :} & \multicolumn{3}{l|}{\textit{umowa przekroczyła swoją datę ważności}}                                                \\ \hline
    \textit{Warunki końcowe :} & \multicolumn{3}{m{11cm}|}{\textit{Kontrakt umowy zostaje zniszczony, wyrejestrowany z AgreementManagera, a środki wpłacone przez petentów zostają do nich odesłane}}                                                \\ \hline
    \textit{Sytuacje wyjątkowe :} & \multicolumn{3}{m{11cm}|}{\textit{brak}} \\ \hline
\end{tabular}
\end{table}

\begin{table}[H]
    \begin{tabular}{|l|l|l|l|}
    \hline
    \multicolumn{4}{|l|}{\textit{\textbf{Karta Wymagania}}}                                                           \\ \hline
    \textit{Identyfikator:}        & \textit{F07}            & \textit{\textbf{Priorytet:}}            & M            \\ \hline
    \textit{Nazwa:}                & \multicolumn{3}{l|}{\textit{Możliwość usunięcia zakończonej umowy A1.1.}}                                    \\ \hline
    \textit{Opis:}                 & \multicolumn{3}{m{11cm}|}{\textit{Umowy zakończone mogą być usuwane przez użytkowników}} \\ \hline
    \textit{Kryteria akceptacji :} & \multicolumn{3}{m{11cm}|}{\textit{Tylko wskazana umowa ulega usunięciu}}    \\ \hline
    \textit{Dane wejściowe :} & \multicolumn{3}{l|}{\textit{brak}}                                                \\ \hline
    \textit{Warunki początkowe :} & \multicolumn{3}{m{11cm}|}{\textit{Umowa jest w stanie Done, ważna (nie przekroczyła daty ważności), a stroną wywołująca jest jest twórca}}                                                \\ \hline
    \textit{Warunki końcowe :} & \multicolumn{3}{m{11cm}|}{\textit{Kontrakt umowy zostaje zniszczony, wyrejestrowany z AgreementManagera, a środki wpłacone przez petentów zostają do nich odesłane}}                                                \\ \hline
    \textit{Sytuacje wyjątkowe :} & \multicolumn{3}{m{11cm}|}{\textit{brak}} \\ \hline
\end{tabular}
\end{table}

\begin{table}[H]
    \begin{tabular}{|l|l|l|l|}
    \hline
    \multicolumn{4}{|l|}{\textit{\textbf{Karta Wymagania}}}                                                           \\ \hline
    \textit{Identyfikator:}        & \textit{F08}            & \textit{\textbf{Priorytet:}}            & M            \\ \hline
    \textit{Nazwa:}                & \multicolumn{3}{l|}{\textit{Możliwość oceniania stron kontraktu}}                                    \\ \hline
    \textit{Opis:}                 & \multicolumn{3}{m{11cm}|}{\textit{Obie strony mają możliwość ocenienia drugiej strony przed zakończeniem kontraktu}} \\ \hline
    \textit{Kryteria akceptacji :} & \multicolumn{3}{m{11cm}|}{\textit{Warunki Satysfakcji (Szczegóły dodane na potrzeby  testów akceptacyjnych)}}    \\ \hline
    \textit{Dane wejściowe :} & \multicolumn{3}{l|}{\textit{Kontrakt został zapoczątkowany}}                                                \\ \hline
    \textit{Warunki początkowe :} & \multicolumn{3}{m{11cm}|}{\textit{Umowa jest w stanie Done, ważna (nie przekroczyła daty ważności), a stroną wywołująca jest jest twórca}}                                                \\ \hline
    \textit{Warunki końcowe :} & \multicolumn{3}{m{11cm}|}{\textit{Kontrakt umowy zostaje zniszczony, wyrejestrowany z AgreementManagera, a środki wpłacone przez petentów zostają do nich odesłane}}                                                \\ \hline
    \textit{Sytuacje wyjątkowe :} & \multicolumn{3}{m{11cm}|}{\textit{brak}} \\ \hline
\end{tabular}
\end{table}

\begin{table}[H]
    \begin{tabular}{|l|l|l|l|}
    \hline
    \multicolumn{4}{|l|}{\textit{\textbf{Karta Wymagania}}}                                                           \\ \hline
    \textit{Identyfikator:}        & \textit{F09}            & \textit{\textbf{Priorytet:}}            & M            \\ \hline
    \textit{Nazwa:}                & \multicolumn{3}{l|}{\textit{Wykonanie umowy A1.1}}                                    \\ \hline
    \textit{Opis:}                 & \multicolumn{3}{m{11cm}|}{\textit{Umowa A1.1 posiada funkcje join, accept i conclude. Strony chętne do uczestnictwa w umowie wykonują join jednocześnie przekazując środki określone w polu cena na konto umowy. Właściciel umowy może wykonać accept wskazując z kim chce przeprowadzić transakcję, jednocześnie pobierając środki z konta umowy. Po tym fakcie obie strony mogą się ocenić z pomocą conclude}} \\ \hline
    \textit{Kryteria akceptacji :} & \multicolumn{3}{m{11cm}|}{\textit{AgreementManager zarejestrował wykonanie umowy}}    \\ \hline
    \textit{Dane wejściowe :} & \multicolumn{3}{l|}{\textit{Kontrakt został zapoczątkowany}}                                                \\ \hline
    \textit{Warunki początkowe :} & \multicolumn{3}{l|}{\textit{brak}}                                                \\ \hline
    \textit{Warunki końcowe :} & \multicolumn{3}{m{11cm}|}{\textit{brak}}                                                \\ \hline
    \textit{Sytuacje wyjątkowe :} & \multicolumn{3}{m{11cm}|}{\textit{brak}} \\ \hline
\end{tabular}
\end{table}

\begin{table}[H]
    \begin{tabular}{|l|l|l|l|}
    \hline
    \multicolumn{4}{|l|}{\textit{\textbf{Karta Wymagania}}}                                                           \\ \hline
    \textit{Identyfikator:}        & \textit{F10}            & \textit{\textbf{Priorytet:}}            & M            \\ \hline
    \textit{Nazwa:}                & \multicolumn{3}{l|}{\textit{Możliwość wypłacania środków z kontraktu}}                                    \\ \hline
    \textit{Opis:}                 & \multicolumn{3}{m{11cm}|}{\textit{Petent powinien mieć możliwość wypłacenia środków.}} \\ \hline
    \textit{Kryteria akceptacji :} & \multicolumn{3}{m{11cm}|}{\textit{brak}}    \\ \hline
    \textit{Dane wejściowe :} & \multicolumn{3}{l|}{\textit{brak}}                                                \\ \hline
    \textit{Warunki początkowe :} & \multicolumn{3}{l|}{\textit{Wpłacenie środków do kontraktu.}}                                                \\ \hline
    \textit{Warunki końcowe :} & \multicolumn{3}{m{11cm}|}{\textit{brak}}                                                \\ \hline
    \textit{Sytuacje wyjątkowe :} & \multicolumn{3}{m{11cm}|}{\textit{brak}} \\ \hline
\end{tabular}
\end{table}

\begin{table}[H]
    \begin{tabular}{|l|l|l|l|}
    \hline
    \multicolumn{4}{|l|}{\textit{\textbf{Karta Wymagania}}}                                                           \\ \hline
    \textit{Identyfikator:}        & \textit{F11}            & \textit{\textbf{Priorytet:}}            & M            \\ \hline
    \textit{Nazwa:}                & \multicolumn{3}{l|}{\textit{Ustawianie czasu ważności umowy przy tworzeniu}}                                    \\ \hline
    \textit{Opis:}                 & \multicolumn{3}{m{11cm}|}{\textit{Użytkownik ma możliwość określenia ile czasu/bloków dana umowa będzie ważna. Przedział dozwolonych czasów ważności jest ograniczony od dołu i od góry i zależny od typu umowy. Po utworzeniu umowy nie jest możliwa modyfikacja tej wartości}} \\ \hline
    \textit{Kryteria akceptacji :} & \multicolumn{3}{m{11cm}|}{\textit{brak}}    \\ \hline
    \textit{Dane wejściowe :} & \multicolumn{3}{l|}{\textit{brak}}                                                \\ \hline
    \textit{Warunki początkowe :} & \multicolumn{3}{l|}{\textit{brak}}                                                \\ \hline
    \textit{Warunki końcowe :} & \multicolumn{3}{m{11cm}|}{\textit{brak}}                                                \\ \hline
    \textit{Sytuacje wyjątkowe :} & \multicolumn{3}{m{11cm}|}{\textit{brak}} \\ \hline
\end{tabular}
\end{table}

\subsubsection{Wymagania niefunkcjonalne}

\begin{table}[H]
    \begin{tabular}{|l|l|l|l|}
    \hline
    \multicolumn{4}{|l|}{\textit{\textbf{Karta Wymagania}}}                                                           \\ \hline
    \textit{Identyfikator:}        & \textit{NF01}            & \textit{\textbf{Priorytet:}}            & S            \\ \hline
    \textit{Nazwa:}                & \multicolumn{3}{l|}{\textit{System zachęty do usuwania przeterminowanych umów}}                                    \\ \hline
    \textit{Opis:}                 & \multicolumn{3}{m{11cm}|}{\textit{Stworzenie umowy wymaga wpłacenia niedużej kaucji (trzeba pomyśleć jak “niedużej”), która jest wypłacana temu, który usunie umowę z systemu. Jeśli umowa jest ważna, tylko osoba uprawniona (zazwyczaj twórca umowy) może ją usunąć i otrzymać z powrotem kaucję. Jeśli umowa jest przeterminowana osoba uprawniona ma wyłączność na usunięcie i zagarnięcie kaucji, ale tylko przez pewien okres ochronny (jak długi) po dacie utraty ważności. Po okresie ochronnym każdy może usunąć umowę i otrzymać daną kaucję}} \\ \hline
    \textit{Kryteria akceptacji :} & \multicolumn{3}{m{11cm}|}{\textit{brak}}    \\ \hline

\end{tabular}
\end{table}

\begin{table}[H]
    \begin{tabular}{|l|l|l|l|}
    \hline
    \multicolumn{4}{|l|}{\textit{\textbf{Karta Wymagania}}}                                                           \\ \hline
    \textit{Identyfikator:}        & \textit{NF02}            & \textit{\textbf{Priorytet:}}            & M            \\ \hline
    \textit{Nazwa:}                & \multicolumn{3}{m{11cm}|}{\textit{System ocen - efekty wystawienia pozytywnej / neutralnej / negatywnej oceny}}                                    \\ \hline
    \textit{Opis:}                 & \multicolumn{3}{m{11cm}|}{\textit{Jeśli strona P1 umowy A wystawia pewną pozytywną ocenę r1 stronie P2, wtedy wiarygodność P2 w2 wzrasta do w2’ według pewnej miary D, tak że D(w2) < D(w2’).
    Jeśli strona P1 umowy A wystawia pewną neutralną ocenę O stronie P2, wtedy wiarygodność P2 w2 nie ulega zmianie.
    Jeśli strona P1 umowy A wystawia pewną negatywną ocenę r1 stronie P2, wtedy wiarygodność P2 w2 spada do w2’ według pewnej miary D, tak że D(w2) > D(w2’).}} \\ \hline
    \textit{Kryteria akceptacji :} & \multicolumn{3}{m{11cm}|}{\textit{brak}}    \\ \hline

\end{tabular}
\end{table}

\begin{table}[H]
    \begin{tabular}{|l|l|l|l|}
    \hline
    \multicolumn{4}{|l|}{\textit{\textbf{Karta Wymagania}}}                                                           \\ \hline
    \textit{Identyfikator:}        & \textit{NF03}            & \textit{\textbf{Priorytet:}}            & M            \\ \hline
    \textit{Nazwa:}                & \multicolumn{3}{l|}{\textit{System ocen - tłumienie}}                                    \\ \hline
    \textit{Opis:}                 & \multicolumn{3}{m{11cm}|}{\textit{Strony P1 i P2 biorą udział w zbiorze umów {A1, A2, …, An}. P1 w kolejnych umowach A1, A2, ..., An wystawia P2 oceny r1, r2, …,rn, co daje ciąg wiarygodności dla P2 w1, w2, …, wn. Wtedy każdy z tych elementów tego ciągu spełnia zależność dR kNi>k Dwi+1-Dwi<i-d.
    To oznacza, że od pewnego momentu (jakiego) waga ocen wystawianych przez P1 dla P2 zaczyna co najmniej wykładniczo spadać.}} \\ \hline
    \textit{Kryteria akceptacji :} & \multicolumn{3}{m{11cm}|}{\textit{brak}}    \\ \hline

\end{tabular}
\end{table}

\begin{table}[H]
    \begin{tabular}{|l|l|l|l|}
    \hline
    \multicolumn{4}{|l|}{\textit{\textbf{Karta Wymagania}}}                                                           \\ \hline
    \textit{Identyfikator:}        & \textit{NF04}            & \textit{\textbf{Priorytet:}}            & M            \\ \hline
    \textit{Nazwa:}                & \multicolumn{3}{l|}{\textit{System ocen - istotność wiarygodności partnera}}                                    \\ \hline
    \textit{Opis:}                 & \multicolumn{3}{m{11cm}|}{\textit{Strony P1, P2, P3, P4 mają wiarygodności w1, w2, w3, w4 gdzie D(w1) < D(w2) i D(w3) = D(w4). Wtedy, jeśli P1 i P3 biorą udział w umowie tego samego typu co P2 i P4 i P1 wystawi P3 ocenę r1, a P2 wystawi P4 r2, takie że r1 = r2, to D(w3)-D(w3') < D(w4)-D(w4').
    Z tego wynika, że oceny wystawiane przez stronę o większej wiarygodności, mają większą wagę niż oceny strony o niższej
    }} \\ \hline
    \textit{Kryteria akceptacji :} & \multicolumn{3}{m{11cm}|}{\textit{brak}}    \\ \hline

\end{tabular}
\end{table}


\subsubsection{Wymagania projektowo-wdrożeniowe}

\begin{table}[H]
    \begin{tabular}{|l|l|l|l|}
    \hline
    \multicolumn{4}{|l|}{\textit{\textbf{Karta Wymagania}}}                                                           \\ \hline
    \textit{Identyfikator:}        & \textit{ŚD01}            & \textit{\textbf{Priorytet:}}            & M            \\ \hline
    \textit{Nazwa:}                & \multicolumn{3}{l|}{\textit{Ekosystem Ethereum}}                                    \\ \hline
    \textit{Opis:}                 & \multicolumn{3}{m{11cm}|}{\textit{Wszystkie produkty wygenerowane w czasie trwania projektu opierają się na technologiach pochodzących z ekosystemu Ethereum, takich jak sieć Ethereum, sieć Swarm, sieć Whisper, biblioteki Web3.}} \\ \hline
    \textit{Kryteria akceptacji :} & \multicolumn{3}{m{11cm}|}{\textit{brak}}    \\ \hline

\end{tabular}
\end{table}

\subsubsection{Wymagania dotyczące procesów wytwarzania}

\begin{table}[H]
    \begin{tabular}{|l|l|l|l|}
    \hline
    \multicolumn{4}{|l|}{\textit{\textbf{Karta Wymagania}}}                                                           \\ \hline
    \textit{Identyfikator:}        & \textit{PW01}            & \textit{\textbf{Status:}}            & M            \\ \hline
    \textit{Nazwa:}                & \multicolumn{3}{l|}{\textit{TDD - Test Driven Development}}                                    \\ \hline
    \textit{Opis:}                 & \multicolumn{3}{m{11cm}|}{\textit{Każda zmiana lub nowa funkcjonalność wprowadzana do systemu, musi być najpierw zamodelowana z pomocą testów. Dopiero po tym testy specyfikują jak docelowy kod ma wyglądać.}} \\ \hline
    \textit{Kryteria akceptacji :} & \multicolumn{3}{m{11cm}|}{\textit{brak}}    \\ \hline
\end{tabular}
\end{table}

\begin{table}[H]
    \begin{tabular}{|l|l|l|l|}
    \hline
    \multicolumn{4}{|l|}{\textit{\textbf{Karta Wymagania}}}                                                           \\ \hline
    \textit{Identyfikator:}        & \textit{PW02}            & \textit{\textbf{Status:}}            & M            \\ \hline
    \textit{Nazwa:}                & \multicolumn{3}{l|}{\textit{Określony git-flow}}                                    \\ \hline
    \textit{Opis:}                 & \multicolumn{3}{m{11cm}|}{\textit{Każda nowa funkcjonalność lub ich zbiór powinien być opisany w Issue. Branche z implementacją danego Issue powinny mieć postać ‘feature/IS\%numer\_issue\%/\%dodatkowy\_opis\%’. Zmianny do testów i do kodu powinny być zawarte w osobnych commitach. Wiadomość w commitach powinny mieć postać ‘IS\%numer\_issue\% test/code - \%opis\_commitu\%’.}} \\ \hline
    \textit{Kryteria akceptacji :} & \multicolumn{3}{m{11cm}|}{\textit{brak}}    \\ \hline
\end{tabular}
\end{table}

\subsection{Specyfikacja Przypadków Uzycia}
\subsubsection{Diagram przypadków uzycia}

\includegraphics[width=1.0\textwidth]{Use_Case_Diagram}

\subsubsection{Diagram sekwencji}

\newpage
\subsubsection{Maszyny Stanowe}

Umowa A1.1\\
\includegraphics[width=1.0\textwidth]{SM-A1_1}


\subsection{Model Obiektowy Systemu}

\subsubsection{Back-End}

\includegraphics[width=1.0\textwidth]{Contract_Class_Diagram_Class_Diagram}

\subsubsection{Front-End}

\subsubsection{Sieć procesów}

\subsection{Analiza Komercyjna}

\newpage
\section{Projektowanie}
\newpage
\section{Architektura}

\newpage
\section{Przyrost Pierwszy}
\subsection{Cele przyrostu}
Podczas pierwszego przyrostu postawiliśmy sobie za cel przeanalizowanie, jak powinny wyglądać sprawiedliwe kontrakty, tak, aby każda ze stron nie czuła się w żaden sposób oszukana. Naszym kolejnym celem było zaproponowanie, jak system powinien wyglądać, jakich akcji może podejmować się użytkownik, jakie powinien mieć ograniczenia w danych przypadkach, oraz ustalić wstępną architekturę systemu. Oprócz tego, wykonaliśmy wstępny harmonogram prac do monitorowania postępów.

\subsection{Analiza back-end'u}
Na samym początku zabraliśmy się za analizę naszego systemu. Postanowiliśmy że naszym głównym przedmiotem badań będzie przeanalizowanie sprawiedliwych kontraktów. Podczas analizowania różnych przypadków postanowiliśmy że system powinien dostarczać różne rodzaje kontraktów w zależności od potrzeb użytkowników którzy będą korzystać z naszego systemu. Podczas prac analizowaliśmy różnego przypadki umów, na koniec postanowiliśmy że w ramach naszej pracy inżynierskiej stworzymy dwa typy kontraktów.

\begin{figure}[h!]
\centering
\includegraphics[width=0.35\textwidth]{A1_1.png}
\caption{Analiza umowy 1.1}
\label{A1_1}
\end{figure}

Umowa przedstawiona powyżej jest najprostszą dostępną umową w naszym systemie. Umowę rozpoczyna Sprzedawca, który tworzy umowę (punkt 0 na rysunku \ref{A1_1}) . Potencjalny kupujący zainteresowany umową wpłaca pieniądze(1), a w następnym kroku Sprzedawca potwierdza transakcję(2). Na koniec kupujący ocenia sprzedającego(3). W umowie istnieją alternatywne scenariusze, ponieważ sprzedający może nie chcieć zawrzeć umowy z nisko ocenianym kupującym i odrzucić ofertę(2a), gdzie następnym krokiem jest wypłacenie tokenów przez kupującego(3a). Jeżeli dojdzie do porozumienia lub też nie, umowa zostaje zakończona(4a).

\begin{figure}[h!]
\centering
\includegraphics[width=0.45\textwidth]{A1_2.png}
\caption{Analiza umowy 1.2}
\label{A1_2}
\end{figure}

Umowa A1.2 jest ewolucją umowy A1.1. W tym przypadku sprzedający dodatkowo ustala zaliczkę oraz czas pozwalający na wycofanie się kupującego. Kolejną ważną cechą jest też możliwość ocenienia sprzedającego przez kupującego. W powyższej umowie dochodzi jedna dodatkowa alternatywna droga do zakończenia umowy. Kupujący może zrezygnować z kontynuowania procesu i może zażądać zwrotu wpłaconych pieniędzy, jednak może to zrobić tylko przed terminem pozwalającym na wycofanie się.

\subsection{Projektowanie back-end'u}
\subsection{Testy walidacyjne}
\subsection{Implementacja}
\subsection{Lista zmian}

\newpage
\section{Przyrost Drugi}
\subsection{Cele przyrostu}
\subsection{Analiza front-end'u}
\subsection{Projektowanie front-end'u}
\subsection{Testy walidacyjne}
\subsection{Testy integracyjne}
\subsection{Implementacja}
\subsection{Lista zmian}

\newpage
\section{Przyrost Trzeci}
\subsection{Cele przyrostu}
\subsection{Projektowanie back-end'u}
\subsection{Projektowanie front-end'u}
\subsection{Testy walidacyjne}
\subsection{Testy integracyjne}
\subsection{Testy systemowe}
\subsection{Implementacja}
\subsection{Lista zmian}

\newpage
\section{Instrukcja Obsługi}

\newpage
\section{Prezentacja Projektu}

\newpage
\section{Raport Końcowy}

\newpage
\section{Całkowity Nakład Pracy}

\newpage
\section{Wkład Własny w Projekt}

\newpage
\section{Podsumowanie}

\newpage
\section{Słownik Pojęć}
\begin{itemize}
	\item Platforma Handlowa --- aplikacje agregujące oferty handlowe, pośredniczące w wykonaniu opłat i dostarczające narzędzia od oceny kupujących i sprzedających. 
	\item Ekosystem Ethereum (zespół technologii Ethereum) --- Ethereum jako środowisko wykonawcze oparte na blockchainie, rozproszona przestrzeń dyskowa Swarm, system komunikatów Whisper, biblioteki web3 i aplikacje klienckie Ethereum.
	\item  Ethereum --- środowisko wykonawcze oparte na blockchainie dla inteligentnych kontraktów.
	\item  Blockchain --- struktura danych mogąca przechowywać dowolne dane w postaci bloków, skonstruowana na zasadzie listy (lub w pewnym stopniu drzewa), gdzie wskaźnikami do poprzednich bloków są ich kryptograficzne hasze.
	\item Drzewa Merkle --- struktura danych mogąca przechowywać dowolne dane, skonstruowana na zasadzie drzewa, gdzie wskaźnikami na rodziców są ich kryptograficzne hasze.
	\item Mainnet --- główna sieć Ethereum o id równym 1. Na niej odbywają się transakcje z użyciem etheru posiadającym wartość dla użytkowników.
	\item Swarm --- zdecentralizowana przestrzeń dyskowa oparta na drzewach Merkle, zintegrowana z Ethereum. 
	\item Whisper --- zdecentralizowany, wysoce anonimowy system komunikatów zintegrowany z Ethereum.
	\item Wirtualna Maszyna Ethereum --- część Ethereum wykonująca bytecode skompilowanych kontraktów.
	\item Ether --- jednostka płatności/możliwości wykonania kodu. Jest podstawową walutą w ekosystemie Ethereum. 1 ether dzieli się na $10$\textsuperscript{18} wei.
\end{itemize}

\newpage
\section{Załączniki}

\newpage
\bibliographystyle{IEEEtranS}
\bibliography{bibliografia}
\end{document}