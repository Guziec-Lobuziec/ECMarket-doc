\documentclass[11pt,oneside]{article}
\usepackage[utf8]{inputenc}
\usepackage[english,polish]{babel}
\usepackage[T1]{fontenc}
\usepackage[a4paper,width=170mm,top=18mm,bottom=22mm,includeheadfoot]{geometry}
\usepackage{graphicx}
\usepackage{comment}
\usepackage{multirow}
\usepackage{courier}
\usepackage{float}
\usepackage{array}
\usepackage{amsmath}
\graphicspath{ {./images/} }

\begin{document}

\begin{figure}[H]

\centering
\includegraphics[width=1.0\textwidth]{logopjwstk}

\begin{Huge}
\begin{center}
	\texttt{\textbf {KARTA PROJEKTU}}
\end{center}
\end{Huge}

\includegraphics[width=0.94\textwidth]{cheat1}
\includegraphics[width=0.94\textwidth]{cheat2}
\end{figure}

\begin{figure}[H]
\includegraphics[width=.4\textwidth]{cheat3}
\end{figure}

\setcounter{tocdepth}{2}
\tableofcontents

\newpage

\section{Podziękowania}

\section{Wstęp}
Na fali pojawiających się nowych zdecentralizowanych aplikacji,
protokołów komunikacj p2p, algorytmów rozproszonego konsensusu i ideii Internetu 3.0,
narodziła się w śród nas koncepcja systemu wsperającego i nadzorującego szeroko rozumianą wymianę obiektów,
w szczególności dóbr fizycznych, który wpasowywałby się do tego nowego wspaniałego świata.
Porwani przez tę wizję, poświęcamy tę pracą pierwszemu podstawowemu wydaniu systemu ECMarket,
ktore będzie wstępem do stworzenia analogu do serwisów aukcyjnych typu Allegro czy Ebay w uniwersum
aplikacji zdecentralizowanych.

\section{Opis Problemu}

\subsection{Prezentacja Problemu}
W związku z rosnącą centralizacją Internetu, problemem cenzury oraz spadkiem zaufania do
instytucji państwa \cite{trustPOL, trustUSA} jak i korporacji, powstało wiele projektów mających
na celu rozproszenie tej koncentracji władzy, jednak przez długi czas rozwiązania te nie zyskiwały
duzej popularności poza specjalistycznymi dziedzinami.
Odwilż w tej kwestii przyniósł protokół Kademila, na której oparte zostały sieci wymiany plików takie
jak BitTorrent, czy Kad Network. Ich funkcjonalność ograniczała się do wymiany plików, jednak nie była
pozbawiona wad takich jak niska żywotność niepopularnych plików, egoistyczne zachowania klientów sieci
(leeching), brak odporności na ataki DDoS.
Przełom nastąpił w 2009 w raz z prezentacją technologi Bitcoin, która oferowała roziązanie problemów z
żywotnością, z którymi borykały się.

\subsection{Rich Picture}

\subsection{Konkurencja}

\subsection{Proponowane Rozwiązanie}



\section{Planowanie}

\subsection{Cele i Zakres Projektu}

Celem projektu jest dostarczenie odpornej na cenzurę, ogólnodostępnej, otwartej platformy, w ramach której możliwe jest przeprowadzanie kupna oraz sprzedaży dóbr w sposób niezależny od osób trzecich lub w ramach grup handlowych, tworzonych przez użytkowników systemu z określanymi przez nich zasadami.
W pierwszym stadium systemu, realizowanym w ramach tego projektu, planujemy pobierać mały stały procent od kwoty przesyłanej w transakcjach między użytkownikami, by zapewnić utrzymanie i dalszy rozwój projektu.
Naszą platformę kierujemy do wszystkich pragnących niezależnego i otwartego systemu do handlu.


\subsection{Kontekst Projektu}

System będzie docelowo uruchomiony na Testnecie Ethereum i będzie dostarczał rozproszoną, odporną na cenzurę platformę do handlu.
Użytkownicy w pierwszym stadium rozwoju platformy będą się dzielić na Deweloperów i na Klientów. Zadaniem Deweloperów będzie dostarczanie kodu i jego deployment oraz ponoszenie kosztów tych operacji. Deweloperzy będą również beneficjentami oprocentowania transakcji. Klientami nazywamy wszystkich użytkowników, którzy dokonują transakcji.

\begin{comment}
, tworzą grupy handlowe lub pełnią jakiekolwiek inne funkcje w ramach grup handlowych
\end{comment}

Klienci mogą zarządzać swoimi portfelami, oraz mogą wchodzić w interakcje z dostępnymi dla nich umowami.

\begin{comment}
, grupami, które stworzyli (chyba że grupa została zdefiniowana inaczej)
\end{comment}

Liczba Deweloperów będzie ograniczona do kilku kilkunastu w zależności od potrzeb. Liczba Klientów nie jest w żaden sposób ograniczona.

\subsection{Harmonogram Prac}

\subsection{Analiza Biznesowa}

\subsection{Plan Zapewnienia Jakości}

\subsection{Udziałowcy}


\begin{table}[H]
\begin{tabular}{|l|m{11cm}|}
\hline
\multicolumn{2}{|l|}{\textit{\textbf{Karta udziałowca}}}                               \\ \hline
\textit{Identyfikator:}  & \textit{UNP 01}                                             \\ \hline
\textit{Nazwa:}          & \textit{PJATK}                                              \\ \hline
\textit{Opis:}           & \textit{PJATK umożliwiło nam wykonanie pracy inżynierskiej} \\ \hline
\textit{Typ udziałowca:} & \textit{Nieożywiony pośredni}                               \\ \hline
\textit{Punkt widzenia:} & \textit{Ekonomicznej}                                       \\ \hline
\textit{Ograniczenia:}   & \textit{PJATK nie może narzucać rozwiązań systemowych}      \\ \hline
\textit{Wymagania:}      & \textit{}                                                   \\ \hline
\end{tabular}
\end{table}

\begin{table}[H]
\begin{tabular}{|l|m{11cm}|}
\hline
\multicolumn{2}{|l|}{\textit{\textbf{Karta udziałowca}}}                               \\ \hline
\textit{Identyfikator:}  & \textit{UNB 02}                                             \\ \hline
\textit{Nazwa:}          & \textit{Sieć Ethereum}                                              \\ \hline
\textit{Opis:}           & \textit{Zdecentralizowana sieć która pozwala nam wykonywać smart contracty.} \\ \hline
\textit{Typ udziałowca:} & \textit{Nieożywiony bezpośredni}                               \\ \hline
\textit{Punkt widzenia:} & \textit{ekonomiczny, techniczny}                                       \\ \hline
\textit{Ograniczenia:}   & \textit{brak}      \\ \hline
\textit{Wymagania:}      & \textit{}                                                   \\ \hline
\end{tabular}
\end{table}

\begin{table}[H]
\begin{tabular}{|l|m{11cm}|}
\hline
\multicolumn{2}{|l|}{\textit{\textbf{Karta udziałowca}}}                               \\ \hline
\textit{Identyfikator:}  & \textit{UOB 01}                                             \\ \hline
\textit{Nazwa:}          & \textit{Zespół Projektowy}                                              \\ \hline
\textit{Opis:}           & \textit{Członkowie zespołu wykonującego system ECMarket.} \\ \hline
\textit{Typ udziałowca:} & \textit{Ożywiony bezpośredni}                               \\ \hline
\textit{Punkt widzenia:} & \textit{techniczny, wykonawczy}                                       \\ \hline
\textit{Ograniczenia:}   & \textit{brak}      \\ \hline
\textit{Wymagania:}      & \textit{}                                                   \\ \hline
\end{tabular}
\end{table}

\begin{table}[H]
\begin{tabular}{|l|m{11cm}|}
\hline
\multicolumn{2}{|l|}{\textit{\textbf{Karta udziałowca}}}                               \\ \hline
\textit{Identyfikator:}  & \textit{UOB 02}                                             \\ \hline
\textit{Nazwa:}          & \textit{Klienci}                                              \\ \hline
\textit{Opis:}           & \textit{Grupa która dzięki systemowi ECMarket kupuje i sprzedaje dobra.} \\ \hline
\textit{Typ udziałowca:} & \textit{Ożywiony bezpośredni}                               \\ \hline
\textit{Punkt widzenia:} & \textit{Operator systemu}                                       \\ \hline
\textit{Ograniczenia:}   & \textit{Klienci nie mogą narzucać wymagań systemowych.}      \\ \hline
\textit{Wymagania:}      & \textit{}                                                   \\ \hline
\end{tabular}
\end{table}

\begin{table}[H]
\begin{tabular}{|l|m{11cm}|}
\hline
\multicolumn{2}{|l|}{\textit{\textbf{Karta udziałowca}}}                               \\ \hline
\textit{Identyfikator:}  & \textit{UOB 03}                                             \\ \hline
\textit{Nazwa:}          & \textit{Współtwórcy}                                              \\ \hline
\textit{Opis:}           & \textit{Grupa która wyznacza kierunek w którym system będzie się rozwijał.} \\ \hline
\textit{Typ udziałowca:} & \textit{Ożywiony bezpośredni}                               \\ \hline
\textit{Punkt widzenia:} & \textit{techniczny, wykonawczy}                                       \\ \hline
\textit{Ograniczenia:}   & \textit{Współtwórcy nie mogą narzucać wymagań systemowych.}      \\ \hline
\textit{Wymagania:}      & \textit{}                                                   \\ \hline
\end{tabular}
\end{table}

\begin{table}[H]
\begin{tabular}{|l|m{11cm}|}
\hline
\multicolumn{2}{|l|}{\textit{\textbf{Karta udziałowca}}}                               \\ \hline
\textit{Identyfikator:}  & \textit{UNP 02}                                             \\ \hline
\textit{Nazwa:}          & \textit{Regulacje prawne}                                              \\ \hline
\textit{Opis:}           & \textit{Prawo ma duży wpływ na możliwości naszego systemu.} \\ \hline
\textit{Typ udziałowca:} & \textit{Nieożywiony pośredni}                               \\ \hline
\textit{Punkt widzenia:} & \textit{prawny}                                       \\ \hline
\textit{Ograniczenia:}   & \textit{}      \\ \hline
\textit{Wymagania:}      & \textit{}                                                   \\ \hline
\end{tabular}
\end{table}


\subsection{Ograniczenia}
Głównym ograniczeniem jest czas na wykonanie projektu, który wynosi niepełny rok.
Każda instrukcja wykonana na Wirtualnej Maszynie Ethereum (EVM), kosztuje pewną ilość etheru, co stanowi ograniczenie jak bardzo skomplikowany kod może być na niej wykonywany. Ten problem można częściowo rozwiązać delegując wykonanie niektórych zadań do klientów, takich które nie wymagają wiarygodności lub takie dla których można skonstruować dowód poprawności obliczeń, którego weryfikacja jest tańsza niż sam kod.
Pewnym ograniczeniem jest fakt, że nie opieramy się na żadnej stronie trzeciej, która byłaby wstanie stanowić autorytet i potwierdzać, że jakieś wydarzenie miało miejsce lub nie miało (np. dostarczenie produktu, który jednocześnie spełniałby pewne wymagania klienta lub innych stron). Te gwarancje próbujemy zbudować na podłożu teorii gier oraz społecznej teorii gier, by doprowadzić do sytuacji, gdzie strony nie wywiązujące się z umów tracą pieniądze lub/i ocenę, więc efektywnie zniechęcane do takich działań.
Istotnym problemem, który chcemy przynajmniej od Siebie odsunąć są kwestie prawne, a więc rosnąca kontrola władzy państwowej nad różnymi aspektami życia. By rozwiązać ten problem planujemy wydać platformę na licencji z rodziny GPL, jak również zbudować w ramach naszej platformy system repozytoriów zarządzanych przez klientów. Jednak realizacja tego drugiego wykracza poza nasze obecne możliwości czasowe.



\subsection{Zagrożenia}

\subsection{Rozważane Strategie}

\subsection{Wybór Strategii i Argumentacja}

\subsection{Przebieg Etapów Pracy}

\subsection{Charakterystyka Zespołu}

\subsection{Infrastruktura Komunikacyjna i Dokumentacyjna}

\subsection{Aspekty Społeczne}



\section{Analiza}

\subsection{Wymagania Systemowe}
\subsubsection{Wymagania ogólne i dziedzinowe}

\begin{table}[H]
\begin{tabular}{|l|l|l|l|}
\hline
\multicolumn{4}{|l|}{\textit{\textbf{Karta Wymagania}}}                                                           \\ \hline
\textit{Identyfikator:}        & \textit{WO1}            & \textit{\textbf{Priorytet:}}            & M            \\ \hline
\textit{Nazwa:}                & \multicolumn{3}{l|}{\textit{System ECMarket}}                                    \\ \hline
\textit{Opis:}                 & \multicolumn{3}{m{11cm}|}{\textit{Napisanie kodu dla ECMarket i wypuszczenie go do mainnetu}} \\ \hline
\textit{Wymagania powiązane :} & \multicolumn{3}{l|}{\textit{brak}}                                                   \\ \hline
\end{tabular}
\end{table}

\begin{table}[H]
\begin{tabular}{|l|l|l|l|}
\hline
\multicolumn{4}{|l|}{\textit{\textbf{Karta Wymagania}}}                                                           \\ \hline
\textit{Identyfikator:}        & \textit{WO2}            & \textit{\textbf{Priorytet:}}            & S            \\ \hline
\textit{Nazwa:}                & \multicolumn{3}{l|}{\textit{Podstawowa aplikacja kliencka}}                                    \\ \hline
\textit{Opis:}                 & \multicolumn{3}{m{11cm}|}{\textit{Przygotowanie podstawowej aplikacji klienckiej zapewniający graficzny interfejs do ECMarket}} \\ \hline
\textit{Wymagania powiązane :} & \multicolumn{3}{l|}{\textit{brak}}                                                   \\ \hline
\end{tabular}
\end{table}

\subsubsection{Wymagania funkcjonalne}

\begin{table}[H]
    \begin{tabular}{|l|l|l|l|}
    \hline
    \multicolumn{4}{|l|}{\textit{\textbf{Karta Wymagania}}}                                                           \\ \hline
    \textit{Identyfikator:}        & \textit{F01}            & \textit{\textbf{Priorytet:}}            & M            \\ \hline
    \textit{Nazwa:}                & \multicolumn{3}{l|}{\textit{Dostęp do podstawowych pól umowy.}}                                    \\ \hline
    \textit{Opis:}                 & \multicolumn{3}{m{11cm}|}{\textit{Każda umowa stworzona w systemie powinna dawać każdemu dostęp do takich podstawowych pól jak: lista uczestników umowy, numer bloku utworzenia, timestamp stworzenia, status umowy}} \\ \hline
    \textit{Kryteria akceptacji :} & \multicolumn{3}{l|}{\textit{Umowy posiadają metody zwracające żądane pola}}     \\ \hline
    \textit{Dane wejściowe :} & \multicolumn{3}{l|}{\textit{brak}}                                                \\ \hline
    \textit{Warunki początkowe :} & \multicolumn{3}{l|}{\textit{Umowa musi istnieć}}                                                \\ \hline
    \textit{Warunki końcowe :} & \multicolumn{3}{l|}{\textit{strona wywołująca otrzymuje żądane pole}}                                                \\ \hline
\end{tabular}
\end{table}

\begin{table}[H]
    \begin{tabular}{|l|l|l|l|}
    \hline
    \multicolumn{4}{|l|}{\textit{\textbf{Karta Wymagania}}}                                                           \\ \hline
    \textit{Identyfikator:}        & \textit{F02}            & \textit{\textbf{Priorytet:}}            & M            \\ \hline
    \textit{Nazwa:}                & \multicolumn{3}{l|}{\textit{Możliwość wpłacania, wypłacania i sprawdzenia bilansu etheru}}                                    \\ \hline
    \textit{Opis:}                 & \multicolumn{3}{m{11cm}|}{\textit{Każdy użytkownik sieci ma możliwość wpłacenia dowolnej ilość Etheru do “Virtual Wallet”-u oraz wpłacenia dowolnej ilości etheru, nie większej niż bilans danego użytkownika . Zmiana bilansu danego użytkownika, pokrywa się jeden-do-jednego z ilością wpłaconego lub wypłaconego przez niego etheru. Jeśli użytkownik nie brał udziału w umowie, ani nie wpłacał etheru jego bilans wynosi 0.  }} \\ \hline
    \textit{Kryteria akceptacji :} & \multicolumn{3}{m{11cm}|}{\textit{Każdy użytkownik może niezależnie od innych wpłacać lub wypłacać ether. Jeśli użytkownik wpłacił pewne x i nie brał udziału w żadnej umowie, może maksymalnie wypłacić x.}}    \\ \hline
    \textit{Dane wejściowe :} & \multicolumn{3}{l|}{\textit{brak}}                                                \\ \hline
    \textit{Warunki początkowe :} & \multicolumn{3}{l|}{\textit{posiadanie etheru}}                                                \\ \hline
    \textit{Warunki końcowe :} & \multicolumn{3}{l|}{\textit{zmiana bilansu}}                                                \\ \hline
    \textit{Sytuacje wyjątkowe :} & \multicolumn{3}{m{11cm}|}{\textit{Jeśli użytkownik żąda wypłaty większej ilości etheru niż posiada go w VirtualWallet, zostaje wyrzucony wyjątek.}}                                                \\ \hline
\end{tabular}
\end{table}

\begin{table}[H]
    \begin{tabular}{|l|l|l|l|}
    \hline
    \multicolumn{4}{|l|}{\textit{\textbf{Karta Wymagania}}}                                                           \\ \hline
    \textit{Identyfikator:}        & \textit{F03}            & \textit{\textbf{Priorytet:}}            & M            \\ \hline
    \textit{Nazwa:}                & \multicolumn{3}{l|}{\textit{Możliwość stworzenia umowy}}                                    \\ \hline
    \textit{Opis:}                 & \multicolumn{3}{m{11cm}|}{\textit{AgreementManager pozwala na stworzenie umowy z parametrami i typem określonym przez użytkownika. Użytkownik otrzymuje adres stworzonej umowy jako wartość zwrotną z funkcji lub jako wydarzenie o nazwie AgreementCreation z dodatkowym polem zawierającym adres utworzonej umowy. Poprawnie utworzona umowa zostaje zarejestrowana przez AgreementManagera.}} \\ \hline
    \textit{Kryteria akceptacji :} & \multicolumn{3}{m{11cm}|}{\textit{AgreementManager przy poprawnych danych tworzy poprawne umowy. Wydarzenie powinno być wygenerowane, tylko wtedy, gdy tworzenie umowy się powiodło.}}    \\ \hline
    \textit{Dane wejściowe :} & \multicolumn{3}{l|}{\textit{brak}}                                                \\ \hline
    \textit{Warunki początkowe :} & \multicolumn{3}{l|}{\textit{brak}}                                                \\ \hline
    \textit{Warunki końcowe :} & \multicolumn{3}{m{11cm}|}{\textit{utworzenie nowej umowy i możliwość zarządzania nią z pomocą innych funkcji AgreementManagera}}                                                \\ \hline
\end{tabular}
\end{table}

\begin{table}[H]
    \begin{tabular}{|l|l|l|l|}
    \hline
    \multicolumn{4}{|l|}{\textit{\textbf{Karta Wymagania}}}                                                           \\ \hline
    \textit{Identyfikator:}        & \textit{F04}            & \textit{\textbf{Priorytet:}}            & M            \\ \hline
    \textit{Nazwa:}                & \multicolumn{3}{l|}{\textit{Możliwość wyszukania umowy}}                                    \\ \hline
    \textit{Opis:}                 & \multicolumn{3}{m{11cm}|}{\textit{AgreementManager powinien mieć możliwość zwrócenia strony z ostatnio stworzonymi umowami.}} \\ \hline
    \textit{Kryteria akceptacji :} & \multicolumn{3}{m{11cm}|}{\textit{AgreementManager przy poprawnych danych tworzy poprawne umowy. Wydarzenie powinno być wygenerowane, tylko wtedy, gdy tworzenie umowy się powiodło.}}    \\ \hline
    \textit{Dane wejściowe :} & \multicolumn{3}{l|}{\textit{brak}}                                                \\ \hline
    \textit{Warunki początkowe :} & \multicolumn{3}{l|}{\textit{AgreementManager musi posiadać zarejestrowane umowy}}                                                \\ \hline
    \textit{Warunki końcowe :} & \multicolumn{3}{m{11cm}|}{\textit{utworzenie nowej umowy i możliwość zarządzania nią z pomocą innych funkcji AgreementManagera}}                                                \\ \hline
    \textit{Sytuacje wyjątkowe :} & \multicolumn{3}{m{11cm}|}{\textit{W przypadku braku jakichkolwiek umów zostaje zwrócona tablica z samymi zerami}} \\ \hline
\end{tabular}
\end{table}

\begin{table}[H]
    \begin{tabular}{|l|l|l|l|}
    \hline
    \multicolumn{4}{|l|}{\textit{\textbf{Karta Wymagania}}}                                                           \\ \hline
    \textit{Identyfikator:}        & \textit{F05}            & \textit{\textbf{Priorytet:}}            & M            \\ \hline
    \textit{Nazwa:}                & \multicolumn{3}{l|}{\textit{Możliwość usunięcia nowo-stworzonej umowy A1.1.}}                                    \\ \hline
    \textit{Opis:}                 & \multicolumn{3}{m{11cm}|}{\textit{Umowy stworzone przez użytkowników mogą być przez nich usuwane}} \\ \hline
    \textit{Kryteria akceptacji :} & \multicolumn{3}{m{11cm}|}{\textit{Tylko wskazana umowa ulega usunięciu}}    \\ \hline
    \textit{Dane wejściowe :} & \multicolumn{3}{l|}{\textit{brak}}                                                \\ \hline
    \textit{Warunki początkowe :} & \multicolumn{3}{m{11cm}|}{\textit{Umowa jest w stanie New, ważna (nie przekroczyła daty ważności), a stroną wywołująca jest jest twórca}}                                                \\ \hline
    \textit{Warunki końcowe :} & \multicolumn{3}{m{11cm}|}{\textit{Kontrakt umowy zostaje zniszczony, wyrejestrowany z AgreementManagera, a środki wpłacone przez petentów zostają do nich odesłane}}                                                \\ \hline
    \textit{Sytuacje wyjątkowe :} & \multicolumn{3}{m{11cm}|}{\textit{brak}} \\ \hline
\end{tabular}
\end{table}

\begin{table}[H]
    \begin{tabular}{|l|l|l|l|}
    \hline
    \multicolumn{4}{|l|}{\textit{\textbf{Karta Wymagania}}}                                                           \\ \hline
    \textit{Identyfikator:}        & \textit{F06}            & \textit{\textbf{Priorytet:}}            & M            \\ \hline
    \textit{Nazwa:}                & \multicolumn{3}{l|}{\textit{Możliwość usunięcia przeterminowanej umowy A1.1.}}                                    \\ \hline
    \textit{Opis:}                 & \multicolumn{3}{m{11cm}|}{\textit{Przeterminowane umowy mogą być usunięte przez każdego.}} \\ \hline
    \textit{Kryteria akceptacji :} & \multicolumn{3}{m{11cm}|}{\textit{Tylko wskazana umowa ulega usunięciu}}    \\ \hline
    \textit{Dane wejściowe :} & \multicolumn{3}{l|}{\textit{brak}}                                                \\ \hline
    \textit{Warunki początkowe :} & \multicolumn{3}{l|}{\textit{umowa przekroczyła swoją datę ważności}}                                                \\ \hline
    \textit{Warunki końcowe :} & \multicolumn{3}{m{11cm}|}{\textit{Kontrakt umowy zostaje zniszczony, wyrejestrowany z AgreementManagera, a środki wpłacone przez petentów zostają do nich odesłane}}                                                \\ \hline
    \textit{Sytuacje wyjątkowe :} & \multicolumn{3}{m{11cm}|}{\textit{brak}} \\ \hline
\end{tabular}
\end{table}

\begin{table}[H]
    \begin{tabular}{|l|l|l|l|}
    \hline
    \multicolumn{4}{|l|}{\textit{\textbf{Karta Wymagania}}}                                                           \\ \hline
    \textit{Identyfikator:}        & \textit{F07}            & \textit{\textbf{Priorytet:}}            & M            \\ \hline
    \textit{Nazwa:}                & \multicolumn{3}{l|}{\textit{Możliwość usunięcia zakończonej umowy A1.1.}}                                    \\ \hline
    \textit{Opis:}                 & \multicolumn{3}{m{11cm}|}{\textit{Umowy zakończone mogą być usuwane przez użytkowników}} \\ \hline
    \textit{Kryteria akceptacji :} & \multicolumn{3}{m{11cm}|}{\textit{Tylko wskazana umowa ulega usunięciu}}    \\ \hline
    \textit{Dane wejściowe :} & \multicolumn{3}{l|}{\textit{brak}}                                                \\ \hline
    \textit{Warunki początkowe :} & \multicolumn{3}{m{11cm}|}{\textit{Umowa jest w stanie Done, ważna (nie przekroczyła daty ważności), a stroną wywołująca jest jest twórca}}                                                \\ \hline
    \textit{Warunki końcowe :} & \multicolumn{3}{m{11cm}|}{\textit{Kontrakt umowy zostaje zniszczony, wyrejestrowany z AgreementManagera, a środki wpłacone przez petentów zostają do nich odesłane}}                                                \\ \hline
    \textit{Sytuacje wyjątkowe :} & \multicolumn{3}{m{11cm}|}{\textit{brak}} \\ \hline
\end{tabular}
\end{table}

\begin{table}[H]
    \begin{tabular}{|l|l|l|l|}
    \hline
    \multicolumn{4}{|l|}{\textit{\textbf{Karta Wymagania}}}                                                           \\ \hline
    \textit{Identyfikator:}        & \textit{F08}            & \textit{\textbf{Priorytet:}}            & M            \\ \hline
    \textit{Nazwa:}                & \multicolumn{3}{l|}{\textit{Możliwość oceniania stron kontraktu}}                                    \\ \hline
    \textit{Opis:}                 & \multicolumn{3}{m{11cm}|}{\textit{Obie strony mają możliwość ocenienia drugiej strony przed zakończeniem kontraktu}} \\ \hline
    \textit{Kryteria akceptacji :} & \multicolumn{3}{m{11cm}|}{\textit{Warunki Satysfakcji (Szczegóły dodane na potrzeby  testów akceptacyjnych)}}    \\ \hline
    \textit{Dane wejściowe :} & \multicolumn{3}{l|}{\textit{Kontrakt został zapoczątkowany}}                                                \\ \hline
    \textit{Warunki początkowe :} & \multicolumn{3}{m{11cm}|}{\textit{Umowa jest w stanie Done, ważna (nie przekroczyła daty ważności), a stroną wywołująca jest jest twórca}}                                                \\ \hline
    \textit{Warunki końcowe :} & \multicolumn{3}{m{11cm}|}{\textit{Kontrakt umowy zostaje zniszczony, wyrejestrowany z AgreementManagera, a środki wpłacone przez petentów zostają do nich odesłane}}                                                \\ \hline
    \textit{Sytuacje wyjątkowe :} & \multicolumn{3}{m{11cm}|}{\textit{brak}} \\ \hline
\end{tabular}
\end{table}

\begin{table}[H]
    \begin{tabular}{|l|l|l|l|}
    \hline
    \multicolumn{4}{|l|}{\textit{\textbf{Karta Wymagania}}}                                                           \\ \hline
    \textit{Identyfikator:}        & \textit{F09}            & \textit{\textbf{Priorytet:}}            & M            \\ \hline
    \textit{Nazwa:}                & \multicolumn{3}{l|}{\textit{Wykonanie umowy A1.1}}                                    \\ \hline
    \textit{Opis:}                 & \multicolumn{3}{m{11cm}|}{\textit{Umowa A1.1 posiada funkcje join, accept i conclude. Strony chętne do uczestnictwa w umowie wykonują join jednocześnie przekazując środki określone w polu cena na konto umowy. Właściciel umowy może wykonać accept wskazując z kim chce przeprowadzić transakcję, jednocześnie pobierając środki z konta umowy. Po tym fakcie obie strony mogą się ocenić z pomocą conclude}} \\ \hline
    \textit{Kryteria akceptacji :} & \multicolumn{3}{m{11cm}|}{\textit{AgreementManager zarejestrował wykonanie umowy}}    \\ \hline
    \textit{Dane wejściowe :} & \multicolumn{3}{l|}{\textit{Kontrakt został zapoczątkowany}}                                                \\ \hline
    \textit{Warunki początkowe :} & \multicolumn{3}{l|}{\textit{brak}}                                                \\ \hline
    \textit{Warunki końcowe :} & \multicolumn{3}{m{11cm}|}{\textit{brak}}                                                \\ \hline
    \textit{Sytuacje wyjątkowe :} & \multicolumn{3}{m{11cm}|}{\textit{brak}} \\ \hline
\end{tabular}
\end{table}

\begin{table}[H]
    \begin{tabular}{|l|l|l|l|}
    \hline
    \multicolumn{4}{|l|}{\textit{\textbf{Karta Wymagania}}}                                                           \\ \hline
    \textit{Identyfikator:}        & \textit{F10}            & \textit{\textbf{Priorytet:}}            & M            \\ \hline
    \textit{Nazwa:}                & \multicolumn{3}{l|}{\textit{Możliwość wypłacania środków z kontraktu}}                                    \\ \hline
    \textit{Opis:}                 & \multicolumn{3}{m{11cm}|}{\textit{Petent powinien mieć możliwość wypłacenia środków.}} \\ \hline
    \textit{Kryteria akceptacji :} & \multicolumn{3}{m{11cm}|}{\textit{brak}}    \\ \hline
    \textit{Dane wejściowe :} & \multicolumn{3}{l|}{\textit{brak}}                                                \\ \hline
    \textit{Warunki początkowe :} & \multicolumn{3}{l|}{\textit{Wpłacenie środków do kontraktu.}}                                                \\ \hline
    \textit{Warunki końcowe :} & \multicolumn{3}{m{11cm}|}{\textit{brak}}                                                \\ \hline
    \textit{Sytuacje wyjątkowe :} & \multicolumn{3}{m{11cm}|}{\textit{brak}} \\ \hline
\end{tabular}
\end{table}

\begin{table}[H]
    \begin{tabular}{|l|l|l|l|}
    \hline
    \multicolumn{4}{|l|}{\textit{\textbf{Karta Wymagania}}}                                                           \\ \hline
    \textit{Identyfikator:}        & \textit{F11}            & \textit{\textbf{Priorytet:}}            & M            \\ \hline
    \textit{Nazwa:}                & \multicolumn{3}{l|}{\textit{Ustawianie czasu ważności umowy przy tworzeniu}}                                    \\ \hline
    \textit{Opis:}                 & \multicolumn{3}{m{11cm}|}{\textit{Użytkownik ma możliwość określenia ile czasu/bloków dana umowa będzie ważna. Przedział dozwolonych czasów ważności jest ograniczony od dołu i od góry i zależny od typu umowy. Po utworzeniu umowy nie jest możliwa modyfikacja tej wartości}} \\ \hline
    \textit{Kryteria akceptacji :} & \multicolumn{3}{m{11cm}|}{\textit{brak}}    \\ \hline
    \textit{Dane wejściowe :} & \multicolumn{3}{l|}{\textit{brak}}                                                \\ \hline
    \textit{Warunki początkowe :} & \multicolumn{3}{l|}{\textit{brak}}                                                \\ \hline
    \textit{Warunki końcowe :} & \multicolumn{3}{m{11cm}|}{\textit{brak}}                                                \\ \hline
    \textit{Sytuacje wyjątkowe :} & \multicolumn{3}{m{11cm}|}{\textit{brak}} \\ \hline
\end{tabular}
\end{table}

\subsubsection{Wymagania niefunkcjonalne}

\begin{table}[H]
    \begin{tabular}{|l|l|l|l|}
    \hline
    \multicolumn{4}{|l|}{\textit{\textbf{Karta Wymagania}}}                                                           \\ \hline
    \textit{Identyfikator:}        & \textit{NF01}            & \textit{\textbf{Priorytet:}}            & S            \\ \hline
    \textit{Nazwa:}                & \multicolumn{3}{l|}{\textit{System zachęty do usuwania przeterminowanych umów}}                                    \\ \hline
    \textit{Opis:}                 & \multicolumn{3}{m{11cm}|}{\textit{Stworzenie umowy wymaga wpłacenia niedużej kaucji (trzeba pomyśleć jak “niedużej”), która jest wypłacana temu, który usunie umowę z systemu. Jeśli umowa jest ważna, tylko osoba uprawniona (zazwyczaj twórca umowy) może ją usunąć i otrzymać z powrotem kaucję. Jeśli umowa jest przeterminowana osoba uprawniona ma wyłączność na usunięcie i zagarnięcie kaucji, ale tylko przez pewien okres ochronny (jak długi) po dacie utraty ważności. Po okresie ochronnym każdy może usunąć umowę i otrzymać daną kaucję}} \\ \hline
    \textit{Kryteria akceptacji :} & \multicolumn{3}{m{11cm}|}{\textit{brak}}    \\ \hline

\end{tabular}
\end{table}

\begin{table}[H]
    \begin{tabular}{|l|l|l|l|}
    \hline
    \multicolumn{4}{|l|}{\textit{\textbf{Karta Wymagania}}}                                                           \\ \hline
    \textit{Identyfikator:}        & \textit{NF02}            & \textit{\textbf{Priorytet:}}            & M            \\ \hline
    \textit{Nazwa:}                & \multicolumn{3}{m{11cm}|}{\textit{System ocen - efekty wystawienia pozytywnej / neutralnej / negatywnej oceny}}                                    \\ \hline
    \textit{Opis:}                 & \multicolumn{3}{m{11cm}|}{\textit{Jeśli strona P1 umowy A wystawia pewną pozytywną ocenę r1 stronie P2, wtedy wiarygodność P2 w2 wzrasta do w2’ według pewnej miary D, tak że D(w2) < D(w2’).
    Jeśli strona P1 umowy A wystawia pewną neutralną ocenę O stronie P2, wtedy wiarygodność P2 w2 nie ulega zmianie.
    Jeśli strona P1 umowy A wystawia pewną negatywną ocenę r1 stronie P2, wtedy wiarygodność P2 w2 spada do w2’ według pewnej miary D, tak że D(w2) > D(w2’).}} \\ \hline
    \textit{Kryteria akceptacji :} & \multicolumn{3}{m{11cm}|}{\textit{brak}}    \\ \hline

\end{tabular}
\end{table}

\begin{table}[H]
    \begin{tabular}{|l|l|l|l|}
    \hline
    \multicolumn{4}{|l|}{\textit{\textbf{Karta Wymagania}}}                                                           \\ \hline
    \textit{Identyfikator:}        & \textit{NF03}            & \textit{\textbf{Priorytet:}}            & M            \\ \hline
    \textit{Nazwa:}                & \multicolumn{3}{l|}{\textit{System ocen - tłumienie}}                                    \\ \hline
    \textit{Opis:}                 & \multicolumn{3}{m{11cm}|}{\textit{Strony P1 i P2 biorą udział w zbiorze umów {A1, A2, …, An}. P1 w kolejnych umowach A1, A2, ..., An wystawia P2 oceny r1, r2, …,rn, co daje ciąg wiarygodności dla P2 w1, w2, …, wn. Wtedy każdy z tych elementów tego ciągu spełnia zależność dR kNi>k Dwi+1-Dwi<i-d.
    To oznacza, że od pewnego momentu (jakiego) waga ocen wystawianych przez P1 dla P2 zaczyna co najmniej wykładniczo spadać.}} \\ \hline
    \textit{Kryteria akceptacji :} & \multicolumn{3}{m{11cm}|}{\textit{brak}}    \\ \hline

\end{tabular}
\end{table}

\begin{table}[H]
    \begin{tabular}{|l|l|l|l|}
    \hline
    \multicolumn{4}{|l|}{\textit{\textbf{Karta Wymagania}}}                                                           \\ \hline
    \textit{Identyfikator:}        & \textit{NF04}            & \textit{\textbf{Priorytet:}}            & M            \\ \hline
    \textit{Nazwa:}                & \multicolumn{3}{l|}{\textit{System ocen - istotność wiarygodności partnera}}                                    \\ \hline
    \textit{Opis:}                 & \multicolumn{3}{m{11cm}|}{\textit{Strony P1, P2, P3, P4 mają wiarygodności w1, w2, w3, w4 gdzie D(w1) < D(w2) i D(w3) = D(w4). Wtedy, jeśli P1 i P3 biorą udział w umowie tego samego typu co P2 i P4 i P1 wystawi P3 ocenę r1, a P2 wystawi P4 r2, takie że r1 = r2, to D(w3)-D(w3') < D(w4)-D(w4').
    Z tego wynika, że oceny wystawiane przez stronę o większej wiarygodności, mają większą wagę niż oceny strony o niższej
    }} \\ \hline
    \textit{Kryteria akceptacji :} & \multicolumn{3}{m{11cm}|}{\textit{brak}}    \\ \hline

\end{tabular}
\end{table}


\subsubsection{Wymagania projektowo-wdrożeniowe}

\begin{table}[H]
    \begin{tabular}{|l|l|l|l|}
    \hline
    \multicolumn{4}{|l|}{\textit{\textbf{Karta Wymagania}}}                                                           \\ \hline
    \textit{Identyfikator:}        & \textit{ŚD01}            & \textit{\textbf{Priorytet:}}            & M            \\ \hline
    \textit{Nazwa:}                & \multicolumn{3}{l|}{\textit{Ekosystem Ethereum}}                                    \\ \hline
    \textit{Opis:}                 & \multicolumn{3}{m{11cm}|}{\textit{Wszystkie produkty wygenerowane w czasie trwania projektu opierają się na technologiach pochodzących z ekosystemu Ethereum, takich jak sieć Ethereum, sieć Swarm, sieć Whisper, biblioteki Web3.}} \\ \hline
    \textit{Kryteria akceptacji :} & \multicolumn{3}{m{11cm}|}{\textit{brak}}    \\ \hline

\end{tabular}
\end{table}

\subsubsection{Wymagania dotyczące procesów wytwarzania}

\begin{table}[H]
    \begin{tabular}{|l|l|l|l|}
    \hline
    \multicolumn{4}{|l|}{\textit{\textbf{Karta Wymagania}}}                                                           \\ \hline
    \textit{Identyfikator:}        & \textit{PW01}            & \textit{\textbf{Status:}}            & M            \\ \hline
    \textit{Nazwa:}                & \multicolumn{3}{l|}{\textit{TDD - Test Driven Development}}                                    \\ \hline
    \textit{Opis:}                 & \multicolumn{3}{m{11cm}|}{\textit{Każda zmiana lub nowa funkcjonalność wprowadzana do systemu, musi być najpierw zamodelowana z pomocą testów. Dopiero po tym testy specyfikują jak docelowy kod ma wyglądać.}} \\ \hline
    \textit{Kryteria akceptacji :} & \multicolumn{3}{m{11cm}|}{\textit{brak}}    \\ \hline
\end{tabular}
\end{table}

\begin{table}[H]
    \begin{tabular}{|l|l|l|l|}
    \hline
    \multicolumn{4}{|l|}{\textit{\textbf{Karta Wymagania}}}                                                           \\ \hline
    \textit{Identyfikator:}        & \textit{PW02}            & \textit{\textbf{Status:}}            & M            \\ \hline
    \textit{Nazwa:}                & \multicolumn{3}{l|}{\textit{Określony git-flow}}                                    \\ \hline
    \textit{Opis:}                 & \multicolumn{3}{m{11cm}|}{\textit{Każda nowa funkcjonalność lub ich zbiór powinien być opisany w Issue. Branche z implementacją danego Issue powinny mieć postać ‘feature/IS\%numer\_issue\%/\%dodatkowy\_opis\%’. Zmianny do testów i do kodu powinny być zawarte w osobnych commitach. Wiadomość w commitach powinny mieć postać ‘IS\%numer\_issue\% test/code - \%opis\_commitu\%’.}} \\ \hline
    \textit{Kryteria akceptacji :} & \multicolumn{3}{m{11cm}|}{\textit{brak}}    \\ \hline
\end{tabular}
\end{table}

\subsection{Specyfikacja Przypadków Uzycia}
\subsubsection{Diagram przypadków uzycia}

\includegraphics[width=1.0\textwidth]{Use_Case_Diagram}

\subsubsection{Diagram sekwencji}

\newpage
\subsubsection{Maszyny Stanowe}

Umowa A1.1\\
\includegraphics[width=1.0\textwidth]{SM-A1_1}


\subsection{Model Obiektowy Systemu}

\subsubsection{Back-End}

\includegraphics[width=1.0\textwidth]{Contract_Class_Diagram_Class_Diagram}

\subsubsection{Front-End}

\subsubsection{Sieć procesów}

\subsection{Analiza Komercyjna}

\section{Projektowanie}
\section{Architektura}

\section{Przyrost Pierwszy}
\subsection{Cele przyrostu}
\subsection{Analiza back-end'u}
\subsection{Projektowanie back-end'u}
\subsection{Testy walidacyjne}
\subsection{Implementacja}
\subsection{Lista zmian}

\section{Przyrost Drugi}
\subsection{Cele przyrostu}
\subsection{Analiza front-end'u}
\subsection{Projektowanie front-end'u}
\subsection{Testy walidacyjne}
\subsection{Testy integracyjne}
\subsection{Implementacja}
\subsection{Lista zmian}

\section{Przyrost Trzeci}
\subsection{Cele przyrostu}
\subsection{Projektowanie back-end'u}
\subsection{Projektowanie front-end'u}
\subsection{Testy walidacyjne}
\subsection{Testy integracyjne}
\subsection{Testy systemowe}
\subsection{Implementacja}
\subsection{Lista zmian}

\newpage  % ---------------------- ACHTUNG! tego żyda trzeba stąd wykurzyć podczas wykańczania dokumentacji

\section{Instrukcja Obsługi}

\section{Prezentacja Projektu}

\section{Raport Końcowy}

\section{Całkowity Nakład Pracy}

\section{Wkład Własny w Projekt}

\section{Podsumowanie}

\section{Słownik Pojęć}
•    Ekosystem Ethereum (zespół technologii Ethereum) – Ethereum jako środowisko wykonawcze oparte na blockchainie, rozproszona przestrzeń dyskowa Swarm, system komunikatów Whisper, biblioteki web3 i aplikacje klienckie Ethereum. \newline
•    Ethereum – środowisko wykonawcze oparte na blockchainie dla inteligentnych kontraktów. \newline
•    Blockchain – struktura danych mogąca przechowywać dowolne dane w postaci bloków, skonstruowana na zasadzie listy (lub w pewnym stopniu drzewa), gdzie wskaźnikami do poprzednich bloków są ich kryptograficzne hasze.\newline
•    Drzewa Merkle – struktura danych mogąca przechowywać dowolne dane, skonstruowana na zasadzie drzewa, gdzie wskaźnikami na rodziców są ich kryptograficzne hasze. \newline
•    Mainnet – główna sieć Ethereum o id równym 1. Na niej odbywają się transakcje z użyciem etheru posiadającym wartość dla użytkowników. \newline
•    Swarm – zdecentralizowana przestrzeń dyskowa oparta na drzewach Merkle, zintegrowana z Ethereum. \newline
•    Whisper – zdecentralizowany, wysoce anonimowy system komunikatów zintegrowany z Ethereum.\newline
•    Wirtualna Maszyna Ethereum – część Ethereum wykonująca bytecode skompilowanych kontraktów. \newline
•    Ether – jednostka płatności/możliwości wykonania kodu. Jest podstawową walutą w ekosystemie Ethereum. 1 ether dzieli się na $10$\textsuperscript{18} wei. \newline

\section{Bibliografia}
bibtex

\section{Załączniki}

\begin{thebibliography}{9}
\bibitem{trustPOL}
	CYBULSKA A., PANKOWSKI K.
	\textit{STOSUNEK DO INSTYTUCJI PAŃSTWA ORAZ PARTII POLITYCZNYCH PO 25 LATACH},
	CBOS,
	Warszawa 2014,
	NR 68/2014,
	ISSN 2353-5822,
	URL \\\texttt{http://okragly-stol.pl/wp-content/uploads/2014/09/CBOS\_Stosunek\_do\_panstwa.pdf}

\bibitem{trustUSA}
	\textit{Beyond Distrust: How Americans View Their Government}
	[\textit{Broad criticism, but positive performance ratings in many areas}],
	Pew Research Cente,
	listopad 2015,
	URL \\\texttt{http://www.people-press.org/2015/11/23/beyond-distrust-how-americans-view-their-government/}
	
\bibitem{ethereum2018whitepaper}
	\textit{White Paper}
	[\textit{A Next-Generation Smart Contract and Decentralized Application Platform}],
	Ethereum Foundation,
	sierpień 2018,
	URL \\\texttt{https://github.com/ethereum/wiki/wiki/White-Paper}
	

\end{thebibliography}

\end{document}