\documentclass[11pt,oneside]{article}
\usepackage[utf8]{inputenc}
\usepackage[polish]{babel}
\usepackage[T1]{fontenc}
\usepackage[paper=A4,pagesize]{typearea}
\usepackage{graphicx}
\usepackage{comment}
\usepackage{multirow}
\usepackage{courier}
\usepackage{float}
\usepackage{array}
\usepackage{amsmath}
\usepackage{url}
\graphicspath{ {./images/} }

\begin{document}

\begin{figure}[H]

\centering
\includegraphics[width=1.0\textwidth]{logopjwstk}

\begin{Huge}
\begin{center}
	\texttt{\textbf {KARTA PROJEKTU}}
\end{center}
\end{Huge}

\includegraphics[width=0.94\textwidth]{cheat1}
\includegraphics[width=0.94\textwidth]{cheat2}
\end{figure}

\begin{figure}[H]
\includegraphics[width=.4\textwidth]{cheat3}
\end{figure}

\newpage

\setcounter{tocdepth}{2}
\tableofcontents

\newpage

\section{Wstęp}

Na fali pojawiających się nowych zdecentralizowanych aplikacji,
protokołów komunikacj p2p, algorytmów rozproszonego konsensusu i ideii Internetu 3.0,
narodziła się w wśród nas koncepcja systemu wspierającego i nadzorującego szeroko rozumianą wymianę obiektów,
w szczególności dóbr fizycznych, który wpasowywałby się do tego nowego, wspaniałego świata.
Porwani przez tę wizję, poświęcamy tę pracą pierwszemu podstawowemu wydaniu systemu ECMarket,
ktore będzie wstępem do stworzenia analogu do serwisów aukcyjnych typu Allegro czy Ebay w uniwersum
aplikacji zdecentralizowanych.

\newpage
\section{Podziękowania}



\newpage
\section{Opis problemu}

\subsection{Prezentacja problemu}

W związku z rosnącą centralizacją Internetu, problemem cenzury oraz spadkiem zaufania do
instytucji państwa\cite{work:trustPOL,website:trustUSA} jak i korporacji, powstało wiele projektów mających
na celu rozproszenie tej koncentracji władzy, jednak przez długi czas rozwiązania te nie zyskiwały
duzej popularności poza specjalistycznymi dziedzinami.\\

Odwilż w tej kwestii przyniósł protokół Kademila\cite{work:Kademila}, na której oparte zostały sieci wymiany plików takie
jak BitTorrent, czy Kad Network. Ich funkcjonalność ograniczała się do wymiany plików, jednak nie była
pozbawiona wad takich jak niska żywotność niepopularnych plików, egoistyczne zachowania klientów sieci
(leeching), brak odporności na ataki DDoS (Distributed Denial-of-Service).\\

Przełom nastąpił w 2009 w raz z prezentacją technologi Bitcoin\cite{work:BitcoinPaper}, która oferowała roziązanie wymienionych problemów, z którymi borykały się poprzednie rozwiązania. Zastosowanie mechanizmu PoW (Proof-of-Work), jako zabezpieczenia przed atakami DDos i Sybili, użycie nowej struktury --- blockchainu, jako rozproszonej bazy danych, a przede wszystki, innowacyjne jak na tamte czasy, zastosowanie zachęty finansowej dla klientów zbierających transakcje i przechowujących blockchain, by zachować żywotność sieci, storzyło mieszankę, która zagwarantowała sukces tej technologii. Dowodem tego jest osiągnięcie przez Bitcoina pod koniec listopada 2013 wartości 1200 USD/BTC.\\

Te wydarzenia dały początek nowej dziedzinie wiedzy nazawanej krypto-ekonomią oraz zapoczątkowały eksplozję nowych kryptowalut i rozwiązań pochodnych. Jednym z nowych rozwiązań, które pojawiło się na horyzoncie na przełomie 2013 i 2014 jest Ethereum\cite{website:ethereum2018whitepaper}. Rozwiązanie zaproponowane przez Vitalika Buterina, Gavina Wooda oraz Josepha Lubina, oferuje rozproszoną platformę obliczeniową z systemem operacyjnym, obsługującym inteligentne kontrakty (smart contracts). Podobnie jak Bitcoin oferuje możliwość przesyłania wewnętrznych tokenów (w przypadku Ethereum nazywanych Etherem) między użytkownikami, a dzięki zmodyfikowanemu konsensusowi Nakamoto (Ethereum 2.0 będzie używać innych mechanizmów), daje podobne gwarancje bezpieczeństawa. Najbardziej istotna różnica występuje jednak w systemie skryptów (inteligentnych kontraktów), te oferowane przez Ethereum, w przeciwieństwie do Bitcoina, są kompletne w sensie Turinga i mogą posiadać bardziej skomplikowany stan.

\subsection{Proponowane rozwiązanie}

Dzięki takiemu zestawowi cech, Ethereum stanowi dobrą platformę, do budowania własnych, wyspecjalizowanych systemów, które są bezpieczne i niezależne od potencjalnie nieuczciwych stron trzecich. Szczególnym przypadkiem się tutaj platformy handlowe, które agregują oferty sprzedających, pośredniczą w wykonaniu opłat i dostarczają narzędzia od oceny kupujących i sprzedających. W przypadku scentralizowanych wersji takich platform, bardzo częstym problemem jest faworyzowanie pewnych grup sprzedawców, różne stopy opłat dla różnych uczestników i stronniczy moderatorzy.\\

Mając na uwadze wymienione wcześniej fakty, proponowanym przez nas rozwiązaniem problemów z centralizacją i nieuczciwymi pośrednikami, przynajmniej w sferze platform handlowych, jest stworzenie takiej aplikacji w ekosystemie Ethereum i zapewnienie, że w późniejszych etapach rozwoju, będzie ona rozwijana w ścisłej współpracy ze społecznością sieci. Z tego powodu chcielibyśmy zaprezentować nasze autorskie podejście w postaci projektu ECMarket.


\subsection{Rich picture}

\begin{figure}[H]
	\includegraphics[width=1.0\textwidth]{richpicture}
\end{figure}



\subsection{Konkurencja}

ECMarket ma na chwilę obecną kilka wartych uwagi konkurentów, którzy operują w trochę innych, ale pokrywających się z naszym obszarach.
\subsubsection{Konkurencja zdecentralizowana}
\begin{itemize}
	\item \textit{uPort} Pozwala użytkownikom tylko zarejestrować własną tożsamość w Ethereum i zarządzać własnymi kluczami i danymi w bezpiecznym portfelu \cite{website:uPort}.
	\item \textit{Ujo Music} Handluje tylko utworami muzycznymi, nie posiada również rozbudowanego systemu konktraktów \cite{website:UjoMusic}.
	\item \textit{IDEX} Oferuje tylko usługi wymiany tokenów ERC20 \cite{website:IDEX}.
	\item \textit{OpenBazaar} Oferuje tylko jeden rozdaj umowy (umowa między dwiema stronami z depozytem)\cite{website:openbazaar}. 
\end{itemize}
\subsubsection{Konkurencja scentralizowana}
\begin{itemize}
    \item \textit{eBay} Największy serwis aukcyjny na świecie. Każdy zarejestrowany użytkownik może stworzyć aukcję lub wziąć udział w niej udział. Na eBayu możemy płaci tylko w standardowy sposób, czyli kartą kredytową, debetową lub za pośrednictwem PayPala. Strona nie oferuje żadnych płatności za pomocą kryptowalut. eBay posiada regulamin, w którym jest jasno określone jakie przedmioty mogą być obiektem sprzedaży. Rownież definuje on sytuacje, w ktorych administracja podejmie kroki, by zamknąć aukcję. eBay również oferuje biuro obsługi klienta dla każdego, kto potrzebuje pomocy w sprawie oszustw, czy też problemu z płatnościami.
    \item \textit{Allegro} Jest polskim odpowiednikiem amerykańskiej platfromy eBay. Jest to portal, który oprócz standardowych transakcji pomiedzy użytkowikami, pozwala na umieszczanie ofert przez profesjonalnych handlarzy. Allegro posiada system ocen który ma za zadanie weryfikować każdego użytkownika, jednak w trudnych przypadkach również oferuje pomoc w ich rozwiązaniu. Tak samo jak amerykański odpowiednik, nie oferuje płatności za pomocą kryptowalut. 
\end{itemize}
    Nasza oferta stanowi swoiste połączenie i wariację niektórych własności oferowanych przez wyżej wymienionych konkurentów, dlatego sądzimy, że mamy szansę wypromować się na tym rynku.


\newpage
\section{Planowanie}

\subsection{Cele i zakres projektu}

Celem projektu jest dostarczenie odpornej na cenzurę, ogólnodostępnej, otwartej platformy, w ramach której możliwe jest przeprowadzanie kupna oraz sprzedaży dóbr w sposób niezależny od osób trzecich lub w ramach grup handlowych, tworzonych przez użytkowników systemu z określanymi przez nich zasadami.

W pierwszym stadium systemu, realizowanym w ramach tego projektu, planujemy pobierać mały, stały procent od kwoty przesyłanej w transakcjach między użytkownikami, by zapewnić utrzymanie i dalszy rozwój projektu.
Naszą platformę kierujemy do wszystkich pragnących niezależnego i otwartego systemu do handlu.


\subsection{Kontekst projektu}

System będzie docelowo uruchomiony na Testnecie Ethereum i będzie dostarczał rozproszoną, odporną na cenzurę, na ile pozwala na to protokół\cite{website:TheDAOHack, website:ethereum2018whitepaper}, platformę do handlu.
Użytkownicy w pierwszym stadium rozwoju platformy będą się dzielić na Deweloperów i na Klientów. Zadaniem Deweloperów będzie dostarczanie kodu i jego deployment oraz ponoszenie kosztów tych operacji. Deweloperzy będą również beneficjentami oprocentowania transakcji. Klientami nazywamy wszystkich użytkowników, którzy dokonują transakcji. Klienci mogą zarządzać swoimi portfelami, oraz mogą wchodzić w interakcje z dostępnymi dla nich umowami.Liczba Klientów nie jest w żaden sposób ograniczona.

\subsection{Analiza biznesowa}

Pierwsze wydanie systemu dostarczy klientom zestaw umów gotowych umów do użycia i system ocen, który będzie dawać nam przewagę nad konkurencją w takich kwestiach jak:
\begin{itemize}
\item Wiarygodny system ocen --- stanowiący kluczową wartość oferowaną przez nasze rozwiązanie. W naszym systemie nie występuje administrator, to też niezwykle istotne było stworzenie środowiska promującego bycie uczciwym. Podobne rozwiązania możemy zaobserwować w przeróżnych serwisach aukcyjnych, gdzie stanowią pewnego rodzaju informację dla kupującego o tym czy może ufać sprzedającemu. Nasze rozwiązanie pozwala na ocenienie drugiej strony umowy --- takowej oceny dokonuje także sprzedający, a nie tylko kupujący. Dodatkowo ocenom tym zostają nadane odpowiednie wagi tak by zapobiec próbom wpłynięcia na system ocen np.\ poprzez wchodzenie w małe transakcje z dużą ilością kont powiązanych ze sprzedającym. Wagi przyznawane są na podstawie ocen posiadanych przez stronę wystawiającą takową ocenę, a także jej wiarygodności dla społeczności platformy wyrażoną poprzez oceny pozytywne oraz ogólną ilość ocen.
\item Niezależność od walut FIAT --- na platformie płatności odbywać się będą przy pomocy Ether'u (ETH), kryptowaluty, której wartość jest niezależna od żadnego centralnego organu finansowego. Naszych środków nie da się zamrozić tak jak np.\ w przypadku Peso Argentyńskiego (ARS) w 2001 roku podczas kryzysu finansowego. Nie grozi nam także hiperinflacja, pieniędzy nie da się po prostu dodrukować przez specyfikę pozyskiwania Ether'u (ETH).
\item Niezależność od stron trzecich --- transakcje odbywają się w ramach zdecentralizowanej sieci Ethereum, gdzie nie występuje żaden administrator czy istytucja będąca w stanie zablokować bądź cofnąć naszą transakcję. Gwarantuje to poczucie bezpieczeństwa i niezależności wyróżniające nas na tle konkurencyjnych serwisów nieopartych na technologii blockchain, gdzie strona trzecia już występuje. Zwiększa to także bezpieczeństwo naszych danych osobowych zachowując większą anonimowość.
\item Bezpieczne wykonanie umów --- ???
\item Różnorodność umów --- sposobów przeprowadzenia transakcji - Przygotowaliśmy 11 przykładowych modeli umów z czego 2 znajdą się w pierwszym wydaniu systemu. Gwarantuje to dostosowanie systemu do potrzeb użytkowników. Dodatkowo ilość ta mogłaby się powiększyć wraz z powiększeniem bazy użytkowników. 
\item Globalny zasięg ofert --- sieć Ethereum wiąże się z brakiem ograniczenia co do lokalizacji na kuli ziemskiej. Korzystać z niej można z dowolnego zakątku, wystarczy jedynie połączenie z internetem. Ether może nabyć każdy co stanowi jedyny wymóg by korzystać z naszej platformy. Daje to o wiele większą wszechstronnoć od większości platform do handlu ograniczonych jedynie do rynku lokalnego. Konkurować z nami pod tym względem mogliby jedynie giganci tacy jak np.\ Amazon czy eBay.
\item Budowanie nowych rozwiązań --- unikalna cecha wyróżniająca nas na rynku. Jestemy otwarci na społeczność tworząca się wokół naszej platformy w znacznie szerszy sposób niż nieblockchainowa konkurencja. Po wypuszczeniu produktu żyłby on własnym życiem kontrolowany w całości przez społeczność posiadającą możliwość wprowadzania modyfikacji, które mogłyby zostać przez większość zaakceptowane bądź odrzucone co stanowi gwarancję dostosowania ECMarket'u do potrzeb potencjalnego użytkownika. Jednakże dodanie funkcjonalności związanych z tą cechą, jest zaplanowane na przyszłe wydania
\end{itemize}

\nocite{website:CyfrowaEkonomia}


\subsection{Rozważane strategie}

Przy określaniu paradygmatu, metodologii i procesu wytwarzania oprogramowania, które zastosujemy w naszym projekcie rozważaliśmy następujące strategie

\begin{itemize}
	\item Ewolucyjno-przyrostowy \\
	Cechy:
	\begin{itemize}
		\item nie wymaga pełnej specyfikacji problemu
		\item można regulować intensywność i czas przyrostów
		\item pozwala na zmianę wymagań i dostosowywanie poprzednich przyrostów
		\item funkcjonalności systemu muszą dawać rozbić się na podfunkcjonalności
		\item proces zajmuje dużo czasu
	\end{itemize}
	\item Scrum \\
	Cechy:
	\begin{itemize}
		\item szybka adaptacja do zmian wymagań
		\item wymagana stała komunikacja w zespole
		\item nieduże przyrosty w małych odstępach
		\item wymaga podobnych umiejętności wszystkich członków zespołu
		\item samozarządzanie się zespołu
		\item szybko tworzone oprogramowanie
	\end{itemize}
	\item Prototypowanie \\
	Cechy:
	\begin{itemize}
		\item wspomaga identyfikację wymagań
		\item może przeszkodzić w kompleksowej analizie problemu
		\item duże nakłady czasu na przygotowanie prototypu
		\item logiczna walidacja problemu
		\item najlepiej się sprawdza przy budowaniu UI
	\end{itemize}
\end{itemize}

\nocite{academy:BYT, wiki:Scrum, wiki:SoftPrototyping}

\subsection{Wybór strategii i argumentacja}

Metodologią, którą rozważaliśmy na początku był Scrum, jednak wymóg stałej komunikacji i intensywności pracy, zdyskwalifikował ją dość szybko. Problemem również była bardzo mocno zróżnicowana wśród zespołu znajomość technologii (Ethereum). Wahaliśmy się, czy nie użyć protypowania, jednakże mieliśmy już na samym początku, w pewnym stopniu określone wymagania, więc zdecydowaliśmy się na bardziej znaną nam metodę ewolucyjno-przyrostową.

\subsection{Plan zapewnienia jakości}

Jakość w naszym produkcie jest zapewniona dzięki wykorzystaniu metodyki zwinnej TDD (Test Driven Development). Zdecydowaliśmy się na taką metodykę, ponieważ nasz system operuje na kryptowalutach oraz na dobrach każdego użytkownika. Sama idea zdecentralizowanej platformy handlowej jest mało popularnym zagadnieniem, więc aby zapewnić użytkowników do zaufania nowej technologii i platformy, musimy dostarczyć wysoką jakość naszego produktu.

\subsection{Udziałowcy}


\begin{table}[H]
\begin{tabular}{|l|m{11cm}|}
\hline
\multicolumn{2}{|l|}{\textit{\textbf{Karta udziałowca}}}                               \\ \hline
\textit{Identyfikator:}  & \textit{UNP 01}                                             \\ \hline
\textit{Nazwa:}          & \textit{PJATK}                                              \\ \hline
\textit{Opis:}           & \textit{PJATK umożliwiło nam wykonanie pracy inżynierskiej} \\ \hline
\textit{Typ udziałowca:} & \textit{Nieożywiony pośredni}                               \\ \hline
\textit{Punkt widzenia:} & \textit{Ekonomicznej}                                       \\ \hline
\textit{Ograniczenia:}   & \textit{PJATK nie może narzucać rozwiązań systemowych}      \\ \hline
\textit{Wymagania:}      & \textit{}                                                   \\ \hline
\end{tabular}
\end{table}

\begin{table}[H]
\begin{tabular}{|l|m{11cm}|}
\hline
\multicolumn{2}{|l|}{\textit{\textbf{Karta udziałowca}}}                               \\ \hline
\textit{Identyfikator:}  & \textit{UNB 02}                                             \\ \hline
\textit{Nazwa:}          & \textit{Sieć Ethereum}                                              \\ \hline
\textit{Opis:}           & \textit{Zdecentralizowana sieć która pozwala nam wykonywać smart contracty.} \\ \hline
\textit{Typ udziałowca:} & \textit{Nieożywiony bezpośredni}                               \\ \hline
\textit{Punkt widzenia:} & \textit{ekonomiczny, techniczny}                                       \\ \hline
\textit{Ograniczenia:}   & \textit{brak}      \\ \hline
\textit{Wymagania:}      & \textit{}                                                   \\ \hline
\end{tabular}
\end{table}

\begin{table}[H]
\begin{tabular}{|l|m{11cm}|}
\hline
\multicolumn{2}{|l|}{\textit{\textbf{Karta udziałowca}}}                               \\ \hline
\textit{Identyfikator:}  & \textit{UOB 01}                                             \\ \hline
\textit{Nazwa:}          & \textit{Zespół Projektowy}                                              \\ \hline
\textit{Opis:}           & \textit{Członkowie zespołu wykonującego system ECMarket.} \\ \hline
\textit{Typ udziałowca:} & \textit{Ożywiony bezpośredni}                               \\ \hline
\textit{Punkt widzenia:} & \textit{techniczny, wykonawczy}                                       \\ \hline
\textit{Ograniczenia:}   & \textit{brak}      \\ \hline
\textit{Wymagania:}      & \textit{}                                                   \\ \hline
\end{tabular}
\end{table}

\begin{table}[H]
\begin{tabular}{|l|m{11cm}|}
\hline
\multicolumn{2}{|l|}{\textit{\textbf{Karta udziałowca}}}                               \\ \hline
\textit{Identyfikator:}  & \textit{UOB 02}                                             \\ \hline
\textit{Nazwa:}          & \textit{Klienci}                                              \\ \hline
\textit{Opis:}           & \textit{Grupa która dzięki systemowi ECMarket kupuje i sprzedaje dobra.} \\ \hline
\textit{Typ udziałowca:} & \textit{Ożywiony bezpośredni}                               \\ \hline
\textit{Punkt widzenia:} & \textit{Operator systemu}                                       \\ \hline
\textit{Ograniczenia:}   & \textit{Klienci nie mogą narzucać wymagań systemowych.}      \\ \hline
\textit{Wymagania:}      & \textit{}                                                   \\ \hline
\end{tabular}
\end{table}

\begin{table}[H]
\begin{tabular}{|l|m{11cm}|}
\hline
\multicolumn{2}{|l|}{\textit{\textbf{Karta udziałowca}}}                               \\ \hline
\textit{Identyfikator:}  & \textit{UNP 02}                                             \\ \hline
\textit{Nazwa:}          & \textit{Regulacje prawne}                                              \\ \hline
\textit{Opis:}           & \textit{Prawo ma duży wpływ na możliwości naszego systemu.} \\ \hline
\textit{Typ udziałowca:} & \textit{Nieożywiony pośredni}                               \\ \hline
\textit{Punkt widzenia:} & \textit{prawny}                                       \\ \hline
\textit{Ograniczenia:}   & \textit{}      \\ \hline
\textit{Wymagania:}      & \textit{}                                                   \\ \hline
\end{tabular}
\end{table}


\subsection{Ograniczenia}

Głównym ograniczeniem jest czas na wykonanie projektu, który wynosi niepełny rok.
Każda instrukcja wykonana na Wirtualnej Maszynie Ethereum (EVM), kosztuje pewną ilość etheru, co stanowi ograniczenie jak bardzo skomplikowany kod może być na niej wykonywany. Ten problem można częściowo rozwiązać delegując wykonanie niektórych zadań do klientów, takich, które nie wymagają wiarygodności lub takie, dla których można skonstruować dowód poprawności obliczeń, którego weryfikacja jest tańsza niż sam kod.

Pewnym ograniczeniem jest fakt, że nie opieramy się na żadnej stronie trzeciej, która byłaby w stanie stanowić autorytet i potwierdzać, że jakieś wydarzenie miało miejsce lub nie miało (np.\ dostarczenie produktu, który jednocześnie spełniałby pewne wymagania klienta lub innych stron). Te gwarancje próbujemy zbudować na podłożu teorii gier oraz społecznej teorii gier, by doprowadzić do sytuacji, gdzie strony nie wywiązujące się z umów tracą pieniądze lub/i ocenę, więc efektywnie zniechęcane do takich działań.

Istotnym problemem, który chcemy przynajmniej od Siebie odsunąć są kwestie prawne, a więc rosnąca kontrola władzy państwowej nad różnymi aspektami życia. Chcielibyśmy rozwiązać ten problem po przez wydanie platformy na licencji z rodziny GPL, jak również zbudować w ramach naszej platformy system repozytoriów zarządzanych przez klientów. Jednak realizacja tego drugiego odbędzie się nie wcześniej, niż w drugim wydaniu aplikacji.



\subsection{Zagrożenia}

\subsection{Harmonogram prac}

\textbf{Przyrost Pierwszy}
\renewcommand{\labelenumii}{\textendash}
\begin{enumerate}
	\item analiza potencjalnych umów do zrealizowania w naszym systemie:
	\begin{enumerate}
		\item analiza umowy A1.1
		\item analiza umowy A1.2
		\item analiza umowy A1.3
		\item analiza umowy A2.1
		\item analiza umowy A2.2
		\item analiza umowy A3.1
	\end{enumerate}
	\item pierwsza analiza systemu ocen --- szkic własności
	\item propozycja pierwszej architektury systemu
	\item stworzenie wstępnego harmonogramu 
	\item rozważanie strategii
	\item dokument założeń wstępnych
	\item pierwsza specyfikacja wymagań systemowych
	\item wstępny diagram przypadków użycia
	\item nauka obsługi ekosystemu Ethereum
\end{enumerate}

\vspace{6mm}

\textbf{Przyrost drugi}
\renewcommand{\labelenumii}{\textendash}
	\begin{enumerate}
        \item Implementacja umowy A1.1
        \item Implementacja wewnętrznych tokenów
        \item Analiza metody zachęcania użytkowników do usuwania zakończonych / przeterminowanych umów
        \item Implementacja AgreementManagera
        \item Analiza scenariuszy umów:
        \begin{enumerate}
            \item analiza umowy A3.2
            \item analiza umowy A3.3
            \item analiza umowy A3.4
            \item analiza umowy A4.2
            \item analiza umowy A4.3
            \item analiza umowy PA1
            \item analiza umowy PA2
            \item analiza umowy WV1
            \item analiza umowy WV2
        \end{enumerate}
\end{enumerate}
%\newline aby oddzielic przyrost 2 i 3

\textbf{Przyrost trzeci}
\renewcommand{\labelenumii}{\textendash}
\begin{enumerate}
    \item Problem z rozbudową umów - migracja do maszyn stanowych
    \item Migracja tokenów systemowych do formatu zgodnego z ERC20
    \item Implementacja serwisu do wyszukiwania umów:
    \item Implementacja widoków dla systemu:
    \begin{enumerate}    
        \item Prototyp widoku do wyszukiwania umów
        \item Szczegółowy widoku umów
        \item Widok do przeglądania akcji danego adresu portfela
        \item Widok do tworzenia umów
    \end{enumerate}

    \item Rozbudowany routing komponentów
    \item Interakcje z umowami
    \item Pełna implementacja interfejsu użytkownika do wyszukiwania umów
\end{enumerate}

\textbf{Przyrost czwarty}
\renewcommand{\labelenumii}{\textendash}
\begin{enumerate}
    \item Implementacja umowy A1.2
    \item Implementacja zdarzeń dla umów
    \item Wdrożenie systemu ocen
    \item Ujednolicenie interfejsu użytkownika
\end{enumerate}

\subsection{Przebieg etapów pracy}

\subsection{Charakterystyka zespołu}

\subsection{Infrastruktura komunikacyjna i dokumentacyjna}

W ramach infrastruktury komunikacyjnej wykorzysztaliśmy platformę https://www.discord.com, za pomocą której występowała komunikacja między członkami zespołami, ponieważ większość pracy została wykonywana zdalnie i był to nasz główny kanał komunikacji.  Do integracji kodu projektu oraz dokumentacji pomiędzy członkami grupy projektowej wykorzystaliśmy system kontroli wersji https://www.github.com.

\subsection{Aspekty społeczne}
Charakterystyczną cechą naszej platfmormy jest niezmienność początkowych warunków użytkowania bez zgody społeczności. Kontrastuje to z coraz częstszą polityką dużych firm na rynku zmieniających nieustannie regulamin tak by móc np. wykorzystywać nasze dane w większym zakresie niż często byśmy chcieli, aczkolwiek z powodu nieświadomości, lenistwa czy też potrzeby korzystania z danego rozwiązania na takowe zmiany się godzimy. Odpowiedź na tego typu politykę stanowi charakterytyka działania Ethereum na której opiera się nasza platfmorma - przy jakiejkolwiek zmianie w kodzie, który stanowi w pewnym rozumieniu odpowiednik regulaminu np. przetwarzania naszych danych musimy wyrazić zgodę na to by korzystać z "nowszej" wersji. W przeciwnym wypadku posiadamy możliwość zrobienia fork'a i korzystania z wersji, która nam odpowiadała tak jak i cała społeczność. Sprawia to że przy odpowiedniej świadomości społeczności nieosiągalne są praktyki takie jak np. sprzedaż danych osobowych do firm zewnętrznych w celach marketingowych. Społeczność skupiona wokół platformy nie wyrazi na to zgody i korzystać będzie ze starszej wersji przed zmianami w kodzie. Zdecentralizowana specyfika platfmormy uniemożliwia brzydkie praktyki stosowane przez serwisy aukcyjne - śledzenie trendów w sprzedaży i wypuszczanie własnych linii produktów okradając innych z własności intelektualnej w zupełnie legalny sposób promując swój produkt na własnej platformie co nie jest zbyt trudne i w ten sposób odbierając potencjalne zyski osobie która oryginalnie wpadła na wyjście z nowatorską inicjatywą. W przypadku naszej platfmormy coś takiego jest nie do pomyślenia przez uczciwość pierwszego wydania jak i zdecentralizowaną strukturę uniemożliwiającą ingerencję "góry" w to co dzieje się w serwisie a tym samym promowanie własnego produktu na lepszych warunkach niż konkurencji. Jakiekolwiek zmiany w kodzie wpływające na naszą korzyść musiałyby zostać zaakceptowane przez społeczność, a co za tym idzie byłyby niemal nieosiągalne. Wszystko to składa się na zupełnie nową jakość w relacjach użytkownik serwisu - jego twóry. Brak nieuczciwych bądź nieetycznych praktyk, których wiele serwisów w swoim początkowym stadium także nie stosowało. Różnica polega na tym, że tu ich nie będzie tam się pojawiły z czasem.

\newpage
\section{Analiza}

\subsection{Wymagania systemowe}
\subsubsection{Wymagania ogólne i dziedzinowe}

\begin{table}[H]
\begin{tabular}{|l|l|l|l|}
\hline
\multicolumn{4}{|l|}{\textit{\textbf{Karta Wymagania}}}                                                           \\ \hline
\textit{Identyfikator:}        & \textit{WO1}            & \textit{\textbf{Priorytet:}}            & M            \\ \hline
\textit{Nazwa:}                & \multicolumn{3}{l|}{\textit{System ECMarket}}                                    \\ \hline
\textit{Opis:}                 & \multicolumn{3}{m{11cm}|}{\textit{Napisanie kodu dla ECMarket i wypuszczenie go do mainnetu}} \\ \hline
\textit{Wymagania powiązane :} & \multicolumn{3}{l|}{\textit{brak}}                                                   \\ \hline
\end{tabular}
\end{table}

\begin{table}[H]
\begin{tabular}{|l|l|l|l|}
\hline
\multicolumn{4}{|l|}{\textit{\textbf{Karta Wymagania}}}                                                           \\ \hline
\textit{Identyfikator:}        & \textit{WO2}            & \textit{\textbf{Priorytet:}}            & S            \\ \hline
\textit{Nazwa:}                & \multicolumn{3}{l|}{\textit{Podstawowa aplikacja kliencka}}                                    \\ \hline
\textit{Opis:}                 & \multicolumn{3}{m{11cm}|}{\textit{Przygotowanie podstawowej aplikacji klienckiej zapewniający graficzny interfejs do ECMarket}} \\ \hline
\textit{Wymagania powiązane :} & \multicolumn{3}{l|}{\textit{brak}}                                                   \\ \hline
\end{tabular}
\end{table}

\subsubsection{Wymagania funkcjonalne}

\begin{table}[H]
    \begin{tabular}{|l|l|l|l|}
    \hline
    \multicolumn{4}{|l|}{\textit{\textbf{Karta Wymagania}}}                                                           \\ \hline
    \textit{Identyfikator:}        & \textit{F01}            & \textit{\textbf{Priorytet:}}            & M            \\ \hline
    \textit{Nazwa:}                & \multicolumn{3}{l|}{\textit{Dostęp do podstawowych pól umowy.}}                                    \\ \hline
    \textit{Opis:}                 & \multicolumn{3}{m{11cm}|}{\textit{Każda umowa stworzona w systemie powinna dawać każdemu dostęp do takich podstawowych pól jak: lista uczestników umowy, numer bloku utworzenia, timestamp stworzenia, status umowy}} \\ \hline
    \textit{Kryteria akceptacji :} & \multicolumn{3}{l|}{\textit{Umowy posiadają metody zwracające żądane pola}}     \\ \hline
    \textit{Dane wejściowe :} & \multicolumn{3}{l|}{\textit{brak}}                                                \\ \hline
    \textit{Warunki początkowe :} & \multicolumn{3}{l|}{\textit{Umowa musi istnieć}}                                                \\ \hline
    \textit{Warunki końcowe :} & \multicolumn{3}{l|}{\textit{strona wywołująca otrzymuje żądane pole}}                                                \\ \hline
\end{tabular}
\end{table}

\begin{table}[H]
    \begin{tabular}{|l|l|l|l|}
    \hline
    \multicolumn{4}{|l|}{\textit{\textbf{Karta Wymagania}}}                                                           \\ \hline
    \textit{Identyfikator:}        & \textit{F02}            & \textit{\textbf{Priorytet:}}            & M            \\ \hline
    \textit{Nazwa:}                & \multicolumn{3}{l|}{\textit{Możliwość wpłacania, wypłacania i sprawdzenia bilansu etheru}}                                    \\ \hline
    \textit{Opis:}                 & \multicolumn{3}{m{11cm}|}{\textit{Każdy użytkownik sieci ma możliwość wpłacenia dowolnej ilość Etheru do “Virtual Wallet”-u oraz wpłacenia dowolnej ilości etheru, nie większej niż bilans danego użytkownika . Zmiana bilansu danego użytkownika, pokrywa się jeden-do-jednego z ilością wpłaconego lub wypłaconego przez niego etheru. Jeśli użytkownik nie brał udziału w umowie, ani nie wpłacał etheru jego bilans wynosi 0.  }} \\ \hline
    \textit{Kryteria akceptacji :} & \multicolumn{3}{m{11cm}|}{\textit{Każdy użytkownik może niezależnie od innych wpłacać lub wypłacać ether. Jeśli użytkownik wpłacił pewne x i nie brał udziału w żadnej umowie, może maksymalnie wypłacić x.}}    \\ \hline
    \textit{Dane wejściowe :} & \multicolumn{3}{l|}{\textit{brak}}                                                \\ \hline
    \textit{Warunki początkowe :} & \multicolumn{3}{l|}{\textit{posiadanie etheru}}                                                \\ \hline
    \textit{Warunki końcowe :} & \multicolumn{3}{l|}{\textit{zmiana bilansu}}                                                \\ \hline
    \textit{Sytuacje wyjątkowe :} & \multicolumn{3}{m{11cm}|}{\textit{Jeśli użytkownik żąda wypłaty większej ilości etheru niż posiada go w VirtualWallet, zostaje wyrzucony wyjątek.}}                                                \\ \hline
\end{tabular}
\end{table}

\begin{table}[H]
    \begin{tabular}{|l|l|l|l|}
    \hline
    \multicolumn{4}{|l|}{\textit{\textbf{Karta Wymagania}}}                                                           \\ \hline
    \textit{Identyfikator:}        & \textit{F03}            & \textit{\textbf{Priorytet:}}            & M            \\ \hline
    \textit{Nazwa:}                & \multicolumn{3}{l|}{\textit{Możliwość stworzenia umowy}}                                    \\ \hline
    \textit{Opis:}                 & \multicolumn{3}{m{11cm}|}{\textit{AgreementManager pozwala na stworzenie umowy z parametrami i typem określonym przez użytkownika. Użytkownik otrzymuje adres stworzonej umowy jako wartość zwrotną z funkcji lub jako wydarzenie o nazwie AgreementCreation z dodatkowym polem zawierającym adres utworzonej umowy. Poprawnie utworzona umowa zostaje zarejestrowana przez AgreementManagera.}} \\ \hline
    \textit{Kryteria akceptacji :} & \multicolumn{3}{m{11cm}|}{\textit{AgreementManager przy poprawnych danych tworzy poprawne umowy. Wydarzenie powinno być wygenerowane, tylko wtedy, gdy tworzenie umowy się powiodło.}}    \\ \hline
    \textit{Dane wejściowe :} & \multicolumn{3}{l|}{\textit{brak}}                                                \\ \hline
    \textit{Warunki początkowe :} & \multicolumn{3}{l|}{\textit{brak}}                                                \\ \hline
    \textit{Warunki końcowe :} & \multicolumn{3}{m{11cm}|}{\textit{utworzenie nowej umowy i możliwość zarządzania nią z pomocą innych funkcji AgreementManagera}}                                                \\ \hline
\end{tabular}
\end{table}

\begin{table}[H]
    \begin{tabular}{|l|l|l|l|}
    \hline
    \multicolumn{4}{|l|}{\textit{\textbf{Karta Wymagania}}}                                                           \\ \hline
    \textit{Identyfikator:}        & \textit{F04}            & \textit{\textbf{Priorytet:}}            & M            \\ \hline
    \textit{Nazwa:}                & \multicolumn{3}{l|}{\textit{Możliwość wyszukania umowy}}                                    \\ \hline
    \textit{Opis:}                 & \multicolumn{3}{m{11cm}|}{\textit{AgreementManager powinien mieć możliwość zwrócenia strony z ostatnio stworzonymi umowami.}} \\ \hline
    \textit{Kryteria akceptacji :} & \multicolumn{3}{m{11cm}|}{\textit{AgreementManager przy poprawnych danych tworzy poprawne umowy. Wydarzenie powinno być wygenerowane, tylko wtedy, gdy tworzenie umowy się powiodło.}}    \\ \hline
    \textit{Dane wejściowe :} & \multicolumn{3}{l|}{\textit{brak}}                                                \\ \hline
    \textit{Warunki początkowe :} & \multicolumn{3}{l|}{\textit{AgreementManager musi posiadać zarejestrowane umowy}}                                                \\ \hline
    \textit{Warunki końcowe :} & \multicolumn{3}{m{11cm}|}{\textit{utworzenie nowej umowy i możliwość zarządzania nią z pomocą innych funkcji AgreementManagera}}                                                \\ \hline
    \textit{Sytuacje wyjątkowe :} & \multicolumn{3}{m{11cm}|}{\textit{W przypadku braku jakichkolwiek umów zostaje zwrócona tablica z samymi zerami}} \\ \hline
\end{tabular}
\end{table}

\begin{table}[H]
    \begin{tabular}{|l|l|l|l|}
    \hline
    \multicolumn{4}{|l|}{\textit{\textbf{Karta Wymagania}}}                                                           \\ \hline
    \textit{Identyfikator:}        & \textit{F05}            & \textit{\textbf{Priorytet:}}            & M            \\ \hline
    \textit{Nazwa:}                & \multicolumn{3}{l|}{\textit{Możliwość usunięcia nowo-stworzonej umowy A1.1.}}                                    \\ \hline
    \textit{Opis:}                 & \multicolumn{3}{m{11cm}|}{\textit{Umowy stworzone przez użytkowników mogą być przez nich usuwane}} \\ \hline
    \textit{Kryteria akceptacji :} & \multicolumn{3}{m{11cm}|}{\textit{Tylko wskazana umowa ulega usunięciu}}    \\ \hline
    \textit{Dane wejściowe :} & \multicolumn{3}{l|}{\textit{brak}}                                                \\ \hline
    \textit{Warunki początkowe :} & \multicolumn{3}{m{11cm}|}{\textit{Umowa jest w stanie New, ważna (nie przekroczyła daty ważności), a stroną wywołująca jest jest twórca}}                                                \\ \hline
    \textit{Warunki końcowe :} & \multicolumn{3}{m{11cm}|}{\textit{Kontrakt umowy zostaje zniszczony, wyrejestrowany z AgreementManagera, a środki wpłacone przez petentów zostają do nich odesłane}}                                                \\ \hline
    \textit{Sytuacje wyjątkowe :} & \multicolumn{3}{m{11cm}|}{\textit{brak}} \\ \hline
\end{tabular}
\end{table}

\begin{table}[H]
    \begin{tabular}{|l|l|l|l|}
    \hline
    \multicolumn{4}{|l|}{\textit{\textbf{Karta Wymagania}}}                                                           \\ \hline
    \textit{Identyfikator:}        & \textit{F06}            & \textit{\textbf{Priorytet:}}            & M            \\ \hline
    \textit{Nazwa:}                & \multicolumn{3}{l|}{\textit{Możliwość usunięcia przeterminowanej umowy A1.1.}}                                    \\ \hline
    \textit{Opis:}                 & \multicolumn{3}{m{11cm}|}{\textit{Przeterminowane umowy mogą być usunięte przez każdego.}} \\ \hline
    \textit{Kryteria akceptacji :} & \multicolumn{3}{m{11cm}|}{\textit{Tylko wskazana umowa ulega usunięciu}}    \\ \hline
    \textit{Dane wejściowe :} & \multicolumn{3}{l|}{\textit{brak}}                                                \\ \hline
    \textit{Warunki początkowe :} & \multicolumn{3}{l|}{\textit{umowa przekroczyła swoją datę ważności}}                                                \\ \hline
    \textit{Warunki końcowe :} & \multicolumn{3}{m{11cm}|}{\textit{Kontrakt umowy zostaje zniszczony, wyrejestrowany z AgreementManagera, a środki wpłacone przez petentów zostają do nich odesłane}}                                                \\ \hline
    \textit{Sytuacje wyjątkowe :} & \multicolumn{3}{m{11cm}|}{\textit{brak}} \\ \hline
\end{tabular}
\end{table}

\begin{table}[H]
    \begin{tabular}{|l|l|l|l|}
    \hline
    \multicolumn{4}{|l|}{\textit{\textbf{Karta Wymagania}}}                                                           \\ \hline
    \textit{Identyfikator:}        & \textit{F07}            & \textit{\textbf{Priorytet:}}            & M            \\ \hline
    \textit{Nazwa:}                & \multicolumn{3}{l|}{\textit{Możliwość usunięcia zakończonej umowy A1.1.}}                                    \\ \hline
    \textit{Opis:}                 & \multicolumn{3}{m{11cm}|}{\textit{Umowy zakończone mogą być usuwane przez użytkowników}} \\ \hline
    \textit{Kryteria akceptacji :} & \multicolumn{3}{m{11cm}|}{\textit{Tylko wskazana umowa ulega usunięciu}}    \\ \hline
    \textit{Dane wejściowe :} & \multicolumn{3}{l|}{\textit{brak}}                                                \\ \hline
    \textit{Warunki początkowe :} & \multicolumn{3}{m{11cm}|}{\textit{Umowa jest w stanie Done, ważna (nie przekroczyła daty ważności), a stroną wywołująca jest jest twórca}}                                                \\ \hline
    \textit{Warunki końcowe :} & \multicolumn{3}{m{11cm}|}{\textit{Kontrakt umowy zostaje zniszczony, wyrejestrowany z AgreementManagera, a środki wpłacone przez petentów zostają do nich odesłane}}                                                \\ \hline
    \textit{Sytuacje wyjątkowe :} & \multicolumn{3}{m{11cm}|}{\textit{brak}} \\ \hline
\end{tabular}
\end{table}

\begin{table}[H]
    \begin{tabular}{|l|l|l|l|}
    \hline
    \multicolumn{4}{|l|}{\textit{\textbf{Karta Wymagania}}}                                                           \\ \hline
    \textit{Identyfikator:}        & \textit{F08}            & \textit{\textbf{Priorytet:}}            & M            \\ \hline
    \textit{Nazwa:}                & \multicolumn{3}{l|}{\textit{Możliwość oceniania stron kontraktu}}                                    \\ \hline
    \textit{Opis:}                 & \multicolumn{3}{m{11cm}|}{\textit{Obie strony mają możliwość ocenienia drugiej strony przed zakończeniem kontraktu}} \\ \hline
    \textit{Kryteria akceptacji :} & \multicolumn{3}{m{11cm}|}{\textit{Warunki Satysfakcji (Szczegóły dodane na potrzeby  testów akceptacyjnych)}}    \\ \hline
    \textit{Dane wejściowe :} & \multicolumn{3}{l|}{\textit{Kontrakt został zapoczątkowany}}                                                \\ \hline
    \textit{Warunki początkowe :} & \multicolumn{3}{m{11cm}|}{\textit{Umowa jest w stanie Done, ważna (nie przekroczyła daty ważności), a stroną wywołująca jest jest twórca}}                                                \\ \hline
    \textit{Warunki końcowe :} & \multicolumn{3}{m{11cm}|}{\textit{Kontrakt umowy zostaje zniszczony, wyrejestrowany z AgreementManagera, a środki wpłacone przez petentów zostają do nich odesłane}}                                                \\ \hline
    \textit{Sytuacje wyjątkowe :} & \multicolumn{3}{m{11cm}|}{\textit{brak}} \\ \hline
\end{tabular}
\end{table}

\begin{table}[H]
    \begin{tabular}{|l|l|l|l|}
    \hline
    \multicolumn{4}{|l|}{\textit{\textbf{Karta Wymagania}}}                                                           \\ \hline
    \textit{Identyfikator:}        & \textit{F09}            & \textit{\textbf{Priorytet:}}            & M            \\ \hline
    \textit{Nazwa:}                & \multicolumn{3}{l|}{\textit{Wykonanie umowy A1.1}}                                    \\ \hline
    \textit{Opis:}                 & \multicolumn{3}{m{11cm}|}{\textit{Umowa A1.1 posiada funkcje join, accept i conclude. Strony chętne do uczestnictwa w umowie wykonują join jednocześnie przekazując środki określone w polu cena na konto umowy. Właściciel umowy może wykonać accept wskazując z kim chce przeprowadzić transakcję, jednocześnie pobierając środki z konta umowy. Po tym fakcie obie strony mogą się ocenić z pomocą conclude}} \\ \hline
    \textit{Kryteria akceptacji :} & \multicolumn{3}{m{11cm}|}{\textit{AgreementManager zarejestrował wykonanie umowy}}    \\ \hline
    \textit{Dane wejściowe :} & \multicolumn{3}{l|}{\textit{Kontrakt został zapoczątkowany}}                                                \\ \hline
    \textit{Warunki początkowe :} & \multicolumn{3}{l|}{\textit{brak}}                                                \\ \hline
    \textit{Warunki końcowe :} & \multicolumn{3}{m{11cm}|}{\textit{brak}}                                                \\ \hline
    \textit{Sytuacje wyjątkowe :} & \multicolumn{3}{m{11cm}|}{\textit{brak}} \\ \hline
\end{tabular}
\end{table}

\begin{table}[H]
    \begin{tabular}{|l|l|l|l|}
    \hline
    \multicolumn{4}{|l|}{\textit{\textbf{Karta Wymagania}}}                                                           \\ \hline
    \textit{Identyfikator:}        & \textit{F10}            & \textit{\textbf{Priorytet:}}            & M            \\ \hline
    \textit{Nazwa:}                & \multicolumn{3}{l|}{\textit{Możliwość wypłacania środków z kontraktu}}                                    \\ \hline
    \textit{Opis:}                 & \multicolumn{3}{m{11cm}|}{\textit{Petent powinien mieć możliwość wypłacenia środków.}} \\ \hline
    \textit{Kryteria akceptacji :} & \multicolumn{3}{m{11cm}|}{\textit{brak}}    \\ \hline
    \textit{Dane wejściowe :} & \multicolumn{3}{l|}{\textit{brak}}                                                \\ \hline
    \textit{Warunki początkowe :} & \multicolumn{3}{l|}{\textit{Wpłacenie środków do kontraktu.}}                                                \\ \hline
    \textit{Warunki końcowe :} & \multicolumn{3}{m{11cm}|}{\textit{brak}}                                                \\ \hline
    \textit{Sytuacje wyjątkowe :} & \multicolumn{3}{m{11cm}|}{\textit{brak}} \\ \hline
\end{tabular}
\end{table}

\begin{table}[H]
    \begin{tabular}{|l|l|l|l|}
    \hline
    \multicolumn{4}{|l|}{\textit{\textbf{Karta Wymagania}}}                                                           \\ \hline
    \textit{Identyfikator:}        & \textit{F11}            & \textit{\textbf{Priorytet:}}            & M            \\ \hline
    \textit{Nazwa:}                & \multicolumn{3}{l|}{\textit{Ustawianie czasu ważności umowy przy tworzeniu}}                                    \\ \hline
    \textit{Opis:}                 & \multicolumn{3}{m{11cm}|}{\textit{Użytkownik ma możliwość określenia ile czasu/bloków dana umowa będzie ważna. Przedział dozwolonych czasów ważności jest ograniczony od dołu i od góry i zależny od typu umowy. Po utworzeniu umowy nie jest możliwa modyfikacja tej wartości}} \\ \hline
    \textit{Kryteria akceptacji :} & \multicolumn{3}{m{11cm}|}{\textit{brak}}    \\ \hline
    \textit{Dane wejściowe :} & \multicolumn{3}{l|}{\textit{brak}}                                                \\ \hline
    \textit{Warunki początkowe :} & \multicolumn{3}{l|}{\textit{brak}}                                                \\ \hline
    \textit{Warunki końcowe :} & \multicolumn{3}{m{11cm}|}{\textit{brak}}                                                \\ \hline
    \textit{Sytuacje wyjątkowe :} & \multicolumn{3}{m{11cm}|}{\textit{brak}} \\ \hline
\end{tabular}
\end{table}

\subsubsection{Wymagania niefunkcjonalne}

\begin{table}[H]
    \begin{tabular}{|l|l|l|l|}
    \hline
    \multicolumn{4}{|l|}{\textit{\textbf{Karta Wymagania}}}                                                           \\ \hline
    \textit{Identyfikator:}        & \textit{NF01}            & \textit{\textbf{Priorytet:}}            & S            \\ \hline
    \textit{Nazwa:}                & \multicolumn{3}{l|}{\textit{System zachęty do usuwania przeterminowanych umów}}                                    \\ \hline
    \textit{Opis:}                 & \multicolumn{3}{m{11cm}|}{\textit{Stworzenie umowy wymaga wpłacenia niedużej kaucji (trzeba pomyśleć jak “niedużej”), która jest wypłacana temu, który usunie umowę z systemu. Jeśli umowa jest ważna, tylko osoba uprawniona (zazwyczaj twórca umowy) może ją usunąć i otrzymać z powrotem kaucję. Jeśli umowa jest przeterminowana osoba uprawniona ma wyłączność na usunięcie i zagarnięcie kaucji, ale tylko przez pewien okres ochronny (jak długi) po dacie utraty ważności. Po okresie ochronnym każdy może usunąć umowę i otrzymać daną kaucję}} \\ \hline
    \textit{Kryteria akceptacji :} & \multicolumn{3}{m{11cm}|}{\textit{brak}}    \\ \hline

\end{tabular}
\end{table}

\begin{table}[H]
    \begin{tabular}{|l|l|l|l|}
    \hline
    \multicolumn{4}{|l|}{\textit{\textbf{Karta Wymagania}}}                                                           \\ \hline
    \textit{Identyfikator:}        & \textit{NF02}            & \textit{\textbf{Priorytet:}}            & M            \\ \hline
    \textit{Nazwa:}                & \multicolumn{3}{m{11cm}|}{\textit{System ocen - efekty wystawienia pozytywnej / neutralnej / negatywnej oceny}}                                    \\ \hline
    \textit{Opis:}                 & \multicolumn{3}{m{11cm}|}{\textit{Jeśli strona P1 umowy A wystawia pewną pozytywną ocenę r1 stronie P2, wtedy wiarygodność P2 w2 wzrasta do w2’ według pewnej miary D, tak że D(w2) < D(w2’).
    Jeśli strona P1 umowy A wystawia pewną neutralną ocenę O stronie P2, wtedy wiarygodność P2 w2 nie ulega zmianie.
    Jeśli strona P1 umowy A wystawia pewną negatywną ocenę r1 stronie P2, wtedy wiarygodność P2 w2 spada do w2’ według pewnej miary D, tak że D(w2) > D(w2’).}} \\ \hline
    \textit{Kryteria akceptacji :} & \multicolumn{3}{m{11cm}|}{\textit{brak}}    \\ \hline

\end{tabular}
\end{table}

\begin{table}[H]
    \begin{tabular}{|l|l|l|l|}
    \hline
    \multicolumn{4}{|l|}{\textit{\textbf{Karta Wymagania}}}                                                           \\ \hline
    \textit{Identyfikator:}        & \textit{NF03}            & \textit{\textbf{Priorytet:}}            & M            \\ \hline
    \textit{Nazwa:}                & \multicolumn{3}{l|}{\textit{System ocen - tłumienie}}                                    \\ \hline
    \textit{Opis:}                 & \multicolumn{3}{m{11cm}|}{\textit{Strony P1 i P2 biorą udział w zbiorze umów {A1, A2, …, An}. P1 w kolejnych umowach A1, A2, ..., An wystawia P2 oceny r1, r2, …,rn, co daje ciąg wiarygodności dla P2 w1, w2, …, wn. Wtedy każdy z tych elementów tego ciągu spełnia zależność dR kNi>k Dwi+1-Dwi<i-d.
    To oznacza, że od pewnego momentu (jakiego) waga ocen wystawianych przez P1 dla P2 zaczyna co najmniej wykładniczo spadać.}} \\ \hline
    \textit{Kryteria akceptacji :} & \multicolumn{3}{m{11cm}|}{\textit{brak}}    \\ \hline

\end{tabular}
\end{table}

\begin{table}[H]
    \begin{tabular}{|l|l|l|l|}
    \hline
    \multicolumn{4}{|l|}{\textit{\textbf{Karta Wymagania}}}                                                           \\ \hline
    \textit{Identyfikator:}        & \textit{NF04}            & \textit{\textbf{Priorytet:}}            & M            \\ \hline
    \textit{Nazwa:}                & \multicolumn{3}{l|}{\textit{System ocen - istotność wiarygodności partnera}}                                    \\ \hline
    \textit{Opis:}                 & \multicolumn{3}{m{11cm}|}{\textit{Strony P1, P2, P3, P4 mają wiarygodności w1, w2, w3, w4 gdzie D(w1) < D(w2) i D(w3) = D(w4). Wtedy, jeśli P1 i P3 biorą udział w umowie tego samego typu co P2 i P4 i P1 wystawi P3 ocenę r1, a P2 wystawi P4 r2, takie że r1 = r2, to D(w3)-D(w3') < D(w4)-D(w4').
    Z tego wynika, że oceny wystawiane przez stronę o większej wiarygodności, mają większą wagę niż oceny strony o niższej
    }} \\ \hline
    \textit{Kryteria akceptacji :} & \multicolumn{3}{m{11cm}|}{\textit{brak}}    \\ \hline

\end{tabular}
\end{table}


\subsubsection{Wymagania projektowo-wdrożeniowe}

\begin{table}[H]
    \begin{tabular}{|l|l|l|l|}
    \hline
    \multicolumn{4}{|l|}{\textit{\textbf{Karta Wymagania}}}                                                           \\ \hline
    \textit{Identyfikator:}        & \textit{ŚD01}            & \textit{\textbf{Priorytet:}}            & M            \\ \hline
    \textit{Nazwa:}                & \multicolumn{3}{l|}{\textit{Ekosystem Ethereum}}                                    \\ \hline
    \textit{Opis:}                 & \multicolumn{3}{m{11cm}|}{\textit{Wszystkie produkty wygenerowane w czasie trwania projektu opierają się na technologiach pochodzących z ekosystemu Ethereum, takich jak sieć Ethereum, sieć Swarm, sieć Whisper, biblioteki Web3.}} \\ \hline
    \textit{Kryteria akceptacji :} & \multicolumn{3}{m{11cm}|}{\textit{brak}}    \\ \hline

\end{tabular}
\end{table}

\subsubsection{Wymagania dotyczące procesów wytwarzania}

\begin{table}[H]
    \begin{tabular}{|l|l|l|l|}
    \hline
    \multicolumn{4}{|l|}{\textit{\textbf{Karta Wymagania}}}                                                           \\ \hline
    \textit{Identyfikator:}        & \textit{PW01}            & \textit{\textbf{Status:}}            & M            \\ \hline
    \textit{Nazwa:}                & \multicolumn{3}{l|}{\textit{TDD - Test Driven Development}}                                    \\ \hline
    \textit{Opis:}                 & \multicolumn{3}{m{11cm}|}{\textit{Każda zmiana lub nowa funkcjonalność wprowadzana do systemu, musi być najpierw zamodelowana z pomocą testów. Dopiero po tym testy specyfikują jak docelowy kod ma wyglądać.}} \\ \hline
    \textit{Kryteria akceptacji :} & \multicolumn{3}{m{11cm}|}{\textit{brak}}    \\ \hline
\end{tabular}
\end{table}

\begin{table}[H]
    \begin{tabular}{|l|l|l|l|}
    \hline
    \multicolumn{4}{|l|}{\textit{\textbf{Karta Wymagania}}}                                                           \\ \hline
    \textit{Identyfikator:}        & \textit{PW02}            & \textit{\textbf{Status:}}            & M            \\ \hline
    \textit{Nazwa:}                & \multicolumn{3}{l|}{\textit{Określony git-flow}}                                    \\ \hline
    \textit{Opis:}                 & \multicolumn{3}{m{11cm}|}{\textit{Każda nowa funkcjonalność lub ich zbiór powinien być opisany w Issue. Branche z implementacją danego Issue powinny mieć postać ‘feature/IS\%numer\_issue\%/\%dodatkowy\_opis\%’. Zmianny do testów i do kodu powinny być zawarte w osobnych commitach. Wiadomość w commitach powinny mieć postać ‘IS\%numer\_issue\% test/code - \%opis\_commitu\%’.}} \\ \hline
    \textit{Kryteria akceptacji :} & \multicolumn{3}{m{11cm}|}{\textit{brak}}    \\ \hline
\end{tabular}
\end{table}

\subsection{Scenariusze i analiza umów}

\subsubsection{Umowa A1.1}
\textbf{Scenariusz:} \\
\begin{table}[H]
    \begin{tabular}{ll}
    \multicolumn{2}{l}{\textit{\textbf{Umowa A1.1 --- umowa prosta}}}                                                           \\ \hline
    0.        & S tworzy
\\ \hline
    1.        & K wpłaca
\\ \hline
    2.        & S potwierdza i wypłaca
\\ \hline
    3.       & K ocenia
\\
	\multicolumn{2}{l}{}
\\        
	2a.        & S odrzuca
\\ \hline   
	3a.        & K wypłaca
\\ \hline     
	4a.        & Koniec
\\                                 
\end{tabular}
\end{table}
\textbf{Legenda:} \\
S --- Sprzedający \\
K --- Kupujący \\ \\

\newpage
\subsubsection{Umowa A1.2}
\textbf{Scenariusz:} \\
\begin{table}[H]
    \begin{tabular}{ll}
    \multicolumn{2}{l}{\textit{\textbf{Umowa A1.2 --- umowa z zaliczką}}}                                                           \\ \hline
    0.        & S tworzy i ustala cenę, zaliczkę i czas do wycofania się
\\ \hline
    1.        & K wpłaca cenę
\\ \hline
    2.        & S potwierdza
\\
	\multicolumn{2}{l}{\textit{(teraz >= ostateczny termin wycofania się)}}
\\   
    3.       & S wypłaca i ocenia
\\ \hline
    4.       & K ocenia
\\ 
	\multicolumn{2}{l}{}
\\        
	2a.        & S odrzuca
\\ \hline   
	3a.        & K wypłaca
\\ \hline     
	4a.        & Koniec
\\   
	\multicolumn{2}{l}{}
\\        
	3b.        & K nie potwierdza i żąda zwrotu (cena – zaliczka) jeśli \textit{(teraz < ostateczny termin wycofania się)}
\\ \hline 
	4b.        & S wypłaca i ocenia
\\ \hline     
	5b.        & Koniec
\\                                
\end{tabular}
\end{table}
\textbf{Legenda:} \\
S --- Sprzedający \\
K --- Kupujący \\ \\

\newpage


\subsubsection{Umowa A1.3}
\textbf{Scenariusz:} \\
\begin{table}[H]
    \begin{tabular}{ll}
    \multicolumn{2}{l}{\textit{\textbf{Umowa A1.3 --- umowa z kaucjami}}}                                                           \\ \hline
    0.        & S tworzy
\\ \hline
    1.        & K wpłaca k1 + cena
\\ \hline
    2.        & S wpłaca k2
\\ \hline
    3.        & K potwierdza, ocenia i wypłaca k1
\\ \hline
    4.        & S wypłaca k2 + cena i ocenia
\\ 
	\multicolumn{2}{l}{}
\\        
	2a.        & S odrzuca
\\ \hline   
	3a.        & K wypłaca k1 + cena
\\ \hline     
	4a.        & Koniec
\\   
	\multicolumn{2}{l}{}
\\        
	3b.        & K odrzuca --- nowa\_cena
\\ \hline 
	4ba.        & S ugoda --- wypłaca nowa\_cena + k2 i ocenia
\\ \hline     
	5ba.        & K wypłata cena - cena\_nowa + k1 i ocenia
\\ \hline  
	6ba.        & Koniec
\\   
	\multicolumn{2}{l}{}
\\        
	4bb.        & S odrzuca --- nowa\_cena\_2
\\ \hline     
	5bb.        & K ugoda --- wypłaca cena - nowa\_cena\_2 + k1 i ocenia
\\ \hline  
	6bb.        & S wypłaca nowa\_cena\_2 + k2 i ocenia   
\\ \hline  
	7bb.		& Koniec     
\\   
	\multicolumn{2}{l}{}
\\   
	\multicolumn{2}{l}{\textit{(liczba odrzuceń > max liczba odrzuceń)}}
\\                
    3+l. odrzuceń.b.*.        & S/K odrzuca --- k1+k2 zostają zabrane przez system
\\ \hline  
	4+l. odrzuceń.b.*.        & S i K oceniają. S wypłaca cenę  
\\ \hline  
	5+l. odrzuceń.b.*.			& Koniec     
\\ 
\end{tabular}
\end{table}
\textbf{Legenda:} \\
S --- Sprzedający \\
K --- Kupujący \\
k1, k2 --- kaucje\\ \\

\newpage
\subsubsection{Umowa A2.1}
\textbf{Abstrakt}\\

Umowa prostej jawnej aukcji, podobnie jak umowa A1.1, wymaga wpłacenia środków przez
klienta (dalej nazywanego K[numer]) na portfel umowy, w celu jej zawarcia (może być dodana
opcja zatwierdzenia K przy pierwszej wpłacie), jednak w przeciwieństwie do niej na początku
następuje sekwencja licytacji. W trakcie jej K1-Km obserwują kwoty wpłacane przez innych
uczestników i przebijają ostatnią najwyższą ofertę. Faza licytacji kończy się pod wpływem decyzji
sprzedającego (dalej nazywany S) lub innego zdarzenia [wymagane jest określenie jakie
wydarzenia mogą zakończyć aukcję i jakie potencjalne pod-umowy mogą generować].\\ \\
\textbf{Scenariusz:} \\
\begin{table}[H]
    \begin{tabular}{ll}
    \multicolumn{2}{l}{\textit{\textbf{Umowa A2.1 --- aukcja prosta jawna}}}                                                           \\ \hline
    0.        & S tworzy umowę
\\ \hline
    1.        & K1 wpłaca kwotę k1
\\ \hline
    2.        & \textit{opcjonalnie S zatwierdza udział K1}
\\ \hline
    3.        & Aukcja się kończy --- S wypłaca k1
\\ \hline
    4.        & K1 ocenia S
\\ 
	\multicolumn{2}{l}{}
\\        
	2a.        & K2 wpłaca kwotę k2 (k2 > k1)
\\ \hline   
	3a.        &\textit{opcjonalnie S zatwierdza udział K2}
\\ \hline     
	4a.        & K1 wypłaca k1
\\   
	\multicolumn{2}{l}{}
\\        
	Na.        & Km wpłaca kwotę kN (kN > k(N-1))
\\ \hline 
	(N+1)a.        & Aukcja się kończy --- S wypłaca kN
\\ \hline     
	5ba.        & Km ocenia S
\\ 
	\multicolumn{2}{l}{}
\\        
	2b.        & \textit{Opcjonalne --- S odrzuca K1}
\\ \hline     
	3b.        & \textit{Opcjonalne --- K1 wypłaca k1}
\\ 
\end{tabular}
\end{table}
\textbf{Legenda:} \\
K1, K2, …, Km – klienci biorący udział w aukcji (m <= N) \\
S – sprzedawca \\ \\
Zatwierdzanie powinno nastąpić tylko raz przy pierwszej wpłacie. W przeciwnym wypadku sprzedający może mieć za dużą kontrolę nad aukcją (potencjalne skutki wymagają weryfikacji) \\
Aukcja się kończy, pod wpływem decyzji S lub innego zdarzenia \\ \\
\textbf{Analiza}

Jawna aukcja posiada tę zaletę, że dają możliwość K1-Km między sobą konkurować, a
przede wszystkim obserwować jak kształtuje się ostateczna cena, jednak tylko wtedy, gdy
sprzedawca jest uczciwy. Nieuczciwy sprzedawca nie może bezpośrednio ingerować w przebieg
aukcji, jednak z pomocą multi-konta lub wspólników może windować cenę, nieproporcjonalnie do
wartości oferowanego produktu. Jedyną ochroną w tej sytuacji dla Kn jest system ocen,
pozwalający sprawdzić sprzedawcę przed zawiązaniem umowy lub go ukarać po jej zakończeniu.

Należy mieć jednak na uwadze przy ocenianiu „sprawiedliwości” tej transakcji, że cała idea
aukcji opiera się na tym, że daje sprzedającemu przewagę, tzn. klienci konkurują między sobą
oferując wyższe kwoty za towar. Ten problem może być rozwiązany z pomocą grup handlowych,
które mogą kontrolować i narzucać dodatkowe zasady na przebieg umowy i jej kontrahentów.

Rola trolli jest ograniczona, głównie przez ich portfel (windowanie ceny wymaga posiadania
funduszy, poza tym troll ryzykuje „wygranie” aukcji) i jest tym bardziej ograniczona im jest
większa cena wejścia do aukcji, a pozostali gracze dysponują duża ilością środków. Jednak nie
eliminuje to całkowicie problemu i prosta aukcja jawna w obecnej postaci może nie być
odpowiednia do transakcji, które wymagają dużej odporności na ataki zamożnych adwersarzy.

\newpage
\subsubsection{Umowa A2.2}
\textbf{Abstrakt}\\

Umowa niejawnej prostej aukcji składa się z trzech faz: licytacja, odsłanianie, wypłaty. W
trakcie licytacji klienci (oznaczeni dalej jako K1-KM) wpłacają depozyty k1-kM (większe równe od
opcjonalnej minimalnej ceny) razem z kryptograficznymi haszami h1-hM. Kolejność w jakiej są
dokonywane wpłaty jest dowolna. Wpłaty mogą wymagać opcjonalnego zatwierdzenia przez
sprzedającego. Faza licytacji jest kończona predefiniowanym zdarzeniem [wymagane jest
określenie jakie wydarzenia mogą zakończyć aukcję i jakie potencjalne pod-umowy mogą
generować], np. decyzją sprzedającego (nazywanego dalej S).

Kolejnym krokiem jest odsłanianie, w trakcie którego K1-KM wysyłają p – faktyczną
kwotę, f – znacznik prawdziwości i s – sól. Na podstawie tych wartości jest wyliczany
kryptograficzny hasz – hasz(p.f.s) i jest on porównywany z wysłaną poprzednio wartością h.
Wpłaty spełniające równość h == hasz(p.f.s), są oznaczane jako poprawne, pozostałe jako
niepoprawne. Wpłaty spełniające dodatkowo warunek f \& k >= p (1.) są nominowane do udziału w
licytacji. Ta faza podobnie jak poprzednia kończy się pod wpływem wydarzenia [wymagane jest
określenie jakie wydarzenia mogą zakończyć aukcję i jakie potencjalne pod-umowy mogą
generować].

Ostatnim etapem są wypłaty, w trakcie których:

\begin{enumerate}
	\item Poprawna wpłata spełniająca 1. z najwyższym p, należąca do pewnego KI, jest wypłacana S
	\item KI wypłaca kI – pI i ocenia S
	\item Poprawne wpłaty są wypłacane przez odpowiednich K
	\item Niepoprawne wpłaty są rekwirowane przez system
\end{enumerate}
Faza wypłat jest kończona wraz z wypłaceniem wszystkich poprawnych wpłat.\newpage
\textbf{Scenariusz:} \\
\begin{table}[H]
    \begin{tabular}{ll}
    \multicolumn{2}{l}{\textit{\textbf{Umowa A2.2 ---  aukcja prosta niejawna}}}                                                           \\ \hline
    0.        & S tworzy umowę
\\ \hline
    1. async1       & K1 wpłaca depozyt k1 i wysyła kryptograficzny hasz h1
\\ \hline
    2. async1        & \textit{opcjonalnie S zatwierdza udział K1 przy pierwszej wpłacie}
\\ \hline
    3. async2        & K2 wpłaca depozyt k2 i wysyła kryptograficzny hasz h2
\\ \hline
    4. async2        & \textit{opcjonalnie S zatwierdza udział K2 przy pierwszej wpłacie} 
\\ 
	\multicolumn{2}{l}{}
\\        
	N. async N        & KM wpłaca depozyt kM i wysyła kryptograficzny hasz hM 
\\ \hline   
	N+1. async N        & \textit{opcjonalnie S zatwierdza udział KM przy pierwszej wpłacie}
\\ \hline     
	N+2.        & Faza licytacji się kończy 
\\    \hline   
	N+3. async1         & K1 wysyła w postaci jawnej wartość p1, sól s1 i znacznik prawdziwości f1
\\ \hline     
	N+4. async2         & K2 wysyła w postaci jawnej wartość p2, sól s2 i znacznik prawdziwości f2

\\   
	\multicolumn{2}{l}{}
\\        
	N+M. async M .        & KM wysyła w postaci jawnej wartość pM, sól sM i znacznik prawdziwości fM
\\ \hline 
	N+M+1.        & Koniec aukcji – S wypłaca środki w ilości pI = max(S), gdzie
\\ \hline
			&  S = \{pK: 1=<K=<N \& wysłane(pK) \& fK \& pK <= kK \& hK == hasz(pK.fK.sK)\}
\\ \hline     
	N+M+2.        &  KI ocenia S i wypłaca kI – pI 
\\ \hline  
	N+M+3.        & Klienci ze zbioru
\\ \hline
			& R = \{KK: 1=<K=<N \& wysłane(pK) \& hK == hash(pK.fK.sK)\} wypłacają wpłacone k 
\\   \hline  
	N+M+4.        & Klienci spoza zbioru R tracą k na rzecz systemu
\\ 
\end{tabular}
\end{table}
\textbf{Legenda:} \\
K1-KM – klienci \\
async[numer] – sąsiadujące kroki oznaczone async o różnych numerach wykonują się w dowolnej kolejności \\
Klienci mogą dokonać więcej niż jednej wpłaty \\ \\
\textbf{Analiza}

W aukcji niejawnej sprzedawca i klienci mogą jedynie dowiedzieć się jaki jest górny limit
potencjalnych ofert, bez gwarancji że wszystkie będą brać udział w wyznaczaniu ostatecznej ceny.
Taki system utrudnia konkurowanie między K, ale też ogranicza nieuczciwe praktyki ze strony S. K
muszą polegać głównie na własnej ocenie produktu i swoich możliwości przy składaniu propozycji
ceny. S nie zna prawdziwych kwot obstawianych przez K, więc nie ma jaki ich skutecznie
windować. Jedyne co może zrobić to podwyższać potencjalny górny limit ofert, co nie gwarantuje,
że K podniosą prawdziwe stawki.

Działania trolli w tym rodzaju aukcji nie są szkodliwe, wręcz przeciwnie, mogą być
pożądane, np. K wystawiają fałszywe oferty, by wystraszyć innych kupców.

Klienci w fazie odsłaniania mają już wgląd do szczegółów ofert innych K, którzy ujawnili o
nich dane, co może rodzić chęć do nie ujawniania części swoich ofert, np. pewien KX złożył dwie
propozycje, jedna opiewająca na relatywnie niską kwotę, a drugą na znacznie większą. Po
odczekaniu, aż pozostali uczestnicy aukcji ujawnią swoje oferty, odkrył, że pierwsza niższa jest
wystarczająca do zwycięstwa, co może zrodzić chęć nie ujawnienia drugiej mniej opłacalnej.Takie działanie kłóci się z ideą tej umowy i by zniechęcać do takich działań, wpłaty które nie
zostały poprawnie odsłonięte, będą rekwirowane przez system.

Problemem do dalszej analizy jest dokładne określenie ograniczeń na zdarzenie kończące
fazę odsłaniania. Jest możliwe, że złośliwy S lub źle skonfigurowane zdarzenie doprowadzi do
sytuacji, gdzie większość K nie zdąży odsłonić swoich ofert, przez co zostaną one pochłonięte przez
system. Można to rozwiązać nakładając na sztywno dodatkowe ograniczenie czasowe, ale dokłada
to kolejny szczegół implementacyjny na wyższym poziomie abstrakcji.

\newpage
\subsubsection{Umowa A3.1}
\textbf{Scenariusz:} \\
\begin{table}[H]
    \begin{tabular}{ll}
    \multicolumn{2}{l}{\textit{\textbf{Umowa A3.1 ---  prosta umowa z świadkami (pasywna)}}}                                                           \\ \hline
    0.        & S tworzy umowę
\\ \hline
    1.       & K zawiera umowę i wpłaca cenę – zgłasza zbiór ŚK 2t+1 świadków 
\\ \hline
    2.        & S zatwierdza K i zgłasza zbiór SŚ 2t+1 świadków, taki, że ŚK  n ŚS =/= O
\\ \hline
    3.        & K potwierdza poprawność 
\\ \hline
    4.        &  S wypłaca i ocenia K 
\\ \hline
    5.        & K ocenia S
\\  \hline
            & KONIEC
\\ 
	\multicolumn{2}{l}{}
\\        
	3a.        & K nie potwierdza  
\\ \hline   
	4a.        & ŚK U ŚS oraz K i S głosują na K lub S – procedura głosowania 
\\ \hline     
	5a.        & Większość za K – K wypłaca i ocenia S i ŚS 
\\    \hline   
	6a.         & S ocenia ŚK i K
\\    
	\multicolumn{2}{l}{}
\\        
	5aa.        & Większość za S – S wypłaca i ocenia K i ŚK 
\\ \hline 
	6aa.        & K ocenia ŚS i S
\\ 
	\multicolumn{2}{l}{}
\\        
		     & if(nieparzysta ilość nie zagłosuje i jest remis) 
\\ \hline
	4ab.        & S wypłaca i ocenia ŚS 
\\ \hline 
	5ab.        & K ocenia S i ŚK
\\ 
	\multicolumn{2}{l}{}
\\        
	7a.	     & ŚK U ŚS (ci co zagłosowali) ocenia S i K (wypłacają prowizje)
\\
\end{tabular}
\end{table}

\newpage
\subsubsection{Umowa A3.2}
\textbf{Scenariusz:} \\
\begin{table}[H]
    \begin{tabular}{ll}
    \multicolumn{2}{l}{\textit{\textbf{Umowa A3.2 ---  prosta umowa z losowaniem świadków (pasywna)}}}                                                           \\ \hline
    0.        & S tworzy umowę i zgłasza swoją połowę zbioru świadków 
\\ \hline
    1.        &  K zawiera umowę, wpłaca cenę
\\ \hline
		& i zgłasza swoją połowę zbioru świadków 
\\ \hline
    2.        &  S zatwierdza K 
\\ \hline
    3.        &  K potwierdza poprawność 
\\ \hline
    4.        & S wypłaca i ocenia K
\\ \hline
    5.	    &  K ocenia S 
\\ \hline
	    & KONIEC
\\
	\multicolumn{2}{l}{}
\\        
	3a.        & K nie potwierdza 
\\ \hline   
	4a.        & System losuje zbiór świadków Ś 2t+1 z sumy zgłoszonych zbiorów 
\\ \hline     
	5a.        & Ś oraz K i S głosują na K lub S – procedura głosowania 
\\   \hline     
	6a.        & Większość za K – K wypłaca i ocenia S i Ś
\\   \hline     
	7a.        & S ocenia Ś i K
\\   
	\multicolumn{2}{l}{}
\\        
	6ab.        & Większość za S – S wypłaca i ocenia K i Ś 
\\ \hline 
	7ab.        &  K ocenia Ś i S
\\
	\multicolumn{2}{l}{}
\\        
	       & if(nieparzysta ilość nie zagłosuje i jest remis) 
\\ \hline     
	5ac.        &  S wypłaca i ocenia Ś
\\ \hline  
	6ac.        &  K ocenia S i Ś 
\\  
	\multicolumn{2}{l}{}
\\   
	8a.		& Ś (ci co zagłosowali) ocenia S i K (wypłacają prowizje)     
\\   
\end{tabular}
\end{table}


\newpage
\subsubsection{Umowa A3.3}
\textbf{Scenariusz:} \\
\begin{table}[H]
    \begin{tabular}{ll}
    \multicolumn{2}{l}{\textit{\textbf{Umowa A3.3 ---  umowa z zaliczką i świadkami (pasywna)}}}                                                           \\ \hline
    0.        & S tworzy umowę oraz ustala cenę i zaliczkę 
\\ \hline
    1.        &   K zawiera, wpłaca cenę i zgłasza zbiór ŚK 2t+1 świadków 
\\ \hline
    2.        & S zatwierdza K i zgłasza zbiór SŚ 2t+1 świadków, taki że ŚK n ŚS =/= O
\\ \hline
    3.        & K potwierdza poprawność 
\\ \hline
    4.        & S wypłaca i ocenia K
\\ \hline
    5.	    &  K ocenia S 
\\ \hline
	    & KONIEC
\\
	\multicolumn{2}{l}{}
\\        
	3a.        &  K nie potwierdza i żąda zwrotu (cena – zaliczka) jeśli 
\\ \hline
		  & (teraz < ostateczny termin wycofania się) 
\\ \hline   
	4a.        &  S wypłaca zaliczkę i ocenia K
\\ \hline     
	5a.        &  K wypłaca (cena – zaliczka) 
\\   \hline     
	       & KONIEC
\\   
	\multicolumn{2}{l}{}
\\        
	3b.        & K nie potwierdza i żąda zwrotu całej kwoty 
\\ \hline 
	4b.        & ŚK U ŚS oraz K i S głosują na K lub S – procedura głosowania 
\\ \hline 
	5b.        &   Większość za K – K wypłaca i ocenia S i ŚS 
\\ \hline 
	6b.        &  S ocenia ŚK i K
\\
	\multicolumn{2}{l}{}
\\        
	5ba.        & Większość za S – S wypłaca i ocenia K i ŚK 
\\ \hline  
	6ba.        &   K ocenia ŚS i S
\\  
	\multicolumn{2}{l}{}
\\   
			& if(nieparzysta ilość nie zagłosuje i jest remis) 
\\ \hline     
	4bb.		& S wypłaca cenę i ocenia ŚS
\\   \hline  
	5bb.        & K ocenia S i ŚK 
\\  \hline  
	7b.        &   ŚK U ŚS (ci co zagłosowali) ocenia S i K (wypłacają prowizje)
\\  
\end{tabular}
\end{table}


\newpage
\subsubsection{Umowa A3.4}
\textbf{Scenariusz:} \\
\begin{table}[H]
    \begin{tabular}{ll}
    \multicolumn{2}{l}{\textit{\textbf{Umowa A3.4 ---   prosta umowa z świadkami (aktywna)}}}                                                           \\ \hline
    0.        &  S tworzy umowę 
\\ \hline
    1.        &    K zawiera umowę i wpłaca cenę – zgłasza zbiór ŚK 2t+1 świadków 
\\ \hline
    2.        & S zatwierdza K i zgłasza zbiór SŚ 2t+1 świadków, taki że ŚK n ŚS =/= O
\\ \hline
    3.        & ŚK U ŚS oraz K i S głosują na K lub S – procedura głosowania
\\ \hline
    4.        & Większość za S – S wypłaca i ocenia K i ŚK 
\\ \hline
    5.	    &  K ocenia ŚS i S 
\\ \hline
    6.	    &  ŚK U ŚS (ci co zagłosowali) ocenia S i K (wypłacają prowizje) 
\\ \hline
		& KONIEC
\\
	\multicolumn{2}{l}{}
\\        
	4a.        &   Większość za K – K wypłaca i ocenia S i ŚS 
\\ \hline   
	5a.        &  S ocenia ŚK i K 
\\ \hline     
	        & powrót do 6.
\\  
	\multicolumn{2}{l}{}
\\   
			& if(nieparzysta ilość nie zagłosuje i jest remis) 
\\ \hline     
	4b.		&  S wypłaca i ocenia ŚS 
\\   \hline  
	5b.        & K ocenia S i ŚK  
\\  \hline  
	        &   powrót do 6.
\\  
\end{tabular}
\end{table}

\newpage
\subsubsection{Umowa A4.2}
\textbf{Scenariusz:} \\
\begin{table}[H]
    \begin{tabular}{ll}
    \multicolumn{2}{l}{\textit{\textbf{Umowa A4.2 ---   aukcja prosta niejawna ze świadkami (pasywna)}}}                                                           \\ \hline
    0.        & S tworzy umowę i zgłasza zbiór świadków ŚS o rozmiarze 2t+1 
\\ \hline
    1. async1       & K1 wpłaca depozyt k1 i wysyła kryptograficzny hasz h1
\\ \hline
		& Jeśli nie ma świadka, to zgłasza zbiór ŚK1 2t+1 świadków, taki że ŚK1 n ŚS =/= O
\\ \hline
    2. async1        &  S zatwierdza udział K1 przy pierwszej wpłacie
\\ \hline
    3. async2        &  K2 wpłaca depozyt k2 i wysyła kryptograficzny hasz h2 
\\ \hline
		& Jeśli nie ma świadka, to zgłasza zbiór ŚK2 2t+1 świadków, taki że ŚK2 n ŚS =/= O
\\ \hline
    4. async2        & S zatwierdza udział K2 przy pierwszej wpłacie
\\ 
	\multicolumn{2}{l}{}
\\        
	N. async N        & KM wpłaca depozyt kM i wysyła kryptograficzny hasz hM
\\ \hline   
	N+2. async N         & S zatwierdza udział KM przy pierwszej wpłacie 
\\ \hline     
	N+3.        & Faza licytacji się kończy 
\\    \hline   
	N+4. async1         & K1 wysyła w postaci jawnej wartość p1, sól s1 i znacznik prawdziwości f1
\\ \hline     
	N+5. async2         & K2 wysyła w postaci jawnej wartość p2, sól s2 i znacznik prawdziwości f2
\\   
	\multicolumn{2}{l}{}
\\        
	N+M+3. async M .        & KM wysyła w postaci jawnej wartość pM, sól sM i znacznik prawdziwości fM
\\ \hline 
	N+M+4.        & Koniec aukcji – wybieramy ŚKI i KI, takie że pI == max(S), gdzie 
\\ \hline
			&  S = \{pK: 1=<K=<N \& wysłane(pK) \& fK \& pK <= kK \& hK == hasz(pK.fK.sK)\}. 
\\ \hline     
	N+M+5.        & KI potwierdza poprawność 
\\ \hline  
	N+M+6.        &  S wypłaca środki w ilości pI i ocenia KI 
\\ \hline  
	N+M+7.        &  KI ocenia S i wypłaca kI – pI
\\  
	\multicolumn{2}{l}{}
\\        
	N+M+5.a.        & KI nie potwierdza
\\ \hline 
	N+M+6.a.        & ŚKI U ŚS oraz K i S głosują na KI lub S – procedura głosowania 
\\ \hline     
	N+M+7.a.        &  Większość za KI – KI wypłaca kI i ocenia S i ŚS 
\\ \hline  
	N+M+8.a.        &  S ocenia ŚKI i KI
\\ 
	\multicolumn{2}{l}{}
\\       
	N+M+7.aa.        &  Większość za S – S wypłaca pI i ocenia KI i ŚKI 
\\ \hline  
	N+M+8.aa.        &   KI ocenia ŚS i S i wypłaca kI – pI
\\ 
	\multicolumn{2}{l}{}
\\      
		& if(nieparzysta ilość nie zagłosuje i jest remis) 
\\ \hline
	N+M+6.ab.        &  S wypłaca pI i ocenia ŚS 
\\ \hline  
	N+M+7.ab.        & KI ocenia S i ŚKI i wypłaca kI – pI [następny N+M+9.a.]
\\ 
	\multicolumn{2}{l}{}
\\  
	N+M+9.a.        & ŚKI U ŚS (ci co zagłosowali) ocenia S i K (wypłacają prowizje) [następny N+M+9.]
\\ 
	\multicolumn{2}{l}{}
\\  
	N+M+9.        &  Klienci ze zbioru 
\\ \hline
			&  R=\{KK: 1=<K=<N\& wysłane(pK)\& hK== hash(pK.fK.sK)\} wypłacają wpłacone k 
\\ 
	\multicolumn{2}{l}{}
\\  
	N+M+10.        &  Klienci spoza zbioru R tracą k na rzecz systemu
\\ 
\end{tabular}
\end{table}
\textbf{Legenda:} \\
K1-KM – klienci \\
async[numer] – sąsiadujące kroki oznaczone async o różnych numerach wykonują się w dowolnej kolejności \\
Klienci mogą dokonać więcej niż jednej wpłaty \\ \\

\newpage
\subsubsection{Umowa A4.3}
\textbf{Scenariusz:} \\
\begin{table}[H]
    \begin{tabular}{ll}
    \multicolumn{2}{l}{\textit{\textbf{Umowa A4.3 ---   aukcja prosta niejawna ze świadkami (aktywna)}}}                                                           \\ \hline
    0.        & S tworzy umowę i zgłasza zbiór świadków ŚS o rozmiarze 2t+1 
\\ \hline
    1. async1       &  K1 wpłaca depozyt k1 i wysyła kryptograficzny hasz h1 (opcjonalnie opłaty dla Ś)
\\ \hline
		& Jeśli nie ma świadka, to zgłasza zbiór ŚK1 2t+1 świadków, taki że ŚK1 n ŚS =/= O
\\ \hline
    2. async1        &   S zatwierdza udział K1 przy pierwszej wpłacie 
\\ \hline
    3. async2        &  K2 wpłaca depozyt k2 i wysyła kryptograficzny hasz h2 (opcjonalnie opłaty dla Ś)
\\ \hline
		& Jeśli nie ma świadka, to zgłasza zbiór ŚK2 2t+1 świadków, taki że ŚK2 n ŚS =/= O
\\ \hline
    4. async2        & S zatwierdza udział K2 przy pierwszej wpłacie
\\ 
	\multicolumn{2}{l}{}
\\        
	N. async N        &  KM wpłaca depozyt kM i wysyła kryptograficzny hasz hM (opcjonalnie opłaty dla Ś)
\\ \hline 
			& Jeśli nie ma świadka, to zgłasza zbiór ŚKM 2t+1 świadków, taki że ŚKM n ŚS =/= O
\\ \hline  
	N+2. async N         & S zatwierdza udział KM przy pierwszej wpłacie 
\\ \hline     
	N+3.        & Faza licytacji się kończy 
\\    \hline   
	N+4. async1         & K1 wysyła w postaci jawnej wartość p1, sól s1 i znacznik prawdziwości f1
\\ \hline     
	N+5. async2         & K2 wysyła w postaci jawnej wartość p2, sól s2 i znacznik prawdziwości f2
\\   
	\multicolumn{2}{l}{}
\\        
	N+M+3. async M .        & KM wysyła w postaci jawnej wartość pM, sól sM i znacznik prawdziwości fM
\\ \hline 
	N+M+4.        & Koniec aukcji – wybieramy ŚKI i KI, takie że pI == max(S), gdzie 
\\ \hline
			&  S = \{pK: 1=<K=<N \& wysłane(pK) \& fK \& pK <= kK \& hK == hasz(pK.fK.sK)\}.  
\\ \hline     
	N+M+5.        &  Głosowanie – większość za 
\\ \hline  
	N+M+6.        &  S wypłaca środki w ilości pI i ocenia KI i ŚKI 
\\ \hline  
	N+M+7.        &  KI ocenia S i ŚS i wypłaca kI – pI 
\\  \hline  
	N+M+8.        &  ŚKI U ŚS (ci co zagłosowali) oceniają KI i S (opcjonalnie wypłacają swoje prowizje) 
\\  \hline  
	N+M+9.        & Klienci ze zbioru
\\ \hline
			&  R=\{KK: 1=<K=<N\& wysłane(pK)\& hK== hash(pK.fK.sK)\} wypłacają wpłacone k 
\\  \hline  
	N+M+10.        &  Klienci spoza zbioru R tracą k na rzecz systemu
\\  
	\multicolumn{2}{l}{}
\\        
	N+M+5.a.        & Większość przeciw 
\\ \hline 
	N+M+6.a.        &  KI wypłaca i ocenia S i ŚS 
\\ \hline     
	N+M+7.a.        &  S ocenia KI i ŚKI 
\\ \hline  
	N+M+8.a.        &  ŚKI U ŚS (ci co zagłosowali) (wypłaca i) ocenia KI i S 
\\ \hline  
	N+M+9.a.        &  powrót do N+M+9.
\\ 
\end{tabular}
\end{table}
\textbf{Legenda:} \\
K1-KM – klienci \\
async[numer] – sąsiadujące kroki oznaczone async o różnych numerach wykonują się w dowolnej kolejności \\
Klienci mogą dokonać więcej niż jednej wpłaty \\
Klient może zgłosić tylko jednego U. Robi to przy pierwszej zaakceptowanej wpłacie.\\ \\

\newpage
\subsubsection{Umowa PA1}
\textbf{Scenariusz:} \\
\begin{table}[H]
    \begin{tabular}{ll}
    \multicolumn{2}{l}{\textit{\textbf{Umowa PA1 ---  proste wiązanie świadków prowizją}}}                                                           \\ \hline
    1.       &  Udziałowiec U tworzy umowę 
\\ \hline
    2.       &    Świadek Ś1 zgłasza się 
\\ \hline
    3.        &  U go akceptuje i wpłaca prowizję (lub nie) 
\\ \hline
    4.        &  Ś2 zgłasza się 
\\ \hline
    5.		&   U go akceptuje  i wpłaca prowizję (lub nie) 
\\
	\multicolumn{2}{l}{}
\\        
	N+1.        &   ŚN zgłasza się 
\\ \hline  
	N+2.        &  U go akceptuje  i wpłaca prowizję (lub nie)  
\\ \hline     
	N+3.        & U zgłasza A1  
\\    \hline   
	N+4.         &  A1 raportuje uczestnictwo świadków 
\\   
	\multicolumn{2}{l}{}
\\        
	N+M+2.        & U zgłasza AM 
\\ \hline 
	N+M+3.        & AM raportuje uczestnictwo świadków 
\\ \hline     
	N+M+4.        &   Świadkowie, którzy uczestniczyli (głosowali poprawnie) we wszystkich 
\\
			& umowach, do których zostali wytypowani, wypłacają prowizje 
\\ \hline  
	N+M+5.        &   U wypłaca resztę 
\\ 
	\multicolumn{2}{l}{}
\\        
	        & if(teraz > T) 
\\ \hline 
	1a.        &  Świadkowie wypłacają prowizje jeśli uczestniczyli we wszystkich 
\\
		& umowach, do których zostali wytypowani 
\\ \hline     
	2a.        &  U wypłaca resztę
\\ 
\end{tabular}
\end{table}
\textbf{Legenda:} \\
dane wejściowe: prowizja na świadka P, ilość świadków N, Ilość umów A po której jest wypłata M, czas wygaśnięcia T \\
dane dodatkowe: licznik zgłoszonych umów I \\ \\

\newpage
\subsubsection{Umowa PA2}
\textbf{Scenariusz:} \\
\begin{table}[H]
    \begin{tabular}{ll}
    \multicolumn{2}{l}{\textit{\textbf{Umowa PA2 ---  proste wiązanie świadków prowizją i kaucją}}}                                                           \\ \hline
    1.       &  Udziałowiec U tworzy umowę 
\\ \hline
    2.       &    Świadek Ś1 zgłasza się i wpłaca kaucję
\\ \hline
    3.        &  U go akceptuje i wpłaca prowizję (lub nie) 
\\ \hline
    4.        &  Ś2 zgłasza się i wpłaca kaucję
\\ \hline
    5.		&   U go akceptuje  i wpłaca prowizję (lub nie) 
\\
	\multicolumn{2}{l}{}
\\        
	N+1.        &   ŚN zgłasza się i wpłaca kaucję
\\ \hline  
	N+2.        &  U go akceptuje  i wpłaca prowizję (lub nie)  
\\ \hline     
	N+3.        & U zgłasza A1  
\\    \hline   
	N+4.         &  A1 raportuje uczestnictwo świadków 
\\   
	\multicolumn{2}{l}{}
\\        
	N+M+2.        & U zgłasza AM 
\\ \hline 
	N+M+3.        & AM raportuje uczestnictwo świadków 
\\ \hline     
	N+M+4.        &   Świadkowie, którzy uczestniczyli (głosowali poprawnie) we wszystkich 
\\
			& umowach, do których zostali wytypowani, wypłacają prowizja + kaucja 
\\ \hline  
	N+M+5.        &   U wypłaca resztę 
\\ 
	\multicolumn{2}{l}{}
\\        
	        & if(teraz > T) 
\\ \hline 
	1a.        &  Świadkowie wypłacają prowizja + kaucja jeśli uczestniczyli we wszystkich
\\
		& umowach, do których zostali wytypowani. Nie wytypowani do żadnych wypłacają samą kaucję
\\ \hline     
	2a.        &  U wypłaca resztę. Kaucje świadków, którzy nie wywiązali się, 
\\
				& zostają zabrane przez system.
\\ 
\end{tabular}
\end{table}
\textbf{Legenda:} \\
dane wejściowe: prowizja na świadka P, ilość świadków N, Ilość umów A po której jest wypłata M, czas wygaśnięcia T, kaucja K \\
dane dodatkowe: licznik zgłoszonych umów I \\ \\

\newpage
\subsubsection{Głosowanie świadków}
\textbf{Głosowanie WV1} \\
\textbf{Scenariusz:} \\
wejście: zbiór Ś, K, S, maksymalny czas trwania, zdarzenie kończące\\ \\
1. oddawanie głosów na K lub S wyjście: wynik, kto oddał głos \\ \\
\textbf{Głosowanie WV2} \\
\textbf{Scenariusz:} \\
wejście: zbiór Ś, K, S, maksymalny czas trwania, zdarzenie kończące fazę głosowania, zdarzenie kończące fazę odsłaniania\\ \\
1. wysyłanie haszy kryptograficznych h \\
2. wysyłanie głosów v i soli s \\
wyjście: wynik, kto oddał ważny głos (h == hash(v.s))\\ \\

\subsection{Specyfikacja przypadków użycia}
\subsubsection{Diagram przypadków uzycia}

\includegraphics[width=1.0\textwidth]{Use_Case_Diagram}

\subsubsection{Diagram sekwencji}

\newpage
\subsubsection{Maszyny stanowe}

Umowa A1.1\\
\includegraphics[width=1.0\textwidth]{SM-A1_1}


\subsection{Analiza komercyjna}

\newpage
\section{Architektura}

\subsection{Ekosystem ethereum}

\subsubsection{Siec ethereum}

Ethereum jest projektem pragnącym stworzyć uogólnioną technologię, pozwalającą na zrealizowanie dowolnego rozwiązania opierającego się na transakcyjnej maszynie stanowej. Co więcej, jednym z jego celów jest dostarczenie deweloperom pełnego, zwartego systemu do budowy aplikacji, we wcześniej szerzej nie znanym paradygmacie "trustful object messaging compute framework".

Innym kluczowym celem przyświecającym Ethereum, jest zestawienie kanału do interakcji między agentami, którzy w innych okolicznościach nie mieliby możliwości sobie zaufać, co do poprawności procedury. Źródłem nieufności może być geograficzna odległość, brak odpowiedniej wiedzy, niechęć, niepewność lub korupcja istniejącego systemu prawnego.\cite{work:ethereum2018yellowpaper}

\subsubsection{Własności sieci}

Kontrakty budowane w takim systemie, posiadają rzadko spotykane w rzeczywistym świecie własności. Bezstronność algorytmicznego interpretera, w naturalny sposób gwarantuje odporne na korupcję wykonanie transakcji. Przejrzystość rozstrzygnięcia kontraktu pochodząca z historii interakcji, jak również z zasad rządzących kodem, nie istniejąca w systemach opartch o języki naturalne i ludzkie działania, które w swojej naturze są nieprecyzyjne\cite{work:ethereum2018yellowpaper}.

Kontrakty są uruchamiane na EVM - Ethereum Virtual Machine, maszynie wykonującej ich byte-code (W Ethereum 2.0 będzie używany eWASM), której model programowy jest uważany za zupełny w sensie Turinga (nieformalnie, formalnie jest automatem liniowo ograniczonym). Pozwala to na oprogramowanie dowolnego algorytmu, który wykonałby domowy komputer\cite{wiki:Ethereum, website:EthTuringMedium}.

Oprócz ekspresywnego zestawu instrukcji, kontrakty mają możliwość zapisywania lub odczytania swojego stanu bezpośrednio z blockchaina, potrzeby bez angażowania w ten proces użytkownika. Posiadają one również prawie pełną swobodę, co do modyfikacji treści swojego stanu, oraz ograniczoną mozliwość zmiany stanu innych kontraktów. Pozwala to na budowanie rozwiązań, które są niemożliwe lub trudne do zbudowania z pomocą bezstanowych UXTO\cite{wiki:Ethereum, website:EthTuringMedium, work:ethereum2018yellowpaper}.

Klienci wchodzą w interakcję z siecią, po przez rozgłaszanie transakcji do znanych węzłów. Poprawnie uformowane i podpisane transakcjie, są zbierane i grupowane, przez wyróżnioną grupę węzłów nazwywaną górnikami (w Ethereum 2.0 walidatorami), w bloki. Mają oni pewną swobodę w tym ile transakcji będzie się znajdować w bloku i w jaki sposób będą one w nim uporządkowane. To ile maksymalnie transakcji może być w jednym bloku jest określane przez zmienną \textit{block gas limit}.
Kiedy górnik zbierze i przygotuję kolejkę transakcji, wykonuje w pętli funkcję tranzycji stanu, która przyjmuje obecny stan sieci i pierwszą transakcję z kolejki, a zwraca nowy stan sieci, dopóki kolejka nie jest pusta. W czasie wykonania pętli zbiera on informacje o tym jakie transakcjie zakończyły się powodzeniem i jakie logi one wygenerowały i zapisuje je w \textit{transaction reciepts trie}. Po wykonaniu wszystkich wybranych transakcji, górnik zapisuje nowy stan sieci w \textit{world state trie}, wykonane transakcje w \textit{transaction trie} i umieszcza je w bloku. Na tak przygotowanej strukturze wykonuje on algorytm Ethash (w Ethereum 2.0 ten mechanizm inaczej wygląda), czyli kopanie bloków. Po znalezieniu liczby spełniającej wymogi algorytmu, dołącza ją do nowo wykopanego bloku i rozgłasza go w sieci.
Zostaje on zaakceptowany przez resztę węzłów, jeśli nowy stan jest poprawny i wynika z transakcji zawartych w bloku, wskazuje na historyczny poprawny blok, a łańcuch do którego należy spełnia warunki protokołu GHOST (w uproszczeniu wybieramy ten łańcuch, nad którym wykonano najwięcej pracy).\cite{work:ethereum2018yellowpaper} Jest jeszcze kilka innych warunków poprawności bloku, ale nie są one bardzo istotne z punktu widzenia tej pracy.


\clearpage
\KOMAoptions{paper=A3,pagesize}
\KOMAoptions{paper=landscape}
\recalctypearea

\begin{center}

\begin{figure}[h]
\caption[Caption for LOF]{Formowanie bloków w Ethereum\footnotemark}
\centering
\includegraphics[width=1.0\textwidth, height=0.7\textheight]{blockchain}
\end{figure}
\footnotetext{Pobrano z \url{https://ethereum.stackexchange.com/questions/268/ethereum-block-architecture/6413#6413}}

\end{center}

\clearpage
\KOMAoptions{paper=portrait}
\KOMAoptions{paper=A4,pagesize}
\recalctypearea


Mechanizmem zachęcającym górników do zbierania i wykonywania transakcji jest obecność opłat za nie. Wykonanie każdej z nich kosztuje \textit{gas}; jednostkę wykonanej pracy, istniejącą tylko w kontekście danej transakcji. Użytkownik, w rozgłaszanej przez siebie transakcji, określia cenę w etherze za jednostkę gazu oraz maksymalną ilość gazu jaką może ona pobrać. Inną zachętą jest nagroda za wyprodukowanie poprawnego bloku, którą górnik może sam sobie przyznać. Takie mechanizmy wspierają żywotność sieci, jednak na ile chronią przed cenzurą i wzmacniają bezpieczeństwo sieci jest spornym tematem.\cite{work:ethereum2018yellowpaper}

Ethereum 1.0 gwarantuje wykonanie kontraktów i poprawne przeprowadzenie transakcji, czyli bezpieczeństwo i żywotność sieci, przy założeniu, że więcej niż 50\% mocy hashującej działa zgodnie z protokołem, a wiadomości o stanie sieci docierają do innych klientów synchronicznie (istnieje znane maksymalne opóźnienie wiadomości). Jeśli, któreś z powyższych założeń jest nie spełnione, to tranasakcje lub wykonanie kodu może być cenzurowane (zignorowane, niewykonane), różni uczestnicy sieci mogą widzieć jej odmienny stan, kontrakty i mechanizmy nie opierające się na podpisach kryptograficznych mogą paść ofiarą "double spending attack" oraz historia już wykonanych transakcji może być zmieniana \cite{work:AnonByzConsBit}.


\subsubsection{Środowisko wykonawcze kontraktów}

Kontrakty są bytami posiadającymi kod, jak również stan, nazywany też pamięcią \textit{storage}. Są one zapisane pod pewnym adresem w przestrzeni kont. Iterakcję z kontraktem przeprowadza się poprzez wysłanie pod jego adres transakcji. Kontrakt dotknięty nią uruchamia swój kod, z zawartością transakcji jako danymi wejściowymi.

Uruchomienie transakcji, jak i każdej wykonanej instrukcji w kodzie kontraktu pochłania pewną określoną ilość gazu. Inicjator transakcji określa cenę w etherze za jednostkę gazu i limit zużycia gazu. Cena za uruchomienie jest naliczana natychmiast, a kolejne opłaty, tylko w momencie wykonania instrukcji. Jeśli incjatorowi zabraknie funduszy na opłacenie gazu, zostanie osiągnięty limit zużycia gazu lub limit gazu na blok, wykonanie kończy się wyjątkiem \textit{out of gas} i zmiany wprowadzone w stanie sieci zostają wycofane. Jednak bez względu na wynik obliczeń, koszty zakupu gazu nie zostają cofnięte, jako że górnicy (lub walidatorzy) ponieśli już koszty z wiązane z ich wykonaniem. Poprawne wykonanie transakcji, czyli bez zwrócenia do inicjatora wyjątku, utrwala zmiany wprowadzone w sieci, a niezużyty gaz zostaje automatycznie odsprzedany i koszty jego zakupu zostają zwrócone do inicjatora, po takiej samej cenie jak zakupu.

Uruchomienie transakcji tworzy kontekst transakcji, czli zbiór zmiennych środowiskowych dostępnych w kontrakcie istniejących, ale tylko dla tego jednego wywołania. Należą do nich adres wywołującego, adres docelowy, ilość przekazanego etheru, ilość przekazanego gazu, cena za jednostkę gazu, adres opłacający transakcję oraz tablicę dodatkowych danych.

Szczególną uwagę należy poświęcić instrukcją \textit{REVERT}, \textit{INVALID}, \textit{CALL}, \textit{DELEGATECALL}, \textit{STATICCALL}, \textit{CREATE}, \textit{SELFDESTRUCT} i \textit{SSTORE}.
\begin{itemize}
	\item \textit{REVERT} kończy wykonanie wyjątkiem określony w kodzie przez jego autora, wycofuje wprowadzone zmiany w procedurze i zwraca niezużyty gaz do nadprocedury lub inicjatora transakcji.
	\item \textit{INVALID} kończy wykonanie wyjątkiem i "spala" cały dostępny gaz w procedurze.
	\item \textit{CALL} pozwala kontraktowi na uruchomienie podprocedury. Procedura wywołująca określa adres docelowy (może to być ten sam lub inny kontrakt), ile etheru chce przekazać, ile gazu chce dostarczyć oraz tablicę dodatkowych danych. Wyjątki wyrzucone w podprocedurze \textbf{nie kończą} automatycznie procedury wywołującej, jedynie zostaje do niej zwrócony kod błedu. Procedura może wysłać tylko tyle etheru, ile w danym momencie posiada kontrakt, podobnie w przypadku gazu; może przekazać maksymalnie 63/64 gazu, którego posiadała w momencie wykonania instrukcji \textit{CALL}. Wywołanie podprocedury, podobnie jak uruchomienie transakcji, tworzy nowy kontekst, jednak w przeciwieństwie do tej drugiej cena za jednostkę gazu i adres opłacającego nigdy nie ulega zmianie. Wynika to z tego, że w obecnej wersji Ethereum kontrakty nie mogą same opłacać swojego wykonania.
	\item \textit{DELEGATECALL} przyjmuje takie same argumenty jak \textit{CALL} jednak nie zmienia w pełni kontekstu tzn.\ podprocedura ma dostęp do tych samych zmiennych środowiskowych i pamięci \textit{storage} co nadprocedura, z wyjątkiem wysłanego etheru i dostępnego gazu. Ta instrukcja jest najczęsciej wykorzystywana w implementacji bibliotek, kontraktów z możliwością aktualizacji, jak również w naszym modelu maszyny stanowej
	\item \textit{STATICCALL} działa podobnie jak \textit{CALL}, jednka nie pozwala na przesyłanie etheru, a wszelkie próby wywołania instrukcji zmieniających stan sieci w podprocedurze, konczą ją i zwracają do nadprocedury kod błędu.
	\item \textit{CREATE} tworzy nowy kontrakt
	\item \textit{SELFDESTRUCT} usuwa z sieci kontrakt (zeruje kod i stan), w którego kontekście została ona wywołana, a cały ethere jaki był w nim zawarty przesyła pod wskazany adres. Jej wywołanie refunduje część zużytego gazu, co ma stanowić zachętę, by transakcje, jeśli to możliwe, usuwały nie potrzbne już kontrakty.
	\item \textit{SSTORE} zapisuje 256-bitowe słowo we wskazane miejsce \textit{storage} kontraktu, w którego kontekście została wywołana. Jeśli w jej wyniku niezerowa wartość zostaje wyzerowana, to refunduje część gazu, jako zachętę by transakcje usuwały nie potrzebne fragmenty stanu sieci
\end{itemize}

\subsubsection{Klienci Ethereum}

\begin{itemize}
	\item Geth
	\item Parity
	\item aleth
\end{itemize}

\subsubsection{Technologie wspierające}

\begin{itemize}
	\item Solidity
	\item Web3
	\item Swarm
	\item Whisper
\end{itemize}

\subsubsection{Ograniczenia technologii}

Kontrakty "widzą" jedynie stan sieci. O jakiekolwiek informacje o świecie zewnętrznym muszą prosić swoich klientów, co może być wektorem ataku na nie odpowiednio przemyślany i zaprojektowany system.

Obecna wersja Ethereum oferuje średnią przepustowość na poziomie 15 tx/s, co może utrudniać działanie aplikacji, w przypdaku dużego zainteresowania.

Problem prywatności w Ethereum jest dalej tematem intensywnych badań i choć są proponowane działające rozwiązania, w obecnej wersji należy zakładać, że każdy może sprawdzić jaki adres wchodził w interkację z jakim kontraktem.

\subsubsection{Zgodność z Ethereum 2.0}

Znaczna cześć zaproponowanych przez nas rozwiązań, będzie musiała być ponownie przeanalizowana, a nawet zmieniona, w kolejnych wersjach Ethreum, głównie z powodu dodania mechanizmu naliczania "czynszu" za używanie pamięci \textit{storage}, jak również pamięci, w której jest przechowywany kod kontraktów. Kolejną istotną zmianą będzie wprowadzenie \textit{shardingu}, który pozwoli zwiększyć przepustowość sieci, jak również doda nowe mechanizmy asynchronicznego, wzajemnego wywoływania się kontraktów. Dodanie obsługi tych funkcjonalności będzie kluczowe, by zapewnić poprawne działanie systemu i zapewnić jak najwyższą wydajność naszych usług.

\subsection{Podział systemu na podsystemy}

\subsubsection{Back-end}

\subsubsection{Front-end}

\subsection{Model obiektowy systemu}

\subsubsection{Back-end}

\clearpage
\KOMAoptions{paper=A3,pagesize}
\KOMAoptions{paper=landscape}
\recalctypearea

\includegraphics[width=1.0\textwidth, height=0.9\textheight]{Contract_Class_Diagram_Class_Diagram}

\clearpage
\KOMAoptions{paper=portrait}
\KOMAoptions{paper=A4,pagesize}
\recalctypearea

\subsubsection{Front-end}

\newpage
\section{Przyrost pierwszy}
\subsection{Cele przyrostu}
Podczas pierwszego przyrostu postawiliśmy sobie za cel przeanalizowanie, jak powinny wyglądać sprawiedliwe kontrakty, tak, aby każda ze stron nie czuła się w żaden sposób oszukana. Naszym kolejnym celem było zaproponowanie, jak system powinien wyglądać, jakich akcji może podejmować się użytkownik, jakie powinien mieć ograniczenia w danych przypadkach, oraz ustalić wstępną architekturę systemu. Oprócz tego, wykonaliśmy wstępny harmonogram prac do monitorowania postępów.

\subsection{Analiza back-end'u}
Na samym początku zabraliśmy się za analizę naszego systemu. Postanowiliśmy że naszym głównym przedmiotem badań będzie przeanalizowanie sprawiedliwych kontraktów. Podczas analizowania różnych przypadków postanowiliśmy że system powinien dostarczać różne rodzaje kontraktów w zależności od potrzeb użytkowników którzy będą korzystać z naszego systemu. Podczas prac analizowaliśmy różnego przypadki umów, na koniec postanowiliśmy że w ramach naszej pracy inżynierskiej stworzymy dwa typy kontraktów.

\begin{figure}[h!]
\centering
\includegraphics[width=0.35\textwidth]{A1_1.png}
\caption{Analiza umowy 1.1}
\label{A1_1}
\end{figure}

Umowa przedstawiona powyżej jest najprostszą dostępną umową w naszym systemie. Umowę rozpoczyna Sprzedawca, który tworzy umowę (punkt 0 na rysunku \ref{A1_1}) . Potencjalny kupujący zainteresowany umową wpłaca pieniądze(1), a w następnym kroku Sprzedawca potwierdza transakcję(2). Na koniec kupujący ocenia sprzedającego(3). W umowie istnieją alternatywne scenariusze, ponieważ sprzedający może nie chcieć zawrzeć umowy z nisko ocenianym kupującym i odrzucić ofertę(2a), gdzie następnym krokiem jest wypłacenie tokenów przez kupującego(3a). Jeżeli dojdzie do porozumienia lub też nie, umowa zostaje zakończona(4a).

\begin{figure}[h!]
\centering
\includegraphics[width=0.45\textwidth]{A1_2.png}
\caption{Analiza umowy 1.2}
\label{A1_2}
\end{figure}

Umowa A1.2 jest ewolucją umowy A1.1. W tym przypadku sprzedający dodatkowo ustala zaliczkę oraz czas pozwalający na wycofanie się kupującego. Kolejną ważną cechą jest też możliwość ocenienia sprzedającego przez kupującego. W powyższej umowie dochodzi jedna dodatkowa alternatywna droga do zakończenia umowy. Kupujący może zrezygnować z kontynuowania procesu i może zażądać zwrotu wpłaconych pieniędzy, jednak może to zrobić tylko przed terminem pozwalającym na wycofanie się.

\subsection{Projektowanie back-end'u}
\subsection{Lista zmian}

\newpage
\section{Przyrost drugi}
\subsection{Cele przyrostu}
Drugi przyrost to przede wszystkim implementacja pierwszej umowy A1.1 oraz AgreementManagera, odpowiedzialnego za rejestrowanie i tworzenie umów w systemie.
Następnym istotnym celem było stworzenie kontraktu wewnętrznego tokena, który posiada możliwość wpłacania i wypłacania ethereu oraz przekazywania go do umów.
W przyroście drugim została również przeprowadzona analiza zachęt do usuwania przetrminowanych umów.
Również zostały przeprowadzona analiza scenariuszy innych bardziej złożonych typów umów.

\subsection{Analiza front-end'u}
\subsection{Projektowanie front-end'u}
\subsection{Testy}
\subsubsection{Środowisko testów}
Aplikacja była testowana za pomocą środowiska developerskiego truffle w którym testy tworzoyny za pomocą javascriptowego środkowska testowego Mocha oraz Chai do tworzenia assercji.
Do testów potrzebujemy środowisko w którym możemy uruchomić kontrakty. Do tych zadań używamy środowiska Ganache, który jest personalnym blockchainenm gdzie możemy uruchamiać kontrakty i testować je.
\subsubsection{Scenariusze testowe}
\begin{table}[H]
    \begin{tabular}{|l|l|}
\hline
    	\multicolumn{1}{|m{6,5cm}|}{\textit{Umowa A1.1: Standardowa ścieżka}}                 & \multicolumn{1}{m{9cm}|}{\textit{Sprawdzamy stworzenie umowy A1.1 z daną ceną w początkowym stanie. Użytkownik wpłaca pieniądze do kontraktu. Twórca wypłaca pieniądze. Każdy z użytkowników zamyka umowę. Umowa zostaje zakończona.}} \\ \hline
    	\multicolumn{1}{|m{6,5cm}|}{\textit{Umowa A1.1: właściwości dołączania}} & \multicolumn{1}{m{11cm}|}{\textit{Sprawdzamy różne scenariusze dołączania do umowy. Sprawdzamy czy umowa posiada do trzech maksymalnie użytkowników. Sprawdzamy czy twórca lub suplikant nie próbuje dołączyć dwa razy do tej samej umowy.}}    \\ \hline
    	\multicolumn{1}{|m{6,5cm}|}{\textit{Umowa A1.1: ograniczenia akecptacji użytkowników}} & \multicolumn{1}{m{9cm}|}{\textit{Sprawdzamy możliwość akceptacji użytkowników. Sprawdzamy czy twórca nie próbuje akceptować samego siebie do umowy. Sprawdzamy, czy tylko twórca ma możliwość akceptacji próśb o dołączenie do umowy. Sprawdzamy również czy nie ma możliwości akceptacji próśb, gdy umowa jest w trakcie realizacji.}}                                                \\ \hline
   	\multicolumn{1}{|m{6,5cm}|}{\textit{Umowa A1.1: właściwości akceptacji zależne od stanu}} & \multicolumn{1}{m{9cm}|}{\textit{Sprawdzamy czy twórca nie ma możliwości podwójnie zaakceptować prośby o dołączenie od tej samej osoby. Sprawdzamy również, czy nie ma możliwości dołączania do umowa gdy jest ona zamknięta.}}                                                \\ \hline
	\multicolumn{1}{|m{6,5cm}|}{\textit{Umowa A1.1: właściwości zamknięcia umowy}} & \multicolumn{1}{m{9cm}|}{\textit{Testujemy, czy obce adresy nie mają możliwości zakończenia umowy. Sprawdzamy również, czy niezaakceptowany adres nie ma możliwości zamknięcia umowy oraz czy nie ma możliwości podówjnie zakończyć umowy.}} \\ \hline
	\multicolumn{1}{|m{6,5cm}|}{\textit{Interkacja umowy z funkcją remove}}    & \multicolumn{1}{m{9cm}|}{\textit{Sprawdzamy czy różne funkcje takie jak join czy accept nie mają wpływu na usunięcie całego kontratktu. Sprawdzamy równiez czy nie ma możliwości usunąć umowy gdy jest w stanie realizacji.}} \\ \hline
	\multicolumn{1}{|m{6,5cm}|}{\textit{Umowa A1.1: właściwości wycofania}}  & \multicolumn{1}{m{9cm}|}{\textit{W tym scenraiuszu testujemy wycofywanie się użytkownika z umowy. Sprawdzamy, czy sprzedawca nie może wycofać się z umowy. Wraz z wycofywaniem się, sprawdzamy czy jest możliwość wypłacenia tokenów. Sprawdzamy czy po staniu się kupującym nie może wycofać się z umowy.}}                                          \\ \hline
	\multicolumn{1}{|m{6,5cm}|}{\textit{AgreementManager: Tworzenie umów}}  & \multicolumn{1}{m{9cm}|}{\textit{Sprawdzamy czy AgreementManager pomyślnie tworzy umowę. Sprawdzamy również czy umowa istnieje pod danym adresem.}}                                          \\ \hline
	\multicolumn{1}{|m{6,5cm}|}{\textit{AgreementManager: Właściwości wyszukiwania}}  & \multicolumn{1}{m{9cm}|}{\textit{Sprawdzamy właściwości wyszukiwania AgreementManagera. Sprawdzamy czy AgreementManager pomyślnie wyszukuje pomyślnie daną umowę.}}                                          \\ \hline
\end{tabular}
\end{table}

\subsection{Implementacja}
\subsection{Lista zmian}

\newpage
\section{Przyrost trzeci}
\subsection{Cele przyrostu}

Głównym zadaniem było stworzenie UI do wygodnego wyszukiwania umów oraz różnych widoków potrzebnych do tworzenia, czy też przeglądania umów.
W trzecim przyroście również postawiliśmy sobie za zadanie przebudowanie wewnętrzengo tokena do standardu
obowiązującego w środowisku Ethereum (ERC20).
W tym przyroście również przygotowaliśmy system do dodania umowy A1.2.

\subsection{Napotkane problemy}

Na tym etapie podczas tworzenia implementacji umów, napotkaliśmy problem natury technicznej.
Każda umowa, którą zaprojektowaliśmy ma różne wspólne cechy, które zgodnie z zasadami dobrego programowania, powinny być umieszczone w osobnych, uogólnionych komponentach, tak by zapewnić hermetyzację i uniknąć powtarzania kodu.
Język programowania, którego używamy, wspiera dziedziczenie, jednak do zamodelowania umów bardziej adekwatne było użycie wzorca projektowgo \textit{Maszyna Stanowa}, który opiera się o agregację. Kontratky jednak nie wspierają agregacji, nie można stworzyć instancji kontraktu w kontrakcie, tak by ten wewnętrzny nie był wyeksponowany i dostępny dla potencjalnie złośliwych aktorów. Konieczne było więc opracowanie autorskiego rozwiązania, które realizuje ten wzorzec, z uwzględnieniem faktu, że stworzeni pojedynczego kontraktu kosztuje relatywnie dużą ilośc gazu.

\subsection{Projektowanie back-end'u}
\subsection{Projektowanie front-end'u}
\subsection{Testy}
\subsection{Scenariusze testowe}
\begin{table}[H]
    \begin{tabular}{|l|l|}
\hline
    \multicolumn{1}{|m{6,5cm}|}{\textit{Standard ECM Token: transfer}}                 & \multicolumn{1}{m{11cm}|}{\textit{Sprawdzamy różne scenariusze przepływu tokenów pomiędzy kontami. Sprawdzamy poprawność transakcji.Sprawdzamy również czy nie ma możliwości wysyłać tokenów do adresu danego tokenu.}} \\ \hline
    \multicolumn{1}{|m{6,5cm}|}{\textit{Standard ECM Token: wpłata oraz wypłata pieniędzy}} & \multicolumn{1}{m{11cm}|}{\textit{Sprawdzamy poprawność wpłat i wypłat tokenów. Funkcje testowane są również na kilku użytkownikach oraz samego zdarzenia transferu.}}    \\ \hline
    \multicolumn{1}{|m{6,5cm}|}{\textit{Maszyna stanowa}} & \multicolumn{1}{m{11cm}|}{\textit{W tym scenariuszu sprawdzamy poprawaność przejścia stanu.}}                \\ \hline
    \multicolumn{1}{|m{6,5cm}|}{\textit{Maszyna stanowa z pamięcią}} & \multicolumn{1}{m{11cm}|}{\textit{W tym scenariuszu sprawdzamy poprawność zapisywania różnych stanów w pamięci}}                                                \\ \hline
    \multicolumn{1}{|m{6,5cm}|}{\textit{Obiekt Storage}} & \multicolumn{1}{m{11cm}|}{\textit{Testujemy inicjowanie obiektu Storage}} \\ \hline
    \multicolumn{1}{|m{6,5cm}|}{\textit{Zakres pamięci}}    & \multicolumn{1}{m{11cm}|}{\textit{Testujemy zachowanie pamięci poza zakresem.}} \\ \hline
    \multicolumn{1}{|m{6,5cm}|}{\textit{Pamięć: podstawy działania}}    & \multicolumn{1}{m{11cm}|}{\textit{Sprawdzamy poprawność funkcji, których celem jest zapisywanie do danego slotu w pamięci.}} \\ \hline
                                    
\end{tabular}
\end{table}
\subsection{Testy systemowe}
\subsection{Implementacja}
\subsection{Lista zmian}

\newpage
\section{Instrukcja obsługi}

\newpage
\section{Prezentacja projektu}

\newpage
\section{Raport końcowy}

\newpage
\section{Całkowity nakład pracy}

\newpage
\section{Wkład własny w projekt}

\newpage
\section{Podsumowanie}

\newpage
\section{Słownik pojęć}
\begin{itemize}
	\item Platforma Handlowa --- aplikacje agregujące oferty handlowe, pośredniczące w wykonaniu opłat i dostarczające narzędzia od oceny kupujących i sprzedających. 
	\item Ekosystem Ethereum (zespół technologii Ethereum) --- Ethereum jako środowisko wykonawcze oparte na blockchainie, rozproszona przestrzeń dyskowa Swarm, system komunikatów Whisper, biblioteki web3 i aplikacje klienckie Ethereum.
	\item  Ethereum --- środowisko wykonawcze oparte na blockchainie dla inteligentnych kontraktów.
	\item  Blockchain --- struktura danych mogąca przechowywać dowolne dane w postaci bloków, skonstruowana na zasadzie listy (lub w pewnym stopniu drzewa), gdzie wskaźnikami do poprzednich bloków są ich kryptograficzne hasze.
	\item Drzewa Merkle --- struktura danych mogąca przechowywać dowolne dane, skonstruowana na zasadzie drzewa, gdzie wskaźnikami na rodziców są ich kryptograficzne hasze.
	\item Mainnet --- główna sieć Ethereum o id równym 1. Na niej odbywają się transakcje z użyciem etheru posiadającym wartość dla użytkowników.
	\item Swarm --- zdecentralizowana przestrzeń dyskowa oparta na drzewach Merkle, zintegrowana z Ethereum. 
	\item Whisper --- zdecentralizowany, wysoce anonimowy system komunikatów zintegrowany z Ethereum.
	\item Wirtualna Maszyna Ethereum --- część Ethereum wykonująca bytecode skompilowanych kontraktów.
	\item Ether --- jednostka płatności/możliwości wykonania kodu. Jest podstawową walutą w ekosystemie Ethereum. 1 ether dzieli się na $10$\textsuperscript{18} wei.
\end{itemize}

\newpage
\section{Załączniki}

\newpage
\bibliographystyle{IEEEtranS}
\bibliography{bibliografia}
\end{document}